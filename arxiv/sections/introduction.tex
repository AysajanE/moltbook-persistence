\section{Introduction}
\label{sec:introduction}

The rapid advancement of large language models (LLMs) has enabled a new class of autonomous AI agents capable of sustained interaction with digital environments. One manifestation of this capability is \emph{Moltbook}, a social network launched in January 2026 that restricts posting privileges to AI agents while permitting human observation \citep{willison2026moltbook}. In this paper, an \emph{AI agent account} denotes an account whose posting and commenting actions are generated by an LLM-driven agent process rather than direct human operation. In the first archived week, the platform accumulated over 25{,}000 such agent accounts and 119{,}677 posts \citep{simulamet2026observatoryarchive}, generating a large, accessible dataset of agent-to-agent interaction.

This emergence of agent-populated social platforms raises fundamental questions about collective AI behavior. Can autonomous agents sustain the extended, multi-turn dialogues necessary for meaningful collaboration? How do architectural constraints---particularly context-window limitations and periodic activation schedules---shape the temporal dynamics of agent discourse? And what distinguishes agent-driven conversations from human-driven ones in structural and temporal terms?

\subsection{Motivation and Research Questions}

Early observations of Moltbook suggested a pattern: agents appear substantially better at \emph{initiating} projects than \emph{sustaining} them \citep{alexander2026afterweekend}. Threads that begin with ambitious coordination proposals---collaborative research, collective governance, creative projects---often stall within days \citep{alexander2026afterweekend}. This pattern suggests that agent social networks may face intrinsic \emph{persistence limitations} stemming from the temporal constraints under which individual agents operate.

We focus on a specific architectural feature that may explain these dynamics: the \emph{heartbeat mechanism}. Moltbook agents are typically configured to check the platform at regular intervals (approximately every four hours), creating a mechanically induced ``attention clock'' \citep{willison2026moltbook}. In observatory data, however, per-account heartbeat schedules are latent: we observe timestamped posts/comments but not direct scheduler logs. We therefore treat heartbeat effects as a testable aggregate hypothesis: sufficiently synchronized check-ins should yield spectral concentration near 4 hours, whereas substantial dephasing or jitter can make the same mechanism weakly detectable or undetectable in aggregate over finite windows. This framing is paired with finite context windows, which motivate separate tests of rapid within-thread staleness.

Our central research questions are how quickly direct-reply responsiveness
decays on Moltbook (conversation persistence, operationalized via reply hazard)
and whether aggregate activity exhibits a detectable $\sim$4-hour periodic signature;
which structural properties (depth, branching, reciprocity) characterize Moltbook
discussion trees relative to human-platform baselines; and which factors---topic
domain, agent reputation, and early engagement---are associated with extended
conversational persistence. Thread duration is analyzed separately as a distinct
outcome.

\subsection{Hypotheses}
\label{sec:introduction:hypotheses}

Guided by the horizon-limited cascade framework (\Cref{sec:model}) and prior
qualitative observations \citep{willison2026moltbook, alexander2026afterweekend},
we test four hypotheses. H1a posits short reply-kernel half-lives consistent with
architectural staleness constraints. H1b posits that heartbeat scheduling can
generate aggregate periodic structure near the hypothesized cadence
($\tau \approx 4$ hours) when check-ins are sufficiently synchronized; under
dephasing or jitter, aggregate detectability may be weak in finite samples.
H2 posits shallower, more root-concentrated Moltbook trees than human-platform
baselines, with lower reciprocity and conditioning-sensitive re-entry profiles; the
re-entry contrast is treated as conditioning-sensitive and may change direction
across overlap-restricted matched strata.
H3 posits topic-level moderation of persistence, including systematic differences
in half-life and depth across submolts. H4 posits that agent-level covariates
(account claim status and follower count) are associated with variation in reply
incidence and conversational persistence.

We operationalize these hypotheses via the metrics and estimators described in \Cref{sec:methods} and evaluate them in \Cref{sec:results}.
For H1b, non-significant evidence at the target frequency is interpreted as
``not detected in this snapshot'' rather than as evidence of absence.

\subsection{Preview of Findings}

In this first-week snapshot, Moltbook conversations are predominantly star-shaped,
with minute-scale reply-kernel decay, low direct-reply incidence, and minimal
reciprocity. Spectral analysis does not detect a statistically significant
4-hour periodic peak. A run-scoped Reddit baseline shows materially longer
persistence and deeper threads. Taken together, these patterns are consistent
with a ``low-incidence/fast-conditional-response'' regime driven by architectural constraints
on agent attention.

\subsection{Approach and Contributions}

We develop a \emph{horizon-limited interaction cascade} model that formalizes conversation dynamics on agent platforms. Our framework combines self-exciting point processes (Hawkes processes) with age-dependent branching dynamics, explicitly incorporating periodic availability modulation to capture the heartbeat mechanism. The model provides expressions relating platform design parameters to observable quantities: interaction half-life, maximum-depth tail behavior, and agent re-entry rates.

We evaluate this framework empirically with a Moltbook-first design using the Moltbook
Observatory Archive \citep{simulamet2026observatoryarchive}. A run-scoped curated Reddit
corpus is used as secondary contextual baseline, not as a gating causal comparison. The
present manuscript reports: (1)~descriptive characterization of Moltbook conversation
geometry, (2)~estimation of Moltbook temporal decay parameters via survival analysis,
(3)~spectral tests for Moltbook periodic activity signatures, (4)~a full-scale Reddit-side
baseline analysis under the same estimators, and (5)~a coarse matched observational
comparison with paired effect estimation.

Our contributions are fourfold. First, we introduce interaction half-life as a
portable metric of collective persistence that is estimable from timestamped
reply data. Second, we provide mechanism-consistent interpretations linking shallow
conversation structure to measurable temporal decay under architectural
constraints. Third, we discuss design levers---including memory scaffolding,
thread summarization, and return-to-thread incentives---that may extend
coordination horizons in agent systems. Fourth, we provide a reproducible
research workflow that makes the empirical claims auditable and extensible.

\subsection{Paper Organization}

The remainder of this paper is organized as follows. \Cref{sec:background} reviews related work on information cascades, conversation modeling, and emerging research on AI agent systems. \Cref{sec:model} presents our formal model of horizon-limited interaction cascades. \Cref{sec:data} describes our datasets and preprocessing pipeline. \Cref{sec:methods} details our empirical methodology. \Cref{sec:results} presents our findings. \Cref{sec:discussion} interprets results and discusses implications for platform design. \Cref{sec:limitations} addresses limitations and ethical considerations. \Cref{sec:conclusion} concludes with key takeaways and directions for future work. \Cref{sec:reproducibility} summarizes reproducibility details (Appendix A), and \Cref{sec:appendix} provides supplementary derivations and robustness material (Appendix B).
