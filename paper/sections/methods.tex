\section{Methods}
\label{sec:methods}

We estimate conversation geometry, reply-kernel decay, periodic signatures, and
coarse matched cross-platform contrasts using a common set of deterministic
preprocessing rules and fixed inferential settings.

\subsection{Conversation Geometry}
\label{sec:methods:geometry}

For each thread \(j\), we compute node depths from \cref{eq:depth}, record the
maximum depth \(D_j\), and summarize the empirical depth distribution with mean,
median, and tail probabilities \(\Prob(D_j \ge k)\) for \(k=1,\ldots,10\).
Using the model implication in \cref{prop:depth-bound}, we estimate an effective
depth-tail parameter \(\hat{\mu}\) by zero-intercept least squares on
\(\log \Prob(D_j \ge k)\). Because the analysis conditions on threads with at
least one comment, \(\Prob(D_j \ge 1)=1\) by construction; the fit is therefore
identified by \(k\ge2\), and \(\hat{\mu}\) is interpreted as an effective
non-root reproduction parameter (\cref{rem:two-type}).

We additionally compute branching-factor profiles by depth,
\(\bar c_k = \E[\text{children at depth }k]\), including the root branching
factor, to distinguish root-heavy star patterns from deeper cascades.

Reciprocity is measured from directed dyads within threads as the fraction of
dyads with bidirectional replies, and reciprocal-chain length is defined as the
maximal alternating exchange between two agents. Re-entry is measured by
\(\mathrm{RE}_j\) in \cref{eq:reentry-rate}, with thread-level distributions and
thread-size-conditioned summaries.

\subsection{Interaction Half-Life Estimation}
\label{sec:methods:halflife}

Our primary temporal estimand is the decay rate \(\beta\), reported as
reply-kernel half-life \(h=\ln 2 / \beta\). For each parent comment \(m\) in
thread \(j\), we define first-direct-reply survival time
\begin{equation}
\label{eq:survival-time}
S_{jm} := \min\{t_{jn} - t_{jm} : p_{jn}=m,\; n>m\}.
\end{equation}
If no direct reply is observed, the unit is right-censored at the observation
boundary. Because the canonical timeline contains a 41.72-hour coverage gap,
we do not impute unobserved replies across that interval; half-life and duration
quantities are interpreted as conditional on observed coverage.

Under \cref{eq:reply-intensity}, the survival hazard for parent age \(s\) is
\begin{equation}
\label{eq:hazard}
\lambda(s \mid t_{jm}) = b(t_{jm}+s)\,\alpha_{a_{jm}}\,e^{-\beta_{a_{jm}}s}.
\end{equation}

\begin{remark}[Estimand interpretation]
\label{rem:estimand}
The \emph{reply-kernel half-life} \(\hat h = \ln 2/\hat\beta\) is a kernel-decay
timescale for direct-reply hazard. It is not a median thread lifetime.
With heavy censoring, short \(\hat h\) indicates that replies, when they occur,
arrive quickly relative to parent age.
\end{remark}

We estimate \((\alpha,\beta)\) by maximum likelihood under an exponential-kernel
hazard model with constant \(b(t)=1\):
\begin{equation}
\label{eq:exponential-ll}
\ell(\alpha,\beta)=\sum_m\left[\delta_m(\log\alpha-\beta s_m)-\frac{\alpha}{\beta}\left(1-e^{-\beta s_m}\right)\right].
\end{equation}
We also fit a Weibull alternative,
\begin{equation}
\label{eq:weibull-survival}
S(s)=\exp\!\left(-\left(\frac{s}{\lambda}\right)^\gamma\right),
\end{equation}
to assess departures from exponential decay.

Given high censoring, we report a decomposition of (i) observed in-window
reply probability \(p_{\mathrm{obs}}\), (ii) conditional median
first-reply time among observed events, and (iii) model-implied eventual reply
probability \(p_\infty=1-\exp(-\hat\alpha/\hat\beta)\) as a diagnostic quantity.
We further report stratified pooled estimates by submolt category and agent
proxies (claim status and follower-count bins).

Uncertainty is quantified with 95\% thread-cluster bootstrap intervals.

\subsection{Periodicity Detection}
\label{sec:methods:periodicity}

To test for heartbeat-scale periodicity, we bin comment counts at 15-minute
resolution in the longest contiguous segment after splitting the timeline at
gaps larger than six hours. Let \(C_t\) denote binned counts; we transform and
preprocess as
\(Y_t=\log(C_t+1)\), 24-hour moving-average detrending, and Hanning windowing.

We estimate power spectral density via Welch's method and test the target
frequency \(f_\tau = 1/\tau \approx 0.25\,\mathrm{hr}^{-1}\) against an AR(1)
red-noise null using 2,000 Monte Carlo simulations for Fisher's \(g\) and
target-frequency power statistics. Robustness checks repeat target-frequency and
dominant-frequency analyses at 5-, 15-, and 30-minute bin widths on the same
contiguous segment.

At agent level, we compute lagged autocorrelation on 15-minute activity series
for agents with at least 10 comments in the contiguous segment and report the
mean lag-4-hour autocorrelation with bootstrap confidence intervals.

\subsection{Cross-Platform Comparison}
\label{sec:methods:comparison}

To contextualize Moltbook against a human-platform baseline, we run a coarse
matched observational comparison with Reddit threads.

Matching uses coarsened exact strata on three controls: first-30-minute
action volume (bins \(\{0, 1\text{--}2, 3\text{--}5, 6\text{--}10, 11+\}\)),
a deterministic coarse topic map (Moltbook and Reddit categories projected to
\texttt{tech}/\texttt{meta}/\texttt{general}/\texttt{spam}), and exact UTC posting
hour \((0,\ldots,23)\). Within each shared stratum, threads are sorted
deterministically and paired one-to-one up to \(\min(n_M,n_R)\).

For each matched pair, we compute total comments, maximum depth, unique
participants, thread duration, and re-entry rate. Reply-kernel half-life is
also estimated on matched-thread subsets for each platform (platform-level
estimation, not thread-level paired survival effects).

Inference uses two-sided Wilcoxon signed-rank tests on paired differences,
paired Cohen's \(d\), and bootstrap 95\% confidence intervals for mean paired
differences (1,000 matched-pair resamples). Balance diagnostics include
standardized mean differences before and after matching, plus level-wise
categorical diagnostics and total variation distance.

All analyses are run in Python with fixed seeds and deterministic preprocessing;
manuscript-level reproducibility metadata are provided in
\Cref{sec:reproducibility}.
