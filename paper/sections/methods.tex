\section{Estimation and Empirical Procedures}
\label{sec:methods}

Estimands are defined in \Cref{sec:model:estimands}. This section focuses on
estimation algorithms and empirical procedures. Low-level operational settings
(binning, detrending, bootstrap mechanics, and matching construction) are
reported in \Cref{sec:appendix:operational-details}.

\subsection{Conversation Geometry}
\label{sec:methods:geometry}

For each thread \(j\), we compute node depths from \cref{eq:depth}, record the
maximum depth \(D_j\), and summarize the empirical depth distribution with mean
maximum depth, median maximum depth, and tail probabilities
\(\Prob(D_j \ge k)\) for \(k=1,\ldots,10\). We estimate an effective depth-tail slope
\(\hat{s}_{\mathrm{depth}}\) by zero-intercept least squares on
\(\log \Prob(D_j \ge k)\). Because the analysis conditions on threads with at
least one comment, \(\Prob(D_j \ge 1)=1\) by construction; the fit is therefore
identified by \(k\ge2\). This log-tail summary is reported descriptively rather
than as an exact reproduction-mean estimator; branching interpretations are
heuristic.

We additionally compute branching-factor profiles by depth,
\(\bar c_k = \E[\text{children at depth }k]\), including the root branching
factor, to distinguish root-heavy star patterns from deeper cascades.

Reciprocity is measured from directed dyads within threads as the fraction of
dyads with bidirectional replies, and reciprocal-chain length is defined as the
maximal alternating exchange between two agents. Re-entry is measured by
\(\mathrm{RE}_j\) in \cref{eq:reentry-rate}. Missing-author-identifier handling rules
are deterministic and documented in \Cref{sec:appendix:operational-details}.

\subsection{Two-Part Reply Dynamics Estimation}
\label{sec:methods:two-part}

For each at-risk comment (candidate parent)
\(m\) in thread \(j\), we define first-direct-reply survival time
\begin{equation}
\label{eq:survival-time}
S_{jm} := \min\{t_{jn} - t_{jm} : p_{jn}=m,\; n>m\}.
\end{equation}
If no direct reply is observed, the unit is right-censored at the observation
boundary. Because the canonical timeline contains a 41.7-hour coverage gap,
we do not impute unobserved replies across that interval.

\paragraph{Part 1: incidence model.}
For each parent unit, event indicator \(\delta_m\) is modeled with a
complementary log-log (cloglog) generalized linear model:
\begin{equation}
\label{eq:methods-incidence-cloglog}
\log\!\left[-\log(1-\Prob(\delta_m=1\mid x_m))\right]=x_m^\top\eta,
\end{equation}
which aligns with a discrete-time hazard interpretation and the asymmetric
tail behavior of rare-event incidence. Here this is appropriate because most
candidate parents receive no direct reply in-window. Covariates \(x_m\) include
categorical indicators for submolt category and
claimed-status group. Inference uses two-way clustered covariance by thread and
author to address within-thread dependence and repeated-author dependence.

\paragraph{Part 2: conditional timing model.}
Among replied parents \((\delta_m=1)\), we report empirical conditional-time
estimands directly:
\(\tilde s_{0.5}\), \(\tilde s_{0.9}\), \(\tilde s_{0.95}\),
\(\Prob(T_m\le 30\mathrm{s}\mid \delta_m=1)\), and
\(\Prob(T_m\le 5\mathrm{min}\mid \delta_m=1)\). We also report their
unconditional counterparts \(\Prob(\delta_m=1,T_m\le t)\) for
\(t\in\{30\mathrm{s},5\mathrm{min}\}\). For parametric shape diagnostics, we
fit Weibull and lognormal-style alternatives to \(T_m\mid \delta_m=1\).

For kernel diagnostics, we fit an exponential-kernel hazard with \(b(t)=1\) as
a timescale-separation approximation for identifying \(\beta\), while periodic
modulation is tested separately at the aggregate level
(\Cref{sec:methods:periodicity}). The full continuous-time intensity and
competing-risks parent-selection formalization are deferred to supplementary
material (Section S0).

\begin{remark}[Estimand interpretation]
\label{rem:estimand}
The \emph{kernel half-life diagnostic} \(\hat h = \ln 2/\hat\beta\) is an
exponential-equivalent kernel-decay
timescale for direct-reply hazard. It is not a median thread lifetime.
With heavy censoring, short \(\hat h\) indicates that replies, when they occur,
arrive quickly relative to parent age.
\end{remark}

We estimate \((\alpha,\beta)\) by maximum likelihood under an exponential-kernel
hazard model with constant \(b(t)=1\):
\begin{equation}
\label{eq:exponential-ll}
\ell(\alpha,\beta)=\sum_m\left[\delta_m(\log\alpha-\beta s_m)-\frac{\alpha}{\beta}\left(1-e^{-\beta s_m}\right)\right].
\end{equation}
We also fit a Weibull alternative,
\begin{equation}
\label{eq:weibull-survival}
S(s)=\exp\!\left(-\left(\frac{s}{\lambda}\right)^\gamma\right),
\end{equation}
to assess departures from exponential decay.

Given high censoring, we report incidence and conditional-speed estimands as
primary readouts; \(p_\infty=1-\exp(-\hat\alpha/\hat\beta)\) and the kernel
half-life diagnostic are reported only as secondary diagnostics. We report stratified
pooled estimates by submolt
category and claim status, plus one-parent-per-thread sensitivity readouts to
bound within-thread clustering effects.

Uncertainty is quantified with thread-cluster bootstrap confidence intervals
using fixed deterministic resampling settings (\Cref{sec:appendix:operational-details}).

\subsection{Periodicity Detection}
\label{sec:methods:periodicity}

Because the canonical timeline contains a 41.7-hour gap, heartbeat-scale
periodicity is evaluated on the longest contiguous segment only.

The primary inferential test is event-time modulo-\(\tau\) circular uniformity
on raw timestamps (with \(\tau=4\) hours). Define
\(\theta_n := 2\pi((t_n \bmod \tau)/\tau)\), resultant
\(R=\left|\frac{1}{N}\sum_{n=1}^N e^{i\theta_n}\right|\), and Rayleigh statistic
\(Z=NR^2\). We report \(R\), \(Z\), Monte Carlo \(p\)-value, and mean phase.

To address detectability explicitly, we estimate a power curve over phase
concentration amplitudes \(\kappa\) using the observed sample size \(N\) and the
same Monte Carlo calibration procedure. We then report
\(\kappa^\star:=\inf\{\kappa:\widehat{\mathrm{power}}(\kappa)\ge0.8\}\), the
minimum concentration amplitude detectable with at least 80\% power under this
window length and event count.

PSD/AR(1) and bin-width diagnostics are retained as supplementary checks
(\Cref{sec:appendix:periodicity-robustness}) rather than primary inferential
tests in the main manuscript.

\subsection{Cross-Platform Comparison}
\label{sec:methods:comparison}

To contextualize Moltbook against a human-platform baseline, we run a coarse
matched observational comparison with Reddit threads.

Matching uses coarsened exact strata on early engagement, coarse topic, and UTC
posting hour, then deterministic one-to-one pairing within shared strata.
Exact strata definitions and pairing rules are documented in
\Cref{sec:appendix:operational-details}.

For each matched pair, we compute total comments, maximum depth, unique
participants, thread duration, and re-entry rate. Kernel half-life diagnostics are
also estimated on matched-thread subsets for each platform (platform-level
estimation, not thread-level paired survival effects).

Inference uses two-sided Wilcoxon signed-rank tests on paired differences,
paired Cohen's \(d\), and bootstrap 95\% confidence intervals for mean paired
differences. Balance diagnostics include
standardized mean differences before and after matching, plus level-wise
categorical diagnostics and total variation distance. Resampling mechanics are
reported in \Cref{sec:appendix:operational-details}.

All analyses are run in Python with fixed seeds and deterministic preprocessing;
manuscript-level reproducibility metadata are provided in
\Cref{sec:reproducibility}.
