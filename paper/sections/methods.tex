\section{Estimation and Empirical Procedures}
\label{sec:methods}

Estimands are defined in \Cref{sec:model:estimands}. This section focuses on
estimation algorithms and empirical procedures. Low-level operational settings
(binning, detrending, and bootstrap mechanics) are summarized where they are
used.

\subsection{Conversation Geometry}
\label{sec:methods:geometry}

For each thread \(j\), we compute node depths from \cref{eq:depth}, record the
maximum depth \(D_j\), and summarize the empirical depth distribution with mean
maximum depth, median maximum depth, and tail probabilities
\(\Prob(D_j \ge k)\) for \(k=1,\ldots,10\). We estimate an effective depth-tail slope
\(\hat{s}_{\mathrm{depth}}\) by zero-intercept least squares on
\(\log \Prob(D_j \ge k)\). Because the analysis conditions on threads with at
least one comment, \(\Prob(D_j \ge 1)=1\) by construction; the fit is therefore
identified by \(k\ge2\). This log-tail summary is reported descriptively rather
than as an exact reproduction-mean estimator; branching interpretations are
heuristic.

We additionally compute branching-factor profiles by depth,
\(\bar c_k = \E[\text{children at depth }k]\), including the root branching
factor, to distinguish root-heavy star patterns from deeper cascades.

Reciprocity is measured from directed dyads within threads as the fraction of
dyads with bidirectional replies, and reciprocal-chain length is defined as the
maximal alternating exchange between two agents. Re-entry is measured by
\(\mathrm{RE}_j\) in \cref{eq:reentry-rate}. Missing-author-identifier handling rules
are deterministic.

\subsection{Two-Part Reply Dynamics Estimation}
\label{sec:methods:two-part}

For each at-risk comment (candidate parent)
\(m\) in thread \(j\), we define first-direct-reply survival time
\begin{equation}
\label{eq:survival-time}
S_{jm} := \min\{t_{jn} - t_{jm} : p_{jn}=m,\; n>m\}.
\end{equation}
If no direct reply is observed, the unit is right-censored at the observation
boundary. Because the canonical timeline contains a 41.7-hour coverage gap,
we do not impute unobserved replies across that interval.
To assess whether this gap can artifactually inflate fast conditional-response
signals, we run four robustness checks: gap-disambiguation diagnostics across
raw archive tables, contiguous-window recomputation (pre-gap and post-gap),
gap-overlap exclusions (\(X\in\{6,24\}\) hours before gap start), and
horizon-standardized \(\Prob(\delta=1,T\le t)\) at
\(t\in\{30\mathrm{s},5\mathrm{min},1\mathrm{h}\}\) using explicit risk sets.
We report these robustness checks in the Results section.

\paragraph{Part 1: incidence model}
Primary incidence readouts are horizon-standardized probabilities at fixed
follow-up windows \(h\in\{5\mathrm{min},1\mathrm{h}\}\) using the
risk-set estimator defined in \Cref{sec:model:estimands}. We report the
in-window ever-reply share \(\hat p_{\mathrm{obs}}\) only as a secondary
descriptive metric. Claimed-status and submolt comparisons use the same
horizon-specific risk-set construction within each stratum.

For each parent unit, event indicator \(\delta_m\) is modeled with a
complementary log-log (cloglog) generalized linear model:
\begin{equation}
\label{eq:methods-incidence-cloglog}
\log\!\left[-\log(1-\Prob(\delta_m=1\mid x_m))\right]=x_m^\top\eta,
\end{equation}
which aligns with a discrete-time hazard interpretation and the asymmetric
tail behavior of rare-event incidence. Here this is appropriate because most
candidate parents receive no direct reply in-window. Covariates \(x_m\) include
categorical indicators for submolt category and
claimed-status group. Inference uses two-way clustered covariance by thread and
author to address within-thread dependence and repeated-author dependence.

\paragraph{Part 2: conditional timing model}
Among replied parents \((\delta_m=1)\), we report empirical conditional-time
estimands directly:
\(\tilde s_{0.5}\), \(\tilde s_{0.9}\), \(\tilde s_{0.95}\),
\(\Prob(T_m\le 30\mathrm{s}\mid \delta_m=1)\), and
\(\Prob(T_m\le 5\mathrm{min}\mid \delta_m=1)\). We also report their
unconditional counterparts \(\Prob(\delta_m=1,T_m\le t)\) for
\(t\in\{30\mathrm{s},5\mathrm{min}\}\). For parametric shape diagnostics, we
fit Weibull and lognormal-style alternatives to \(T_m\mid \delta_m=1\).

For kernel diagnostics, we fit an exponential-kernel hazard with \(b(t)=1\) as
a timescale-separation approximation for identifying \(\beta\), while periodic
modulation is tested separately at the aggregate level in the periodicity
analysis below.

\begin{remark}[Estimand interpretation]
\label{rem:estimand}
The \emph{kernel half-life diagnostic} \(\hat h = \ln 2/\hat\beta\) is an
exponential-equivalent kernel-decay
timescale for direct-reply hazard. It is not a median thread lifetime.
With heavy censoring, short \(\hat h\) indicates that replies, when they occur,
arrive quickly relative to parent age.
\end{remark}

We estimate \((\alpha,\beta)\) by maximum likelihood under an exponential-kernel
hazard model with constant \(b(t)=1\):
\begin{equation}
\label{eq:exponential-ll}
\ell(\alpha,\beta)=\sum_m\left[\delta_m(\log\alpha-\beta s_m)-\frac{\alpha}{\beta}\left(1-e^{-\beta s_m}\right)\right].
\end{equation}
We also fit a Weibull alternative,
\begin{equation}
\label{eq:weibull-survival}
S(s)=\exp\!\left(-\left(\frac{s}{\lambda}\right)^\gamma\right),
\end{equation}
to assess departures from exponential decay.

Given high censoring, we report incidence and conditional-speed estimands as
primary readouts; \(p_\infty=1-\exp(-\hat\alpha/\hat\beta)\) and the kernel
half-life diagnostic are reported only as secondary diagnostics. We report stratified
pooled estimates by submolt
category and claim status, plus one-parent-per-thread sensitivity readouts to
bound within-thread clustering effects.

Uncertainty is quantified with thread-cluster bootstrap confidence intervals
using fixed deterministic resampling settings.

\subsection{Periodicity Detection}
\label{sec:methods:periodicity}

Because the canonical timeline contains a 41.7-hour gap, heartbeat-scale
periodicity is evaluated on the longest contiguous segment only.
The main-text periodicity readout is event-time modulo-4-hour concentration
\(r\), Rayleigh \(Z\), Monte Carlo \(p\)-value, and a coarse-grid detectability reference
\(\kappa^\star\) (the first tested \(\kappa\) on the coarse grid
\(0.0, 0.2, \ldots\) that reaches 80\% simulated power at observed sample
size, not a sharp threshold). Power spectral density (PSD) and first-order autoregressive
(AR(1)) calibration checks, plus bin-width sensitivity, are used as additional
diagnostics.

\subsection{Reddit Baseline Context}
\label{sec:methods:comparison}

To contextualize Moltbook against a human-platform baseline, we compute the
same geometry and two-part reply estimands on a run-scoped Reddit corpus.

All analyses are run in Python with fixed seeds and deterministic preprocessing;
reproducibility and data-availability details are provided in the required
manuscript statements.
