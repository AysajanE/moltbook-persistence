\section{Discussion}
\label{sec:discussion}

We interpret the empirical patterns in \Cref{sec:results} through the model
framework in \Cref{sec:model}, focusing on collective behavior, platform
design, and identification scope. The evidence combines shallow geometry,
incidence-versus-conditional-timing decomposition, and model-to-data checks;
periodicity and cross-platform analyses provide secondary context.

\subsection{Interpreting Incidence and Conditional Reply Speed}
\label{sec:discussion:two-part}

The primary persistence readout is two-part: low direct-reply incidence (9.60\%)
and very fast conditional reply speed (\(t_{50}=0.076\) minutes,
\(t_{90}=0.834\) minutes, \(t_{95}=2.204\) minutes). Unconditional short-window
probabilities are similarly concentrated (8.47\% within 30 seconds and 9.41\%
within 5 minutes), implying that most observed replies occur almost immediately.
The remaining majority of parents receive no observed direct reply in-window.

This pattern is consistent with finite context retention, rapid task switching,
and weak return-to-thread memory support. Half-life remains useful as a
secondary kernel-timescale diagnostic, but the incidence/conditional-speed split
is the operationally informative summary for coordination limits.

\subsection{Structural Signatures of Limited Persistence}
\label{sec:discussion:structure}

Shallow, root-heavy trees are consistent with a low effective depth-tail slope
(\(\hat{s}_{\mathrm{depth}}=0.154\)) and rapidly decaying depth tails. Under a
heuristic branching interpretation (\cref{sec:model:branching}), this pattern
is compatible with low effective non-root reproduction. Low reciprocity and
modest re-entry in the overall sample further support a broadcast-dominant
interaction pattern, while the matched overlap subset shows that re-entry
direction can change under conditioning.

The model-to-observable validation loop supports this interpretation: predicted
and observed incidence align closely overall and by key strata, but the model
overpredicts non-root branching and depth tails. This indicates that the fitted
timing/incidence dynamics are directionally consistent with observed activity,
while deeper cascade generation remains weaker in the data than in the coarse
branching approximation.

\subsection{Implications for AI Agent Coordination}
\label{sec:discussion:coordination}

These dynamics matter for multi-step coordination tasks. If engagement decays on
minute-scale conditional timescales and repeated participation is limited,
projects requiring
extended deliberation or multi-day follow-through are difficult to sustain
without explicit coordination scaffolds. Updated heterogeneity analyses suggest
meaningful differences by account status in observed reply incidence, but
causal attribution remains out of scope.

\subsection{Design Implications}
\label{sec:discussion:design}

We frame platform design as an operations-research decision-support mapping from
inputs to outputs under explicit trade-offs. The design goal is to increase the
probability of sustained multi-turn coordination (for example, larger
\(\Prob(D_j\ge K)\) for \(K\ge3\)). Candidate levers are mechanism-linked:
memory scaffolding and thread summaries (\(\beta\downarrow\)), resurfacing and
visibility weighting (\(\alpha\uparrow\)), and scheduled re-entry prompts or
cadence alignment (\(b(t)\) phase/coherence). Model-aligned outputs are direct
reply incidence, expected non-root branching
\((\mu\approx\alpha/\beta)\), and depth-tail probabilities.

A worked sensitivity illustrates the scale of these implications under the
current approximation. Using existing fitted overall diagnostics, the baseline
half-life is 0.691 minutes (\Cref{tab:reply-dynamics}) and predicted non-root
branching is 0.104 (\Cref{tab:model-observable-validation}). If \(\beta\) is
reduced by 50\% (equivalently, half-life doubles to
\(2\times 0.691=1.382\) minutes) while holding \(\alpha\) and \(b(t)\) fixed,
then \(\mu\) doubles from 0.104 to 0.208 under
\(\mu\approx\alpha/\beta\). Using the same validation approximation as
\Cref{tab:model-observable-validation} (\(\Pr(D_j\ge3)\approx\mu^2\),
\(\Pr(D_j\ge5)\approx\mu^4\)), predicted depth-tail probabilities become
\(\Pr(D_j\ge3)\approx(0.208)^2\approx 0.0436\) (equivalently
\(4\times 0.0109\)) and
\(\Pr(D_j\ge5)\approx(0.208)^4\approx 0.00192\) (equivalently
\(16\times 0.00012\)). This remains subcritical (\(\mu<1\)) but implies a
materially thicker deep-tail than the current fitted baseline.

The same framework also clarifies trade-offs: higher persistence can increase
notification/computation load, reallocate limited agent attention away from new
threads, and raise spam or strategic-manipulation risk under aggressive
resurfacing. These intervention implications are mechanism-consistent
hypotheses under the fitted model, not experimentally tested causal effects.

\subsection{Broader Implications}
\label{sec:discussion:broader}

Reply incidence and conditional reply speed are portable metrics for comparing
collective persistence across agent communities and model generations; half-life
is a secondary diagnostic for kernel-timescale interpretation. The current cross-platform
results add overlap-region context beyond raw platform contrasts via
deterministic matching, but are not a primary causal identification strategy;
the full matched diagnostics are therefore kept in appendix material.

Periodicity evidence should be read similarly: AR(1)-calibrated PSD tests do not
show a significant 4-hour line, while event-time Rayleigh detects a small but
non-zero phase concentration. This is an effect-size-versus-detectability result
rather than a contradiction.

Stronger causal interpretation still requires richer exposure controls and more
granular semantic alignment. Because Moltbook is rapidly evolving, longitudinal
tracking is necessary to determine whether the persistence gap narrows as agents,
moderation, and interface design mature.
