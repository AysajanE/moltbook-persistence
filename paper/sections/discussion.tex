\section{Discussion}
\label{sec:discussion}

We interpret the empirical patterns in \Cref{sec:results} through the model
framework in \Cref{sec:model}, focusing on collective behavior, platform
design, and identification scope. The evidence combines shallow geometry,
incidence-versus-conditional-timing decomposition, and model-to-data checks;
cross-platform baseline analyses provide secondary context.

\subsection{Interpreting Incidence and Conditional Reply Speed}
\label{sec:discussion:two-part}

The primary persistence readout is two-part: low horizon-standardized incidence
(\(p_{5\mathrm{m}}=9.42\%\), \(p_{1\mathrm{h}}=9.82\%\)) and very fast
conditional reply speed (\(t_{50}=4.55\) seconds, \(t_{90}=50.05\) seconds,
\(t_{95}=132.23\) seconds). Secondary ever-reply incidence is 9.60\% in-window.
Together, these values imply that most parents receive no observed direct reply
even at a one-hour horizon.

This pattern is consistent with finite context retention, rapid task switching,
and weak return-to-thread memory support. The kernel half-life diagnostic remains
useful as a secondary timescale diagnostic, but the incidence/conditional-speed
split is the operationally informative summary for coordination limits.

\subsection{Structural Signatures of Limited Persistence}
\label{sec:discussion:structure}

Shallow, root-heavy trees are consistent with a low effective depth-tail slope
(\(\hat{s}_{\mathrm{depth}}=0.154\)) and rapidly decaying depth tails. Under a
heuristic branching interpretation (\cref{sec:model:branching}), this pattern
is compatible with low effective non-root reproduction. Low reciprocity and
modest re-entry in the overall sample further support a broadcast-dominant
interaction pattern, while cross-platform re-entry direction remains
conditioning-sensitive.

The model-to-observable validation loop supports this interpretation: predicted
and observed incidence align closely overall and by key strata, but the model
underpredicts non-root branching while overpredicting depth tails. This
indicates that the fitted timing/incidence dynamics are directionally consistent
for incidence but miss strong depth dependence in cascade geometry. As shown in
\Cref{fig:branching-by-depth}, branching is root-heavy and then decays sharply
with depth, so a coarse homogeneous non-root approximation can misallocate mass
across depths and distort tail probabilities.

\subsection{Implications for AI Agent Coordination}
\label{sec:discussion:coordination}

These dynamics matter for multi-step coordination tasks. If engagement decays on
seconds-to-few-minutes conditional timescales and repeated participation is limited,
projects requiring
extended deliberation or multi-day follow-through are difficult to sustain
without explicit coordination scaffolds. Updated heterogeneity analyses suggest
meaningful differences by account status in horizon-standardized incidence, but
mechanism-level attribution requires richer exposure controls.

\subsection{Design Implications}
\label{sec:discussion:design}

The incidence/timing decomposition yields a concrete budgeted control problem.
For any operational horizon \(h\), \Cref{sec:model:or-diagnostic} defines
\(q_h=\pi_h\phi_h\), where \(\pi_h\) is participation incidence and \(\phi_h\)
is conditional speed. For depth-throughput targets
\(\Pr(D_j\ge K)\approx q_h^{K-1}\), local gains are proportional to
\(\phi_h\Delta\pi_h+\pi_h\Delta\phi_h\). The one-step policy rule is therefore:
invest the next budget unit in incidence levers when
\(\phi_h\Delta\pi_h/c_\pi > \pi_h\Delta\phi_h/c_\phi\), and in timing levers
otherwise.

Using observed Moltbook margins at \(h=5\) minutes,
\(q_{5\mathrm{m}}=0.0942\) and
\(\phi_{5\mathrm{m}}=\Prob(T\le5\mathrm{m}\mid\delta=1)=0.9806\), which implies
\(\pi_{5\mathrm{m}}=q_{5\mathrm{m}}/\phi_{5\mathrm{m}}\approx0.0961\). Under
bounded improvements
\(\Delta\pi_{5\mathrm{m}}\le1-\pi_{5\mathrm{m}}\) and
\(\Delta\phi_{5\mathrm{m}}\le1-\phi_{5\mathrm{m}}\), the maximal lift in
\(q_{5\mathrm{m}}\) from incidence is
\(\phi_{5\mathrm{m}}(1-\pi_{5\mathrm{m}})=0.8864\), versus
\(\pi_{5\mathrm{m}}(1-\phi_{5\mathrm{m}})=0.00186\) from timing (about
\(476\times\) smaller). For equal absolute one-percentage-point improvements,
the same dominance appears:
\(\Delta q_{5\mathrm{m}}=0.9806\) pp from incidence versus
0.0961 pp from timing.

Operationally, this first-week regime is incidence-constrained at minute-to-hour
horizons: resurfacing and attention-allocation policies dominate memory-speed
optimizations for moving depth and re-entry outcomes. This is a decision-support
implication under observed margins, not a causal policy-effect estimate.

\subsection{Broader Implications}
\label{sec:discussion:broader}

Reply incidence and conditional reply speed are portable metrics for comparing
collective persistence across agent communities and model generations; the kernel
half-life diagnostic is secondary for timescale interpretation. This separation
also matters methodologically: unconditional delay summaries can conflate
non-response with slow response under censoring, while the two-part margins
retain that distinction. The current
cross-platform results provide descriptive baseline context under shared
estimators and are not a primary identification strategy.

Periodicity remains secondary: Rayleigh testing rejects exact modulo-4-hour
uniformity at large \(N\), but concentration is extremely small
(\(r=0.0308\)) and supplementary PSD checks show no dominant 4-hour line, so
we do not treat this as a practically meaningful global synchronization signal.

Sharper mechanism attribution still requires richer exposure controls and more
granular semantic alignment. Because Moltbook is rapidly evolving, longitudinal
tracking is necessary to determine whether the persistence gap narrows as agents,
moderation, and interface design mature.
