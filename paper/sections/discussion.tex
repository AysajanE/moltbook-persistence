\section{Discussion}
\label{sec:discussion}

We interpret the empirical patterns in \Cref{sec:results} through the model
framework in \Cref{sec:model}, focusing on collective behavior, platform
design, and identification scope. The evidence combines shallow geometry,
incidence-versus-conditional-timing decomposition, and model-to-data checks;
periodicity and cross-platform analyses provide secondary context.

\subsection{Interpreting Incidence and Conditional Reply Speed}
\label{sec:discussion:two-part}

The primary persistence readout is two-part: low direct-reply incidence (9.60\%)
and very fast conditional reply speed (\(t_{50}=4.55\) seconds,
\(t_{90}=50.05\) seconds, \(t_{95}=132.23\) seconds). Unconditional short-window
probabilities are similarly concentrated (8.47\% within 30 seconds and 9.41\%
within 5 minutes), implying that most observed replies occur almost immediately.
The remaining majority of parents receive no observed direct reply in-window.

This pattern is consistent with finite context retention, rapid task switching,
and weak return-to-thread memory support. The kernel half-life diagnostic remains
useful as a secondary timescale diagnostic, but the incidence/conditional-speed
split is the operationally informative summary for coordination limits.

\subsection{Structural Signatures of Limited Persistence}
\label{sec:discussion:structure}

Shallow, root-heavy trees are consistent with a low effective depth-tail slope
(\(\hat{s}_{\mathrm{depth}}=0.154\)) and rapidly decaying depth tails. Under a
heuristic branching interpretation (\cref{sec:model:branching}), this pattern
is compatible with low effective non-root reproduction. Low reciprocity and
modest re-entry in the overall sample further support a broadcast-dominant
interaction pattern, while the matched overlap subset shows that re-entry
direction can change under conditioning.

The model-to-observable validation loop supports this interpretation: predicted
and observed incidence align closely overall and by key strata, but the model
overpredicts non-root branching and depth tails. This indicates that the fitted
timing/incidence dynamics are directionally consistent with observed activity,
while deeper cascade generation remains weaker in the data than in the coarse
branching approximation.

\subsection{Implications for AI Agent Coordination}
\label{sec:discussion:coordination}

These dynamics matter for multi-step coordination tasks. If engagement decays on
seconds-to-few-minutes conditional timescales and repeated participation is limited,
projects requiring
extended deliberation or multi-day follow-through are difficult to sustain
without explicit coordination scaffolds. Updated heterogeneity analyses suggest
meaningful differences by account status in observed reply incidence, but
mechanism-level attribution requires richer exposure controls.

\subsection{Design Implications}
\label{sec:discussion:design}

We frame platform design as an operations-research decision-support mapping from
inputs to outputs under explicit trade-offs. A practical template is: (1) set
inputs \((K,\tau_K,B,L)\), where \(\tau_K\) is a target depth probability
threshold (\(\Pr(D_j\ge K)\ge\tau_K\)), \(B\) is compute/attention budget, and
\(L\) is acceptable notification load; (2) compute the required persistence
multiplier
\(m_\mu=\tau_K^{1/(K-1)}/\mu_0\) under
\(\Pr(D_j\ge K)\approx\mu^{K-1}\); (3) allocate that multiplier across
incidence versus conditional-timing margins, either through an incidence proxy
\(p_{\mathrm{reply,req}}\approx m_\mu p_{\mathrm{reply},0}\) (holding
conditional timing shape fixed) or a timing proxy
\(\beta_{\mathrm{req}}\approx\beta_0/m_\mu\) (holding \(\alpha\) fixed in
\(\mu\approx\alpha/\beta\)).

When instantiated, this template should draw \(\mu_0\) and
\(p_{\mathrm{reply},0}\) only from already reported baseline metrics
(\Cref{tab:model-observable-validation,tab:reply-dynamics}). The resulting
\(m_\mu\), \(p_{\mathrm{reply,req}}\), and \(\beta_{\mathrm{req}}\) are
planning calculations under the same approximation, not new empirical results.

The same metrics provide an explicit A/B evaluation map. Memory-aid arms are
scored on service continuity (higher incidence among eligible parents and/or
slower conditional decay as a \(\beta\)-proxy) relative to \((B,L)\)
constraints. Resurfacing arms are scored as increases in effective arrival rate
into the service system (incidence/\(\alpha\)-proxy shifts) and associated load
costs. Re-entry prompt arms are scored on changes in re-entry frequency and
depth-tail outcomes per incremental notification load. In capacity-allocation
terms, resurfacing raises effective arrival rates, while memory scaffolding
improves service continuity and can reduce abandonment from stale threads.
These intervention implications are mechanism-consistent decision-support
hypotheses under the fitted model.

\subsection{Broader Implications}
\label{sec:discussion:broader}

Reply incidence and conditional reply speed are portable metrics for comparing
collective persistence across agent communities and model generations; the kernel
half-life diagnostic is secondary for timescale interpretation. The current cross-platform
results add overlap-region context beyond raw platform contrasts via
deterministic matching, but are not a primary identification strategy;
the full matched diagnostics are therefore kept in appendix material.

Periodicity evidence should be read similarly: the modulo-4-hour phase
concentration is small (\(r=0.0308\)) relative to the detectability target
(\(\kappa^\star=0.2\)). Consistent with that effect size, AR(1)-calibrated PSD
tests do not show a strong 4-hour line. This is an
effect-size-versus-detectability result rather than a contradiction.

Sharper mechanism attribution still requires richer exposure controls and more
granular semantic alignment. Because Moltbook is rapidly evolving, longitudinal
tracking is necessary to determine whether the persistence gap narrows as agents,
moderation, and interface design mature.
