\section{Discussion}
\label{sec:discussion}

This section interprets the empirical patterns in \Cref{sec:results} through the
model framework in \Cref{sec:model}, focusing on implications for collective
agent behavior, platform design, and identification scope. The central
empirical chain is: shallow geometry, then persistence decomposition into
incidence versus conditional timing, then model-to-data consistency checks.
Together these support a ``fast-response-or-silence'' regime as the core
contribution; periodicity and cross-platform analyses remain secondary context.

\subsection{Interpreting Minute-Scale Reply Decay}
\label{sec:discussion:halflife}

The primary Moltbook estimate is a sub-minute reply-kernel half-life, accompanied
by heavy censoring. Together, these results indicate a bursty regime: when direct
replies occur, they tend to occur quickly; most at-risk comments receive no
observed direct reply during the observation window.

This pattern is consistent with finite context retention, rapid task switching in
high-volume feeds, and weak return-to-thread memory support. The Reddit baseline,
by contrast, shows hour-scale reply-kernel half-life under the same estimator.
Matched-subset half-life estimates preserve this large timing gap, but the matched
design is intentionally coarse and overlap-restricted, so it is used as
secondary contextual evidence rather than a core causal estimate.
The matched Moltbook survival subset includes only 22 events on narrow overlap
support, so its half-life estimate is noisy and not directly comparable with the
overall Moltbook estimate.

\subsection{Structural Signatures of Limited Persistence}
\label{sec:discussion:structure}

Shallow, root-heavy trees are consistent with a low effective depth-tail slope
(\(\hat{\mu}=0.154\)) and rapidly decaying depth tails. Under a branching
interpretation (\cref{sec:model:branching}), this pattern is compatible with low
effective non-root reproduction. Low reciprocity and modest re-entry in the
overall sample further support a broadcast-dominant interaction pattern, while
the matched overlap subset shows that re-entry direction can change under
conditioning.

\subsection{Implications for AI Agent Coordination}
\label{sec:discussion:coordination}

These dynamics matter for multi-step coordination tasks. If engagement decays on
minute timescales and repeated participation is limited, projects requiring
extended deliberation or multi-day follow-through are difficult to sustain
without explicit coordination scaffolds. Updated heterogeneity analyses suggest
meaningful differences by account status in observed reply incidence, but
follower-bin patterns remain non-monotonic and sparse in upper bins, so stronger
hub claims require longer windows and richer controls.

\subsection{Design Implications}
\label{sec:discussion:design}

The measured dynamics point to practical interventions that can be tested in
future platform studies: memory scaffolding for active threads, summary-based
re-entry support, feed prioritization of threads with prior participation, and
incentive structures that reward sustained engagement rather than raw posting
volume. Structured coordination interfaces (for example, commitments, milestones,
and scheduled check-ins) are especially relevant if the default interaction
regime remains fast-response-or-silence.

\subsection{Broader Implications}
\label{sec:discussion:broader}

Interaction half-life is a portable metric for comparing collective persistence
across agent communities and model generations. The current cross-platform
results add overlap-region context beyond raw platform contrasts via
deterministic matching, but are not a primary causal identification strategy;
the full matched diagnostics are therefore kept in appendix material.
Stronger causal interpretation still requires richer exposure controls and more
granular semantic alignment. Because Moltbook is rapidly evolving, longitudinal
tracking is necessary to determine whether the persistence gap narrows as agents,
moderation, and interface design mature.
