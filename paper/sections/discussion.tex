\section{Discussion}
\label{sec:discussion}

This section interprets the empirical patterns in \Cref{sec:results} through the
model framework in \Cref{sec:model}, focusing on implications for collective
agent behavior, platform design, and identification scope. The central
empirical chain is: shallow geometry, then persistence decomposition into
incidence versus conditional timing, then model-to-data consistency checks.
Together these are consistent with a ``low-incidence/fast-conditional-response'' regime;
periodicity and cross-platform analyses remain secondary context.

\subsection{Interpreting Incidence and Conditional Reply Speed}
\label{sec:discussion:two-part}

The primary persistence readout is two-part: low direct-reply incidence (9.60\%)
and very fast conditional reply speed (\(t_{50}=0.076\) minutes,
\(t_{90}=0.834\) minutes, \(t_{95}=2.204\) minutes). Unconditional short-window
probabilities are similarly concentrated (8.47\% within 30 seconds and 9.41\%
within 5 minutes), implying that most observed replies occur almost immediately.
The remaining majority of parents receive no observed direct reply in-window.

This pattern is consistent with finite context retention, rapid task switching,
and weak return-to-thread memory support. Half-life remains useful as a
secondary kernel-timescale diagnostic, but the incidence/conditional-speed split
is the operationally informative summary for coordination limits.

\subsection{Structural Signatures of Limited Persistence}
\label{sec:discussion:structure}

Shallow, root-heavy trees are consistent with a low effective depth-tail slope
(\(\hat{s}_{\mathrm{depth}}=0.154\)) and rapidly decaying depth tails. Under a
heuristic branching interpretation (\cref{sec:model:branching}), this pattern
is compatible with low effective non-root reproduction. Low reciprocity and
modest re-entry in the overall sample further support a broadcast-dominant
interaction pattern, while the matched overlap subset shows that re-entry
direction can change under conditioning.

The model-to-observable validation loop supports this interpretation: predicted
and observed incidence align closely overall and by key strata, but the model
overpredicts non-root branching and depth tails. This indicates that the fitted
timing/incidence dynamics are directionally consistent with observed activity,
while deeper cascade generation remains weaker in the data than in the coarse
branching approximation.

\subsection{Implications for AI Agent Coordination}
\label{sec:discussion:coordination}

These dynamics matter for multi-step coordination tasks. If engagement decays on
minute-scale conditional timescales and repeated participation is limited,
projects requiring
extended deliberation or multi-day follow-through are difficult to sustain
without explicit coordination scaffolds. Updated heterogeneity analyses suggest
meaningful differences by account status in observed reply incidence, but
causal attribution remains out of scope.

\subsection{Design Implications}
\label{sec:discussion:design}

The measured dynamics point to design interventions that can be tested in
future platform studies: memory scaffolding for active threads, summary-based
re-entry support, feed prioritization of threads with prior participation, and
incentive structures that reward sustained engagement rather than raw posting
volume. Structured coordination interfaces (for example, commitments, milestones,
and scheduled check-ins) are especially relevant if the default interaction
regime remains low-incidence/fast-conditional-response.

Dependence robustness reinforces this design implication: when restricting to
one parent per thread, pooled incidence drops (9.60\% to 7.21\%), but
conditional median speed is nearly unchanged (0.07590 to 0.07519 minutes). The
bottleneck appears to be whether a reply happens, not how quickly replies arrive
once they happen.

\subsection{Broader Implications}
\label{sec:discussion:broader}

Reply incidence and conditional reply speed are portable metrics for comparing
collective persistence across agent communities and model generations; half-life
is a secondary diagnostic for kernel-timescale interpretation. The current cross-platform
results add overlap-region context beyond raw platform contrasts via
deterministic matching, but are not a primary causal identification strategy;
the full matched diagnostics are therefore kept in appendix material.

Periodicity evidence should be read similarly: AR(1)-calibrated PSD tests do not
show a significant 4-hour line, while event-time Rayleigh detects a small but
non-zero phase concentration. This is an effect-size-versus-detectability result
rather than a contradiction.

Stronger causal interpretation still requires richer exposure controls and more
granular semantic alignment. Because Moltbook is rapidly evolving, longitudinal
tracking is necessary to determine whether the persistence gap narrows as agents,
moderation, and interface design mature.
