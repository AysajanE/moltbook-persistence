\section*{Supplementary Material: Deferred Appendices and Extended Technical Details}

This supplementary file is the full home for technical/deferred Appendix A/B
content from the main manuscript: formal derivations, periodicity
detectability details, matching diagnostics/mechanics, robustness tables moved
from main, and reproducibility/data-availability details.

\subsection*{S0. Deferred Main-Text Formalism and Proof Targets}
\label{sec:appendix}
\label{sec:appendix:proofs}

\paragraph{S0.1 Full intensity and competing-risks parent selection.}
For thread \(j\), let \(\mathcal{H}_j(t)\) denote history up to \(t\). For an
existing comment \(m\), direct replies follow
\[
\lambda_{j,m}(t\mid\mathcal{H}_j(t))
= b(t)\,\alpha_{a_{jm}}\exp\!\left[-\beta_{a_{jm}}(t-t_{jm})\right]\mathbf{1}\{t>t_{jm}\}.
\]
Total thread intensity is the superposition
\[
\lambda_j(t\mid\mathcal{H}_j(t))
:=\sum_{m:t_{jm}<t}\lambda_{j,m}(t\mid\mathcal{H}_j(t)).
\]
Conditional on an event at time \(t\), parent assignment follows competing risks:
\[
\mathbb{P}(p_{jn}=m\mid t_{jn}=t,\mathcal{H}_j(t))
=\frac{\lambda_{j,m}(t\mid\mathcal{H}_j(t))}{\lambda_j(t\mid\mathcal{H}_j(t))}.
\]
This is the formal construction deferred from the main manuscript section
``Framework and Estimands.''

\paragraph{S0.2 Deferred formal statements.}

\paragraph{S0.2.1 Influence--persistence trade-off (formal statement).}
\label{supp:s0-tradeoff}
\label{app:proof-tradeoff}
Under nonnegative availability,
\(\mu_i(s)=\int_0^\infty b(s+u)\alpha_i e^{-\beta_i u}\,du\) is increasing in
\(\alpha_i\) and decreasing in \(\beta_i\).

\paragraph{S0.2.2 Root-special expected thread size (formal statement).}
\label{supp:s0-thread-size}
\label{app:proof-thread-size}
With root offspring mean \(\mu_0\),
non-root offspring mean \(\mu<1\), and standard independence assumptions,
\[
\mathbb{E}[N_j]=\frac{\mu_0}{1-\mu}.
\]

\paragraph{S0.2.3 Periodic mean-intensity detectability (formal statement).}
\label{supp:s0-periodicity}
\label{app:proof-periodicity}
If
\(\lambda(t)=b(t)g(t)\), \(b(t)\) is \(\tau\)-periodic, and \(\mathbb{E}[g(t)]\) is
\(\tau\)-periodic (or approximately constant on the \(\tau\)-scale), then
\(\mathbb{E}[\lambda(t)]\) is \(\tau\)-periodic and long-window binned-count spectra
exhibit mass near \(\ell/\tau\), up to finite-window and leakage effects.

\paragraph{S0.2.4 Horizon-throughput decomposition and priority rule.}
\label{supp:s0-or-diagnostic}
For horizon \(h\), define
\[
R_h:=\{C_m\ge h\ \text{or}\ T_m\le h\},\quad
\pi_h:=\mathbb{P}(\delta_m=1\mid R_h),\quad
\phi_h:=\mathbb{P}(T_m\le h\mid\delta_m=1,R_h),
\]
and
\[
q_h:=\mathbb{P}(T_m\le h\mid R_h).
\]
Then \(q_h=\pi_h\phi_h\). For any differentiable objective \(G(q_h)\),
\[
dG=G'(q_h)\left(\phi_h\,d\pi_h+\pi_h\,d\phi_h\right).
\]
With per-unit intervention costs \(c_\pi,c_\phi>0\), one-step budget allocation
prioritizes incidence over timing when
\(\phi_h\,d\pi_h/c_\pi > \pi_h\,d\phi_h/c_\phi\), and timing otherwise.

\paragraph{S0.3 Deferred model extensions for future estimation.}
\label{sec:model:extensions}
Richer extensions (hierarchical random effects, re-entry self-excitation, and
visibility-weighted parent assignment) are deferred from the main paper and are
not used for identification in this first-week snapshot.

\subsection*{S1. Full Proofs}

\paragraph{S1.1 Proof of S0.2.1 (influence--persistence trade-off).}
\label{supp:s1-tradeoff}
For fixed \(s\), use
\[
\mu_i(s)=\int_0^\infty b(s+u)\,\alpha_i e^{-\beta_i u}\,du,
\]
with \(b(\cdot)\ge0\). Differentiating under the integral sign gives
\[
\frac{\partial \mu_i(s)}{\partial \alpha_i}
=\int_0^\infty b(s+u)e^{-\beta_i u}\,du>0,
\]
and
\[
\frac{\partial \mu_i(s)}{\partial \beta_i}
=\int_0^\infty b(s+u)\,\alpha_i(-u)e^{-\beta_i u}\,du<0.
\]
Hence \(\mu_i(s)\) is strictly increasing in \(\alpha_i\) and strictly
decreasing in \(\beta_i\).

\paragraph{S1.2 Proof of S0.2.2 (root-special expected thread size).}
\label{supp:s1-thread-size}
Let \(X_0\) denote depth-1 comments from the root, with \(\mathbb{E}[X_0]=\mu_0\).
Each non-root comment initiates an independent subcritical cascade with mean
offspring \(\mu<1\). Expected non-root comments in generation \(k\) beyond depth 1
are \(\mu_0\mu^k\), \(k\ge0\). Therefore
\[
\mathbb{E}[N_j]=\sum_{k=0}^{\infty}\mu_0\mu^k=\frac{\mu_0}{1-\mu}.
\]

\paragraph{S1.3 Proof of S0.2.3 (periodic mean-intensity detectability).}
\label{supp:s1-periodicity}
Under assumptions, mean intensity is
\[
m(t):=\mathbb{E}[\lambda(t)]=\mathbb{E}[b(t)g(t)]=b(t)\mathbb{E}[g(t)].
\]
If \(\mathbb{E}[g(t)]\) is \(\tau\)-periodic, then \(m(t)\) is \(\tau\)-periodic.
For binned counts \(C_r:=N((r\Delta,(r+1)\Delta])\),
\[
\mathbb{E}[C_r]=\int_{r\Delta}^{(r+1)\Delta} m(s)\,ds,
\]
so expected discrete-time counts inherit periodic structure and produce elevated
periodogram/power spectral density (PSD) mass near harmonics \(\ell/\tau\), up to
finite-window leakage.

\paragraph{S1.4 Proof of S0.2.4 (horizon-throughput decomposition and priority rule).}
\label{supp:s1-or-diagnostic}
By the law of total probability on \(R_h\),
\[
\mathbb{P}(T_m\le h\mid R_h)
=\mathbb{P}(\delta_m=1\mid R_h)\mathbb{P}(T_m\le h\mid\delta_m=1,R_h)
+\mathbb{P}(\delta_m=0\mid R_h)\underbrace{\mathbb{P}(T_m\le h\mid\delta_m=0,R_h)}_{0},
\]
which gives \(q_h=\pi_h\phi_h\).

For differentiable \(G\), the total differential is
\[
dG=G'(q_h)\,dq_h
=G'(q_h)\left(\phi_h\,d\pi_h+\pi_h\,d\phi_h\right).
\]
Under one-step budget allocation with costs \(c_\pi,c_\phi>0\), compare
cost-normalized local gains:
\[
\frac{G'(q_h)\phi_h\,d\pi_h}{c_\pi}
\quad\text{versus}\quad
\frac{G'(q_h)\pi_h\,d\phi_h}{c_\phi}.
\]
Since \(G'(q_h)\ge0\) for monotone throughput objectives, the preferred margin
is incidence iff \(\phi_h\,d\pi_h/c_\pi > \pi_h\,d\phi_h/c_\phi\); otherwise
timing is preferred.

\subsection*{S2. Submolt Keyword Triggers (Expanded)}
\label{sec:appendix:submolts}

\begin{table}[h]
\centering
\caption{Representative keyword triggers per submolt category. Categories are
applied in priority order (Spam first).}
\label{tab:submolt-examples}
\small
\begin{tabular}{@{}lp{9cm}@{}}
\toprule
\textbf{Category} & \textbf{Keyword triggers (examples)} \\
\midrule
Spam/Low-Signal & crypto, bitcoin, airdrop, nft, defi, token, solana, scam, shitpost \\
Builder/Technical & programming, coding, build, builders, dev, engineering, tools, automation, research, framework, mcp, tech \\
Philosophy/Meta & philosophy, consciousness, existential, meta, souls, musings, aithoughts, ponderings \\
Creative & writing, poetry, music, creative, story, theatre, shakespeare \\
Social/Casual & general, casual, introductions, jokes, gaming, humanwatching, social, todayilearned, random \\
Other & (default: no keyword match) \\
\bottomrule
\end{tabular}
\end{table}

\paragraph{S2.1 Keyword-mapping sensitivity.}
\label{sec:appendix:submolts-sensitivity}
Deterministic keyword triggers are transparent but coarse. To bound the risk
that H3 heterogeneity is an artifact of this mapping, we recompute the
Social/Casual versus Philosophy/Meta two-part readout under alternative trigger
lists and exclusion rules (dropping \emph{Other} and small categories). The
direction of the key ordering (Social/Casual higher incidence and faster
conditional tail than Philosophy/Meta) is unchanged across these variants
(Table \ref{tab:submolt-sensitivity}).

\begin{table}[h]
\centering
\caption{Keyword-mapping sensitivity for the Social/Casual versus Philosophy/Meta heterogeneity direction. Source table: \texttt{paper/tables/moltbook\_submolt\_keyword\_sensitivity.csv}.}
\label{tab:submolt-sensitivity}
\scriptsize
\setlength{\tabcolsep}{3pt}
\resizebox{\linewidth}{!} & \textbf{$t_{90}$ (minutes)} & \textbf{Parents} & \textbf{Incidence \%} & \textbf{$t_{90}$ (minutes)} \\
\midrule
baseline (token+substring) & Drop Other & 189,765 & 10.95 & 0.41 & 5,831 & 1.80 & 47.81 \\
baseline (token+substring) & Drop Other + small (<1000) & 189,765 & 10.95 & 0.41 & 5,831 & 1.80 & 47.81 \\
baseline (token+substring) & None & 189,765 & 10.95 & 0.41 & 5,831 & 1.80 & 47.81 \\
baseline (token-only) & Drop Other & 189,666 & 10.96 & 0.41 & 5,826 & 1.85 & 47.70 \\
baseline (token-only) & Drop Other + small (<1000) & 189,666 & 10.96 & 0.41 & 5,826 & 1.85 & 47.70 \\
baseline (token-only) & None & 189,666 & 10.96 & 0.41 & 5,826 & 1.85 & 47.70 \\
expanded (token+substring) & Drop Other & 189,764 & 10.95 & 0.41 & 5,796 & 1.90 & 46.35 \\
expanded (token+substring) & Drop Other + small (<1000) & 189,764 & 10.95 & 0.41 & 5,796 & 1.90 & 46.35 \\
expanded (token+substring) & None & 189,764 & 10.95 & 0.41 & 5,796 & 1.90 & 46.35 \\
\bottomrule
\end{tabular}

}
\end{table}

\subsection*{S3. Periodicity Robustness and Detectability Mechanics}
\label{sec:appendix:periodicity-robustness}

\begin{figure}[h]
\centering
\includegraphics[width=0.9\linewidth]{figures/periodicity_bin_robustness_moltbook.png}
\caption{Bin-width robustness for Moltbook periodicity tests (5, 15, 30 minutes).}
\label{fig:supp-periodicity-bin-robustness}
\end{figure}

\paragraph{S3.1 Event-time modulo-\(4\)-hour test and detectability setup.}
Periodicity is evaluated on the longest contiguous segment only (segment index
1; 63.5175 hours; \(N=220{,}461\) events; window
2026-02-02 04:20:50Z to 2026-02-04 19:51:53Z; gap-threshold setting 6 hours).
Event-time modulo-4-hour testing gives resultant length \(r=0.0308\),
\(Z=209.57\), Monte Carlo \(p=5\times10^{-6}\), and mean phase 153.2 minutes.
This rejects exact phase uniformity statistically, but the concentration effect
size is extremely small. The null calibration uses 200,000 Monte Carlo draws at
\(\alpha=0.05\), giving critical \(Z=2.9940\).

\begin{table}[h]
\centering
\caption{Detectability simulation mechanics and provenance for the Moltbook
periodicity test. Source tables:
\texttt{paper/tables/moltbook\_periodicity\_event\_time\_test.csv} and
\texttt{paper/tables/moltbook\_periodicity\_detectability\_power.csv}.}
\label{tab:supp-periodicity-detectability}
\small
\begin{tabular}{@{}ll@{}}
\toprule
\textbf{Component} & \textbf{Value} \\
\midrule
Target period \(\tau\) & 4.0 hours \\
Segment event count & 220,461 \\
Segment duration & 63.5175 hours \\
Null Monte Carlo reps (Rayleigh) & 200,000 \\
Power simulation method & noncentral\_chi\_square\_monte\_carlo \\
\(\kappa\) grid & 0.0, 0.2, \ldots, 3.0 \\
Power reps per \(\kappa\) & 50,000 \\
Seed & 20260208 \\
Estimated null size at \(\kappa=0\) & 0.04964 \\
First tested \(\kappa\) with estimated power \(\ge 80\%\) (\(\kappa^\star\)) & 0.2 \\
Estimated power at \(\kappa=0.2\) & 1.0 \\
\bottomrule
\end{tabular}
\end{table}

The simulation stores mean resultant-length mapping
\(\rho=I_1(\kappa)/I_0(\kappa)\); for example, \(\rho=0.0995\) at
\(\kappa=0.2\), which is over three times the observed \(r=0.0308\). Because
the \(\kappa\) grid is coarse (0.0, 0.2, \ldots), \(\kappa^\star=0.2\) is a
grid-crossing artifact rather than a sharp detectability threshold. This
section is the detailed detectability record deferred from main text.

\paragraph{S3.2 PSD robustness.}
Supplementary PSD robustness for the 4-hour target frequency yields first-order
autoregressive (AR(1))-calibrated \(p\)-values 0.508 (5-minute bins), 0.501
(15-minute bins), and 0.556 (30-minute bins).

\paragraph{S3.3 Reddit baseline supplementary diagnostics.}
\label{sec:appendix:reddit-details}
Under the same estimators, Reddit shows deeper threads (mean maximum depth
2.17), higher direct-reply incidence (36.2\%), and longer kernel half-life
diagnostic values (2.61 hours; 95\% confidence interval (CI): [2.29, 2.95]) than Moltbook.

\subsection*{S4. Cross-Platform Matching Diagnostics and Mechanics}
\label{sec:appendix:comparison-details}

\paragraph{S4.1 Matched-sample flow.}
\begin{figure}[h]
\centering
\includegraphics[width=0.9\linewidth]{figures/cross_platform_matched_sample_flow.png}
\caption{Matched-sample flow for the coarse cross-platform design.}
\label{fig:supp-matched-sample-flow}
\end{figure}

Input includes 34,730 Moltbook threads and 1,104 Reddit threads. Exact-overlap
strata on coarse topic, Coordinated Universal Time (UTC) posting hour, and early-engagement bin retain 2,641
Moltbook threads and 888 Reddit threads across 118 shared strata. Deterministic
1:1 matching yields 813 pairs (2.34\% of Moltbook threads).

\paragraph{S4.2 Balance summary (before/after matching).}
\begin{table}[h]
\centering
\caption{Covariate balance before and after coarsened exact matching.
SMD = standardized mean difference (absolute value);
TVD = total variation distance (categorical covariates only).}
\label{tab:supp-matching-balance}
\small
\begin{tabular}{@{}lrrrr@{}}
\toprule
& \multicolumn{2}{c}{\textbf{Before Matching}} & \multicolumn{2}{c}{\textbf{After Matching}} \\
\cmidrule(lr){2-3} \cmidrule(lr){4-5}
\textbf{Covariate} & \textbf{$|$SMD$|$} & \textbf{TVD} & \textbf{$|$SMD$|$} & \textbf{TVD} \\
\midrule
Post hour (UTC) & 0.158 & --- & 0.000 & --- \\
Early comments (30 min) & 0.836 & --- & 0.052 & --- \\
Topic (coarse) & 3.89 & 0.901 & 0.000 & 0.000 \\
Post hour bin & 0.217 & 0.156 & 0.000 & 0.000 \\
Early engagement bin & 0.873 & 0.586 & 0.000 & 0.000 \\
\bottomrule
\end{tabular}
\end{table}

\paragraph{S4.3 Matched paired-effects summary.}
\begin{table}[h]
\centering
\caption{Matched paired-effects diagnostics from
\texttt{paper/tables/cross\_platform\_matched\_paired\_effects.csv}. Mean difference is Moltbook minus Reddit (M-R).}
\label{tab:cross-platform-paired-effects}
\scriptsize
\begin{tabular}{@{}lrrrr@{}}
\toprule
\textbf{Outcome} & \textbf{\(n\) pairs} & \textbf{Mean difference (M-R)} & \textbf{95\% CI} & \textbf{Wilcoxon \(p\)} \\
\midrule
Comments per thread & 813 & -7.70 & [-9.43, -6.06] & \(1.43\times10^{-49}\) \\
Max depth & 813 & -1.19 & [-1.34, -1.04] & \(5.16\times10^{-54}\) \\
Unique participants & 813 & -5.01 & [-6.10, -4.00] & \(7.62\times10^{-45}\) \\
Thread duration (hours) & 813 & -8.61 & [-10.00, -7.34] & \(6.50\times10^{-41}\) \\
Re-entry rate & 792 & -0.040 & [-0.054, -0.027] & \(4.77\times10^{-9}\) \\
\bottomrule
\end{tabular}
\end{table}

\paragraph{S4.4 Matched-subset kernel half-life diagnostic details.}
\label{sec:appendix:match-halflife}
Restricting survival units to matched threads gives 0.063 hours on Moltbook
(3.77 minutes; 95\% CI: [1.19, 7.06] minutes; \(n=1{,}841\), 22 events) and
2.44 hours on Reddit (95\% CI: [2.13, 2.79] hours; \(n=7{,}979\), 2,882
events). These are platform-level contextual contrasts, not paired
thread-level survival effects.

\subsection*{S5. Operational Estimation Settings}
\label{sec:appendix:operational-details}

\begin{itemize}
\item Coverage and at-risk rules: each non-root comment is one candidate-parent
survival unit; first-reply durations are right-censored at observation boundary;
no replies are imputed across the canonical 41.7-hour timeline gap.
\item Periodicity preprocessing: tests use the longest contiguous segment
(63.5 hours). Supplementary PSD checks use 15-minute bins with 5- and 30-minute
robustness repeats.
\item Bootstrap mechanics: Moltbook-only and Reddit-only confidence intervals
use thread-cluster bootstrap with \texttt{bootstrap\_reps=400}; matched paired
effects use \texttt{bootstrap\_reps=1000}; matched-subset kernel half-life
diagnostic intervals use \texttt{half\_life\_bootstrap\_reps=400}.
\item Matching mechanics: coarsened exact matching uses first-30-minute action
volume bins, deterministic coarse topic map
(\texttt{tech}/\texttt{meta}/\texttt{general}/\texttt{spam}), and exact UTC posting hour;
within each shared stratum, threads are paired deterministically one-to-one.
\end{itemize}

\subsection*{S6. Robustness Tables Relocated from Main Text}

\paragraph{S6.1 Timing-model misspecification diagnostics (moved from main).}
\begin{table}[h]
\centering
\caption{Observed vs.\ fitted timing-model diagnostics for conditional reply
times and event probability (migrated from main results). Large quantile
residuals indicate misspecification of early-time shape under a single-family
parametric timing fit.}
\label{tab:timing-model-fit}
\scriptsize
\begin{tabular}{@{}lrrr@{}}
\toprule
\textbf{Moltbook (seconds)} & \textbf{Observed} & \textbf{Fitted} & \textbf{Residual (fit - obs)} \\
\midrule
Event probability (\%) & 9.60 & 9.91 & +0.31 \\
\(p_{10}\) (s) & 3.67 & 6.00 & +2.33 \\
\(p_{50}\) (s) & 4.55 & 39.94 & +35.39 \\
\(p_{90}\) (s) & 50.05 & 134.98 & +84.93 \\
\midrule
\textbf{Reddit (seconds; minutes in parentheses)} & \textbf{Observed} & \textbf{Fitted} & \textbf{Residual (fit - obs)} \\
\midrule
Event probability (\%) & 36.20 & 38.70 & +2.50 \\
\(p_{10}\) (s) & 200.00 (3.33 min) & 1138.35 (18.97 min) & +938.35 (+15.64 min) \\
\(p_{50}\) (s) & 2359.00 (39.32 min) & 7835.63 (130.59 min) & +5476.63 (+91.27 min) \\
\(p_{90}\) (s) & 33732.50 (562.21 min) & 28141.25 (469.02 min) & -5591.25 (-93.19 min) \\
\bottomrule
\end{tabular}
\end{table}

For Moltbook, the fitted median and upper quantile overstate observed values by
+35.39 and +84.93 seconds despite near-calibration of event probability.
This pattern is consistent with a spike-plus-tail shape that the single-family
parametric fit does not capture well in the seconds-to-minutes region used for
main interpretation. Accordingly, main-text inference uses nonparametric
conditional quantiles and early-mass probabilities as primary timing evidence,
with parametric half-life summaries retained only as secondary diagnostics.

\paragraph{S6.2 Model-observable validation table (moved from main).}
\begin{table}[h]
\centering
\caption{Model-to-observable validation: predicted vs.\ observed incidence,
non-root branching, and depth tails (overall and key stratifications; migrated
from main results).}
\label{tab:model-observable-validation}
\begingroup
\setlength{\tabcolsep}{4pt}
\scriptsize
\resizebox{\linewidth}{!} & \textbf{Obs. inc. \%} & \textbf{Pred. branch} & \textbf{Obs. branch} & \textbf{Pred. \(\Pr(D\ge3)\)} & \textbf{Obs. \(\Pr(D\ge3)\)} & \textbf{Pred. \(\Pr(D\ge5)\)} & \textbf{Obs. \(\Pr(D\ge5)\)} \\
\midrule
Overall & 9.91 & 9.60 & 0.104 & 0.134 & 0.0109 & 0.0013 & 0.00012 & 0.00001 \\
Claimed & 20.49 & 19.23 & 0.229 & 0.274 & 0.0526 & 0.0030 & 0.00276 & 0.00000 \\
Unclaimed & 8.90 & 8.65 & 0.093 & 0.120 & 0.0087 & 0.0012 & 0.00008 & 0.00001 \\
Builder/Technical & 1.72 & 1.72 & 0.017 & 0.017 & 0.0003 & 0.0001 & 0.00000 & 0.00000 \\
Creative & 1.62 & 1.62 & 0.016 & 0.016 & 0.0003 & 0.0000 & 0.00000 & 0.00000 \\
Other & 2.27 & 2.26 & 0.023 & 0.025 & 0.0005 & 0.0001 & 0.00000 & 0.00000 \\
Philosophy/Meta & 1.81 & 1.80 & 0.018 & 0.018 & 0.0003 & 0.0002 & 0.00000 & 0.00000 \\
Social/Casual & 11.36 & 10.95 & 0.121 & 0.154 & 0.0145 & 0.0015 & 0.00021 & 0.00001 \\
Spam/Low-Signal & 0.88 & 0.88 & 0.009 & 0.009 & 0.0001 & 0.0000 & 0.00000 & 0.00000 \\
\bottomrule
\end{tabular}
\par}
\endgroup
\end{table}

\paragraph{S6.3 One-parent-per-thread robustness table (moved from main).}
\begin{table}[h]
\centering
\caption{One-parent-per-thread robustness against within-thread clustering
dependence (migrated from main results).}
\label{tab:dependence-robustness}
\small
\begin{tabular}{@{}lrrrr@{}}
\toprule
\textbf{Metric} & \textbf{Primary} & \textbf{One-parent/thread} & \textbf{Abs.\ diff.} & \textbf{Rel.\ diff. \%} \\
\midrule
Reply incidence \(\Pr(\delta=1)\) & 0.09596 & 0.07213 & 0.02383 & -24.84 \\
Conditional \(t_{50}\) (s) & 4.55 & 4.51 & 0.04 & -0.93 \\
Conditional \(t_{90}\) (s) & 50.05 & 41.08 & 8.97 & -17.92 \\
Kernel half-life (diagnostic; min) & 0.68451 & 0.43601 & 0.24850 & -36.30 \\
\bottomrule
\end{tabular}
\end{table}

\paragraph{S6.4 Coverage-gap disambiguation and robustness package.}

\begin{table}[h]
\centering
\caption{Comment-gap disambiguation evidence across raw archive tables. The
comment interval gap is 2026-01-31 10:37:53Z to 2026-02-02 04:20:50Z. Source:
\texttt{paper/tables/moltbook\_gap\_disambiguation\_evidence.csv}.}
\label{tab:supp-gap-disambiguation}
\small
\begin{tabular}{@{}lrrr@{}}
\toprule
\textbf{Table} & \textbf{Records} & \textbf{Records in comment-gap interval} & \textbf{Max inter-event gap (h)} \\
\midrule
comments & 226,173 & 0 & 41.72 \\
posts & 119,677 & 38,166 & 11.20 \\
snapshots & 114 & 39 & 2.28 \\
word\_frequency & 15,346 & 5,039 & 3.00 \\
\bottomrule
\end{tabular}
\end{table}

\begin{table}[h]
\centering
\caption{Two-part decomposition robustness across contiguous windows and
gap-overlap exclusions. Source:
\texttt{paper/tables/moltbook\_gap\_window\_robustness.csv}.}
\label{tab:supp-gap-window-robustness}
\scriptsize
\begin{tabular}{@{}lrrrrrrr@{}}
\toprule
\textbf{Scenario} & \textbf{Parents} & \textbf{Replies} & \textbf{Incidence \%} & \textbf{\(t_{50}\) (s)} & \textbf{\(t_{90}\) (s)} & \textbf{\(\Pr(\mathrm{reply}\le30\mathrm{s})\) \%} & \textbf{\(\Pr(\mathrm{reply}\le5\mathrm{m})\) \%} \\
\midrule
Full window & 223,316 & 21,430 & 9.60 & 4.55 & 50.05 & 8.47 & 9.41 \\
Pre-gap contiguous & 2,856 & 30 & 1.05 & 513.48 & 1699.77 & 0.04 & 0.35 \\
Post-gap contiguous & 220,460 & 21,400 & 9.71 & 4.55 & 48.66 & 8.58 & 9.53 \\
Exclude gap overlap (6h) & 220,460 & 21,400 & 9.71 & 4.55 & 48.66 & 8.58 & 9.53 \\
Exclude gap overlap (24h) & 220,460 & 21,400 & 9.71 & 4.55 & 48.66 & 8.58 & 9.53 \\
\bottomrule
\end{tabular}
\end{table}

\begin{table}[h]
\centering
\caption{Horizon-standardized reply probabilities using explicit risk sets.
Source: \texttt{paper/tables/moltbook\_gap\_horizon\_standardized.csv}.}
\label{tab:supp-gap-horizon-standardized}
\small
\begin{tabular}{@{}llrr@{}}
\toprule
\textbf{Scenario} & \textbf{Horizon} & \textbf{Risk-set \(n\)} & \textbf{\(\Pr(\mathrm{reply}\le t)\) \%} \\
\midrule
Full window & 30s & 223,312 & 8.47 \\
Full window & 5m & 223,102 & 9.42 \\
Full window & 1h & 217,472 & 9.82 \\
Pre-gap contiguous & 30s & 2,854 & 0.04 \\
Pre-gap contiguous & 5m & 2,842 & 0.35 \\
Pre-gap contiguous & 1h & 2,063 & 1.36 \\
Post-gap contiguous & 30s & 220,456 & 8.58 \\
Post-gap contiguous & 5m & 220,244 & 9.54 \\
Post-gap contiguous & 1h & 214,610 & 9.94 \\
Exclude gap overlap (6h) & 30s & 220,456 & 8.58 \\
Exclude gap overlap (6h) & 5m & 220,246 & 9.54 \\
Exclude gap overlap (6h) & 1h & 214,616 & 9.94 \\
Exclude gap overlap (24h) & 30s & 220,456 & 8.58 \\
Exclude gap overlap (24h) & 5m & 220,246 & 9.54 \\
Exclude gap overlap (24h) & 1h & 214,616 & 9.94 \\
\bottomrule
\end{tabular}
\end{table}

\paragraph{S6.5 Horizon-standardized incidence by group (C4 robustness).}
\begin{table}[h]
\centering
\caption{Follow-up-standardized incidence by group. Primary incidence metrics
are \(p_{5\mathrm{m}}\) and \(p_{1\mathrm{h}}\) computed with horizon-specific
risk sets; \(p_{\mathrm{obs}}\) is the secondary in-window ever-reply share.
Claimed/unclaimed excludes parents with missing author identifier (\(n=906\)).
Source: \texttt{paper/tables/moltbook\_horizon\_incidence\_by\_group.csv}.}
\label{tab:supp-horizon-incidence-groups}
\small
\begin{tabular}{@{}llrrrr@{}}
\toprule
\textbf{Family} & \textbf{Group} & \textbf{Parents} & \textbf{\(p_{5\mathrm{m}}\) \%} & \textbf{\(p_{1\mathrm{h}}\) \%} & \textbf{\(p_{\mathrm{obs}}\) \%} \\
\midrule
overall & Overall & 223,316 & 9.42 & 9.82 & 9.60 \\
claimed\_status & Claimed & 20,667 & 18.95 & 19.56 & 19.23 \\
claimed\_status & Unclaimed & 201,743 & 8.48 & 8.86 & 8.65 \\
submolt\_category & Social/Casual & 189,765 & 10.80 & 11.21 & 10.95 \\
submolt\_category & Other & 16,396 & 1.98 & 2.29 & 2.26 \\
submolt\_category & Builder/Technical & 8,396 & 1.42 & 1.74 & 1.72 \\
submolt\_category & Philosophy/Meta & 5,831 & 1.32 & 1.70 & 1.80 \\
submolt\_category & Spam/Low-Signal & 2,495 & 0.88 & 0.93 & 0.88 \\
submolt\_category & Creative & 433 & 1.62 & 1.71 & 1.62 \\
\bottomrule
\end{tabular}
\end{table}

\subsection*{S7. Full Submolt Uncertainty Table}

\begin{table}[h]
\centering
\caption{Category-stratified uncertainty from
\texttt{paper/tables/moltbook\_results\_category\_uncertainty.csv}
(thread-cluster bootstrap, 400 reps, 95\% CI).}
\label{tab:supp-submolt-uncertainty}
\begingroup
\setlength{\tabcolsep}{3pt}
\scriptsize
\resizebox{\linewidth}{!}{%
\begin{tabular}{@{}lrrrrrrr@{}}
\toprule
\textbf{Category} & \textbf{Parents} & \textbf{Reply incidence \% (95\% CI)} & \textbf{$t_{50}$ (s, 95\% CI)} & \textbf{$t_{90}$ (s, 95\% CI)} & \textbf{Pooled reciprocity \% (95\% CI)} & \textbf{Re-entry mean (95\% CI)} & \textbf{Re-entry median (95\% CI)} \\
\midrule
Social/Casual & 189,765 & 10.95 [10.77, 11.13] & 4.52 [4.50, 4.54] & 24.83 [16.77, 36.12] & 0.87 [0.77, 0.96] & 0.213 [0.211, 0.216] & 0.200 [0.200, 0.200] \\
Philosophy/Meta & 5,831 & 1.80 [1.27, 2.42] & 103.00 [67.67, 154.04] & 2868.78 [310.05, 4188.38] & 1.44 [0.92, 2.18] & 0.108 [0.098, 0.119] & 0.000 [0.000, 0.000] \\
Builder/Technical & 8,396 & 1.72 [1.26, 2.19] & 103.87 [73.55, 127.29] & 434.87 [287.29, 1110.22] & 1.51 [1.03, 2.09] & 0.132 [0.123, 0.142] & 0.000 [0.000, 0.000] \\
Creative & 433 & 1.62 [0.48, 2.96] & 28.97 [9.80, 43.57] & 78.20 [28.97, 130.16] & 0.93 [0.00, 2.05] & 0.128 [0.095, 0.165] & 0.000 [0.000, 0.000] \\
Spam/Low-Signal & 2,495 & 0.88 [0.43, 1.33] & 78.88 [33.33, 166.82] & 253.22 [133.40, 276.48] & 0.60 [0.16, 1.14] & 0.144 [0.126, 0.163] & 0.000 [0.000, 0.000] \\
Other & 16,396 & 2.26 [1.84, 2.67] & 90.26 [70.50, 107.09] & 349.84 [247.15, 594.53] & 2.21 [1.73, 2.77] & 0.147 [0.139, 0.154] & 0.000 [0.000, 0.000] \\
\midrule
Overall & 223,316 & 9.60 [9.44, 9.75] & 4.55 [4.53, 4.58] & 50.05 [39.16, 60.52] & 1.00 [0.91, 1.08] & 0.195 [0.193, 0.197] & 0.167 [0.143, 0.167] \\
\bottomrule
\end{tabular}
\par}
\endgroup
\end{table}

\subsection*{S8. Data and Code Availability (migrated from main Appendix A)}
\label{sec:reproducibility}

Code and derived reproducibility artifacts for this manuscript version are
archived in a Zenodo release (\texttt{eziz2026zenodoejorrepro}). The archive
contains analysis scripts, manuscript-facing figures/tables, run manifests,
checksum manifests, and sanitized instance-level derived tables.

The Moltbook source dataset is publicly available on Hugging Face. Raw Reddit
exports are not redistributed because of platform terms; only IDs/anonymized
derivatives and aggregate outputs are shared.

\paragraph{Reproducibility recipe (canonical runs).}
\begin{itemize}
\item Create the Python 3.11 environment with \texttt{make install}.
\item Use pinned curated inputs documented in the Zenodo manifest and repository
\texttt{README.md}.
\item Re-run the three analysis pipelines with seed \texttt{20260206}.
\item Verify run manifests for Moltbook-only, Reddit-only, and matched runs.
\item Confirm manuscript artifact linkage with
\texttt{MANUSCRIPT\_ARTIFACT\_PROVENANCE.csv}.
\item Validate release integrity with \texttt{SHA256SUMS.txt}.
\item Use repository docs for environment and operational details not needed in
main text.
\end{itemize}
