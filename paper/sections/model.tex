\section{Model: Horizon-Limited Interaction Cascades}
\label{sec:model}

We model conversation dynamics on agent social networks with a generative framework
that combines age-dependent self-excitation, branching structure, and platform-level
availability modulation. The core model components in this section correspond directly
to the empirical estimands in \Cref{sec:methods}; richer formulations for hierarchical
heterogeneity and visibility-weighted reply assignment are provided in
\Cref{sec:model:extensions} in the appendix.

\subsection{Setting and Notation}
\label{sec:model:notation}

Let \(\mathcal{J}\) denote the set of threads (root posts). For thread
\(j\in\mathcal{J}\), events are indexed by \(n\in\{0,1,\ldots,N_j\}\), where
\(n=0\) is the root post and \(n\ge1\) are comments. Each event is
\((t_{jn},a_{jn},p_{jn})\), where \(t_{jn}\ge0\) is timestamp,
\(a_{jn}\in\mathcal{A}\) is author (an AI agent account), and \(p_{jn}\in\{0,\ldots,n-1\}\) is
parent index with \(p_{j0}=0\).

These parent links define a rooted reply tree \(T_j\). Depth is defined recursively:
\begin{equation}
\label{eq:depth}
d_{j0}=0, \qquad d_{jn}=d_{j,p_{jn}}+1 \quad (n\ge1).
\end{equation}
Maximum thread depth is \(D_j:=\max_{0\le n\le N_j} d_{jn}\).

Let \(\mathcal{H}_j(t)\) be thread history up to \(t\), and
\(N_j(t):=\sum_{n=1}^{N_j}\mathbf{1}\{t_{jn}\le t\}\) the cumulative
comment count by time \(t\).

\Cref{tab:notation} summarizes notation.

\begin{table}[t]
\centering
\caption{Summary of notation.}
\label{tab:notation}
\small
\begin{tabular}{@{}ll@{}}
\toprule
\textbf{Symbol} & \textbf{Description} \\
\midrule
$\calJ$ & Set of threads (root posts) \\
$\calA$ & Set of agents \\
$N_j$ & Number of comments in thread $j$ \\
$t_{jn}$ & Timestamp of event $n$ in thread $j$ \\
$a_{jn}$ & Author of event $n$ in thread $j$ \\
$p_{jn}$ & Parent index of event $n$ in thread $j$ \\
$d_{jn}$ & Depth of event $n$ in thread $j$ \\
$D_j$ & Maximum depth of thread $j$ \\
$\calH_j(t)$ & History of thread $j$ up to time $t$ \\
$b(t)$ & Aggregate availability function (attention clock) \\
$\alpha_i$ & Influence amplitude of agent $i$ \\
$\beta_i$ & Decay rate of agent $i$ \\
$h_i = \ln 2 / \beta_i$ & Half-life of agent $i$ \\
$\mu$ & Effective branching ratio \\
\bottomrule
\end{tabular}
\end{table}

\subsection{The Attention Clock: Availability and Staleness}
\label{sec:model:attention}

Agent interactions are governed by two distinct temporal mechanisms: platform-level
availability and within-thread staleness.

\begin{definition}[Availability]
\label{def:availability}
The \emph{aggregate availability function}
\(b:\mathbb{R}_{\ge0}\to\mathbb{R}_{\ge0}\) represents active-agent mass
at time \(t\), normalized so that
\(\bar b:=\frac{1}{T}\int_0^T b(t)\,dt=1\) over a long window.
\end{definition}

\begin{remark}[Observability limit]
\label{rem:availability-observability}
In observatory data, \(b(t)\) is latent because per-account heartbeat events are
not directly logged. We therefore infer heartbeat implications from aggregate
periodic signatures in timestamped activity, not from direct heartbeat-event
observation.
\end{remark}

For heartbeat-like scheduling, a tractable form is
\begin{equation}
\label{eq:availability-periodic}
b(t)=1+\kappa\cos\!\left(\frac{2\pi}{\tau}t+\phi\right), \qquad |\kappa|<1,
\end{equation}
where \(\tau\) is the characteristic check-in period, \(\kappa\) amplitude,
and \(\phi\) phase.

\begin{definition}[Staleness decay]
\label{def:staleness}
Conditional on exposure, reply propensity decays with age. For content authored by
agent \(i\) at time \(s\), staleness at \(t>s\) is
\(\exp[-\beta_i(t-s)]\), with \(\beta_i>0\).
\end{definition}

Availability \((\tau,\kappa,\phi)\) and staleness \((\beta_i)\) govern different
mechanisms and combine multiplicatively in event intensity.

\subsection{Temporal Dynamics: Self-Exciting Reply Processes}
\label{sec:model:temporal}

We model direct replies with an age-dependent Hawkes-type construction
\citep{hawkes1971spectra}.

\begin{definition}[Direct-reply intensity]
\label{def:reply-intensity}
Conditional on \(\mathcal{H}_j(t)\), each existing node \(m\) generates direct
replies as an inhomogeneous Poisson process with
\begin{equation}
\label{eq:reply-intensity}
\lambda_{j,m}(t\mid\mathcal{H}_j(t))
:= b(t)\,\alpha_{a_{jm}}\exp\!\left[-\beta_{a_{jm}}(t-t_{jm})\right]\mathbf{1}\{t>t_{jm}\},
\end{equation}
where \(\alpha_i>0\) is influence amplitude and \(\beta_i>0\) decay rate.
\end{definition}

Total thread intensity is superposition,
\begin{equation}
\label{eq:total-intensity}
\lambda_j(t\mid\mathcal{H}_j(t)):=\sum_{m:t_{jm}<t}\lambda_{j,m}(t\mid\mathcal{H}_j(t)),
\end{equation}
and parent selection follows competing risks,
\begin{equation}
\label{eq:parent-selection}
\Prob(p_{jn}=m\mid t_{jn}=t,\mathcal{H}_j(t))
=\frac{\lambda_{j,m}(t\mid\mathcal{H}_j(t))}{\lambda_j(t\mid\mathcal{H}_j(t))}.
\end{equation}

\begin{remark}[Connection to standard Hawkes processes]
\label{rem:hawkes}
Marginalizing explicit parent labels yields a marked Hawkes process with kernel
\(g_i(\Delta)=\alpha_i e^{-\beta_i\Delta}\) modulated by \(b(t)\). Hawkes-style
self-excitation is well established in social-media dynamics, though kernel
families differ across applications \citep{crane2008robust,zhao2015seismic,rizoiu2017expecting}.
\end{remark}

\begin{remark}[What is specific in this formulation]
\label{rem:model-scope}
The Hawkes kernel and branching interpretation are standard. The paper-specific
step is mechanism mapping for agent platforms: heartbeat-style activation enters
through \(b(t)\), context-window staleness enters through \(\beta_i\), and their
implications are evaluated through three linked observables, reply-kernel
half-life \((\ln 2/\beta)\), depth-tail decay \((\mu)\), and aggregate spectral
power near \(1/\tau\). This is a mechanism-grounded measurement framework, not a
new stochastic-process class.
\end{remark}

\subsection{Interaction Half-Life}
\label{sec:model:halflife}

\begin{definition}[Agent-level half-life]
\label{def:halflife}
For agent \(i\), interaction half-life is
\begin{equation}
\label{eq:halflife}
h_i:=\frac{\ln 2}{\beta_i},
\end{equation}
so that \(e^{-\beta_i h_i}=1/2\).
\end{definition}

\begin{remark}[Estimation link]
Under \cref{eq:reply-intensity}, first-reply waiting times have hazard proportional
to \(e^{-\beta\Delta}\), so \(\beta\) is estimable by likelihood-based survival
methods from parent-relative reply times. In that analysis, each non-root
candidate parent comment is an at-risk unit (``at-risk comment'').
\end{remark}

\subsection{Structural Dynamics: Branching Process Interpretation}
\label{sec:model:branching}

The direct-reply process induces a branching interpretation in which each node
produces offspring over continuous time.

\subsubsection{Expected Offspring and Branching Ratio}

For a node authored by \(i\) at time \(s\), expected direct replies are
\begin{equation}
\label{eq:offspring-mean}
\mu_i(s):=\E[\#\{\text{direct replies}\}\mid a_{jm}=i,t_{jm}=s]
=\int_0^{\infty} b(s+u)\,\alpha_i e^{-\beta_i u}\,du.
\end{equation}
A phase-averaged or slowly varying approximation gives
\begin{equation}
\label{eq:offspring-approx}
\mu_i\approx \alpha_i\int_0^{\infty} e^{-\beta_i u}du=\frac{\alpha_i}{\beta_i}.
\end{equation}

\begin{proposition}[Influence--persistence trade-off]
\label{prop:tradeoff}
Expected replies are increasing in influence \((\alpha_i)\) and decreasing in
staleness decay \((\beta_i)\). Thus, higher persistence (lower \(\beta_i\)) can
sustain larger expected engagement even at moderate \(\alpha_i\).
\end{proposition}
Proof is provided in \Cref{app:proof-tradeoff}.

Define effective branching ratio
\begin{equation}
\label{eq:branching-ratio}
\mu(s):=\E_{i\sim\pi(s)}[\mu_i(s)],
\end{equation}
with author mixture \(\pi(s)\). Under normalization, \(\mu\approx\E[\alpha_i/\beta_i]\).

\subsubsection{Subcriticality and Expected Thread Size}

\begin{assumption}[Subcriticality]
\label{ass:subcritical}
The non-root effective branching ratio satisfies \(\mu<1\).
\end{assumption}

\begin{proposition}[Expected comment count with root-special branching]
\label{prop:thread-size}
Let \(\mu_0\) be expected direct replies to the root post and \(\mu<1\) the mean
direct replies generated by a non-root comment. Then expected thread comment count is
\begin{equation}
\label{eq:expected-size}
\E[N_j]\approx\mu_0\sum_{k=0}^{\infty}\mu^k=\frac{\mu_0}{1-\mu}.
\end{equation}
The single-type special case \(\mu_0=\mu\) recovers \(\E[N_j]=\mu/(1-\mu)\).
\end{proposition}
Proof is provided in \Cref{app:proof-thread-size}.

In Moltbook-like trees, root fan-out is much larger than non-root reproduction,
so this root-special form is the relevant approximation.

\begin{remark}[Two-type branching interpretation]
\label{rem:two-type}
\Cref{prop:thread-size} corresponds to a two-type Galton--Watson process with root
offspring mean \(\mu_0\) and non-root mean \(\mu\). When
\(\mu_0\gg1\gg\mu\), trees are wide near depth 1 and shallow in deeper levels.
Depth-tail estimates should therefore be read as an approximate non-root
effective reproduction signal via tail decay, not an exact identity.
\end{remark}

\subsubsection{Depth Distribution and Tail Bounds}

\begin{proposition}[Depth tail bound]
\label{prop:depth-bound}
Under the single-type approximation, let \(Z_k\) be the generation-\(k\)
population size in the associated Galton--Watson process (\(Z_0=1\) at the
root). Then
\begin{equation}
\label{eq:depth-tail}
\Prob(D_j\ge k)\le \E[Z_k]=\mu^k.
\end{equation}
\end{proposition}

\begin{proof}
If \(D_j\ge k\), then generation-\(k\) population \(Z_k\ge1\). By Markov's
inequality, \(\Prob(D_j\ge k)\le\E[Z_k]\). Under mean offspring \(\mu\),
\(\E[Z_k]=\mu^k\).
\end{proof}

\begin{remark}[Implication for shallow threads]
\label{rem:shallow}
Deep threads become exponentially unlikely unless \(\mu\) is near one; high decay
rates (large \(\beta_i\)) imply small \(\mu_i\) and thus shallow trees.
\end{remark}

\begin{remark}[Interpretation of \(\mu\) in depth tails]
\label{rem:mu-tail-interpretation}
\Cref{eq:depth-tail} is an upper bound, not an exact identity for the empirical
depth tail. Accordingly, \(\hat{\mu}\) estimated from
\(\log \Prob(D_j\ge k)\approx c+k\log\mu\) should be read as a descriptive
effective tail-slope parameter motivated by the bound, rather than an exact
equality-based estimator of reproduction mean.
\end{remark}

\subsection{Reciprocity, Re-Entry, and Agent Heterogeneity}
\label{sec:model:heterogeneity}

For thread \(j\), define directed interaction edges
\begin{equation}
\label{eq:edge-set}
E_j:=\{(u,v): \exists n\ge1 \text{ with } a_{jn}=u,\ a_{j,p_{jn}}=v,\ u\neq v\}.
\end{equation}
Multiple replies in the same direction are collapsed to a single directed edge.
Let
\begin{equation}
\label{eq:dyad-set}
\Delta_j:=\{\{u,v\}: (u,v)\in E_j \ \text{or}\ (v,u)\in E_j\}
\end{equation}
be the unordered dyad set with at least one observed directional reply. We define
thread-level reciprocity as
\begin{equation}
\label{eq:reciprocity-rate}
\mathrm{R}_j
:=\frac{1}{|\Delta_j|}\sum_{\{u,v\}\in\Delta_j}
\mathbf{1}\{(u,v)\in E_j\ \text{and}\ (v,u)\in E_j\},
\end{equation}
with pooled reciprocity obtained by summing numerator and denominator over
threads.

We quantify sustained participation with thread-level re-entry rate
\begin{equation}
\label{eq:reentry-rate}
\mathrm{RE}_j
:=\frac{\#\{n: a_{jn}\in\{a_{j1},\ldots,a_{j,n-1}\}\}}{N_j}.
\end{equation}
Low \(\mathrm{RE}_j\) indicates broadcast-style interaction; higher values
indicate repeated participation within threads.
This definition uses comment-stream history only: root-post authorship
\(a_{j0}\) is excluded from the prior-author set unless that account appears in
the comment sequence. The scope aligns the numerator and denominator on
non-root comments (\(N_j\)).

Observed agent activity is heterogeneous. In this manuscript, empirical analysis
uses stratified pooled summaries by observable proxies (claim status,
follower-count bins), while richer hierarchical and re-entry-augmented
specifications are formalized in \Cref{sec:model:extensions}.

\subsection{Periodicity Signatures from the Attention Clock}
\label{sec:model:periodicity}

\begin{proposition}[Periodicity signatures]
\label{prop:periodicity}
If \(b(t)\) has period \(\tau\), aggregated activity implied by the model
exhibits spectral concentration at frequency \(1/\tau\) and harmonics.
\end{proposition}
Proof sketch is provided in \Cref{app:proof-periodicity}.

This prediction is tested with PSD-based target-frequency inference and
agent-level autocorrelation diagnostics.

\subsection{Connection to Platform Mechanisms}
\label{sec:model:mechanisms}

Model parameters map to design mechanisms. Heartbeat scheduling affects
availability periodicity \(b(t)\) and therefore potential spectral structure.
Memory limits and context loss affect \(\beta_i\), shifting half-life and
branching depth. Interface visibility and ranking affect which parents attract
replies; a formal visibility-weighted parent-assignment rule is given in
\Cref{sec:model:extensions}.

\subsection{Summary of Model Predictions}
\label{sec:model:summary}

\begin{table}[t]
\centering
\caption{Summary of model predictions and corresponding empirical tests.}
\label{tab:predictions}
\small
\begin{tabular}{@{}p{4cm}p{5cm}p{4.5cm}@{}}
\toprule
\textbf{Prediction} & \textbf{Mechanism} & \textbf{Empirical Test} \\
\midrule
Short interaction half-life ($h = \ln 2/\beta$) & Context limits + task switching (staleness) & Survival analysis of inter-reply times \\
Shallow conversation trees & High $\beta$ $\Rightarrow$ low $\mu$ $\Rightarrow$ exponential depth decay & Depth distribution analysis \\
Star-shaped structure & Root-reply affordances + staleness decay & Branching factor by depth level \\
Periodic activity patterns (near $\tau \approx 4$ hours) & Heartbeat synchronization $b(t)$ & Spectral analysis of timestamps \\
Heterogeneous re-entry & Variable agent re-entry propensity & Re-entry rate distribution \\
Topic moderation & Topic-conditioned persistence heterogeneity & Half-life by submolt category \\
\bottomrule
\end{tabular}
\end{table}

Row-level empirical readout is as follows: short interaction half-life \(\rightarrow\) \S\ref{sec:results:halflife}; shallow conversation trees \(\rightarrow\) \S\ref{sec:results:geometry}; star-shaped structure \(\rightarrow\) \S\ref{sec:results:geometry}; periodic activity near \(\tau\approx4\) hours \(\rightarrow\) \S\ref{sec:results:periodicity}; heterogeneous re-entry \(\rightarrow\) \S\ref{sec:results:geometry} (with overlap-region contrast in \S\ref{sec:results:comparison}); topic moderation \(\rightarrow\) \S\ref{sec:results:halflife}.
