\section{Framework and Estimands}
\label{sec:model}

This section adopts an estimands-first framework for conversation persistence in
agent social networks. The main text keeps only the measurement targets and
structural definitions used in the empirical analysis. Full continuous-time
intensity formalism, competing-risks parent-selection details, and formal
statements/proofs are deferred to supplementary material.

\subsection{Scope and Mechanism Interpretation}
\label{sec:model:scope}

We interpret persistence through two mechanisms. First, platform-level
\emph{availability} captures when agents are active and likely to observe
threads. Second, within-thread \emph{staleness} captures how reply propensity
falls as a parent comment ages.

We represent exposure/resurfacing effects through an effective participation
amplitude,
\begin{equation}
\label{eq:alpha-exposure}
\alpha_i=\bar{\alpha}_i\,\xi_i,
\end{equation}
where \(\bar{\alpha}_i\) is baseline reply propensity and \(\xi_i\) is an
exposure multiplier that absorbs visibility/ranking and resurfacing effects in
observational data. Staleness is captured by a decay-rate parameter \(\beta_i\).
The joint interpretation is mechanistic rather than causal: higher
availability/exposure and slower staleness decay are associated with greater
observed persistence.

\subsection{Structural Definitions: Depth, Branching, Reciprocity, and Re-Entry}
\label{sec:model:branching}

Let \(\mathcal{J}\) denote threads (root posts). For thread \(j\in\mathcal{J}\),
events are indexed by \(n\in\{0,1,\ldots,N_j\}\), with \(n=0\) the root and
\(n\ge 1\) comments. Event \(n\) is \((t_{jn},a_{jn},p_{jn})\), where
\(t_{jn}\) is timestamp, \(a_{jn}\) author, and \(p_{jn}\in\{0,\ldots,n-1\}\)
its parent index.

Depth is defined recursively:
\begin{equation}
\label{eq:depth}
d_{j0}=0, \qquad d_{jn}=d_{j,p_{jn}}+1 \quad (n\ge 1).
\end{equation}
Maximum thread depth is \(D_j:=\max_{0\le n\le N_j} d_{jn}\).

For node \((j,n)\), let
\(c_{jn}:=\#\{r>n: p_{jr}=n\}\) be its direct-child count. The
branching-factor profile by depth is
\begin{equation}
\label{eq:branching-by-depth}
\bar c_k:=\E\!\left[c_{jn}\mid d_{jn}=k\right],
\end{equation}
which distinguishes root-heavy star patterns from deeper cascading activity.

For thread \(j\), define directed reply edges
\begin{equation}
\label{eq:edge-set}
E_j:=\{(u,v): \exists n\ge1\ \text{with}\ a_{jn}=u,\ a_{j,p_{jn}}=v,\ u\neq v\}.
\end{equation}
Let
\begin{equation}
\label{eq:dyad-set}
\Delta_j:=\{\{u,v\}: (u,v)\in E_j\ \text{or}\ (v,u)\in E_j\}
\end{equation}
be the unordered dyads with at least one directional reply. Thread-level
reciprocity is
\begin{equation}
\label{eq:reciprocity-rate}
\mathrm{R}_j
:=\frac{1}{|\Delta_j|}\sum_{\{u,v\}\in\Delta_j}
\mathbf{1}\{(u,v)\in E_j\ \text{and}\ (v,u)\in E_j\}.
\end{equation}

Thread-level re-entry is
\begin{equation}
\label{eq:reentry-rate}
\mathrm{RE}_j
:=\frac{\#\{n: a_{jn}\in\{a_{j1},\ldots,a_{j,n-1}\}\}}{N_j}.
\end{equation}
As in the empirical implementation, this definition conditions on non-root
comments; root author \(a_{j0}\) enters only if it appears later in comments.

\subsection{Primary Estimands and Two-Part Decomposition}
\label{sec:model:estimands}
\label{sec:methods:def-est}

For each candidate parent comment \(m\), let \(T_m\) denote first direct-reply
time (if any), \(C_m\) right-censoring time from parent timestamp to observation
end, and
\[
s_m:=\min(T_m,C_m), \qquad \delta_m:=\mathbf{1}\{T_m\le C_m\}.
\]

The empirical persistence model is two-part:
\begin{align}
\delta_m &\sim \mathrm{Bernoulli}(\pi_m), \label{eq:two-part-incidence}\\
\log\!\left[-\log(1-\pi_m)\right] &= x_m^\top\eta, \label{eq:two-part-incidence-link}
\end{align}
and
\begin{equation}
T_m \mid (\delta_m=1,z_m)\sim F_\theta(\cdot\mid z_m).
\label{eq:two-part-timing}
\end{equation}
This is a \emph{hurdle / cure-style decomposition} of persistence into a
\emph{participation margin} (whether any direct reply occurs) and a
\emph{conditional timing margin} (how quickly replies arrive once participation
occurs).

Primary estimands are:
\begin{itemize}
\item \textbf{Direct-reply incidence:}
\(p_{\mathrm{obs}}:=\Prob(\delta_m=1)\), estimated as
\(N_{\mathrm{reply}}/N_{\mathrm{risk}}\).
\item \textbf{Conditional reply timing distribution:}
\(F_{T\mid\delta=1}(t):=\Prob(T_m\le t\mid\delta_m=1)\), summarized by
\((\tilde s_{0.5},\tilde s_{0.9},\tilde s_{0.95})\) and short-window
probabilities.
\item \textbf{Unconditional early-reply probability:}
\(\Prob(\delta_m=1,T_m\le t)\) for operational windows
\(t\in\{30\text{ s},5\text{ min}\}\).
\item \textbf{Structural summaries:} maximum depth \(D_j\), depth-tail slope
\(\hat s_{\mathrm{depth}}\), branching-factor profile \(\bar c_k\),
reciprocity \(\mathrm{R}_j\), and re-entry \(\mathrm{RE}_j\).
\item \textbf{Secondary timing diagnostic:}
kernel half-life diagnostic \(h=\ln 2/\beta\), interpreted as secondary to
incidence and conditional timing.
\end{itemize}
