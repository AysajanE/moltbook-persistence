\section{Background and Related Work}
\label{sec:background}

This section positions the paper against prior work that is directly
operationalized in our empirical design. We organize the review around three
pillars that map one-to-one to the measurement targets in \Cref{sec:model}:
event-time reinforcement, censored time-to-reply analysis, and
branching/cascade structure. We then place agent-populated social platforms as
the application context.

\subsection{Pillar 1: Event-Time Reinforcement and Service Availability}

Self-exciting point-process models provide a canonical way to represent
temporal reinforcement: each event raises short-run intensity for subsequent
events, with decay over time \citep{hawkes1971spectra}. This class has been
used both in operations-facing settings and in online interaction systems.
Within service operations, recent work applies Hawkes-style formulations to
co-production interaction streams \citep{daw2025coproduction}. In computational
social systems, related models have been used to describe popularity bursts,
response cascades, and diffusion timing on digital platforms
\citep{crane2008robust,zhao2015seismic,rizoiu2017expecting}. Across these
domains, the common mechanism is that recent activity makes near-term follow-up
activity more likely.

Our use of this pillar is intentionally measurement-first rather than
model-first. We do not claim that a fully specified Hawkes process is
identified as the data-generating process for Moltbook. Instead, Hawkes
intuition motivates how we interpret the availability--staleness mechanism and
why a recency-sensitive timescale is relevant. This is why the
exponential-equivalent half-life is treated as a secondary diagnostic summary,
while the primary estimands remain direct-reply incidence and conditional reply
timing.

\subsection{Pillar 2: Censored Reply-Time Analysis via Survival and Hazards}

Survival and hazard frameworks are the standard statistical foundation for
time-to-event outcomes under censoring \citep{cox1972regression}. In service
systems, these tools connect naturally to abandonment and impatience: even when
potential service opportunities exist, many cases do not complete within the
observation horizon \citep{whitt2004efficiency,reed2012hazard,jouini2010online}.
That logic maps directly to reply dynamics in threaded conversations, where
many parent--child opportunities remain unrealized during finite windows.

This paper therefore adopts a two-part persistence decomposition that is
consistent with hazard-based reasoning: an incidence margin (whether a direct
reply occurs) and a conditional timing margin (how fast it occurs when it does).
The decomposition is designed to separate distinct operational bottlenecks. Low
incidence indicates scarce engagement capacity on candidate parent comments,
whereas slower conditional timing indicates latency conditional on engagement.
This separation is central to our OR diagnosis and to the estimands reported in
the main results.

\subsection{Pillar 3: Branching/Cascade Structure as Depth Consequences}

Branching-process theory links local reproduction behavior to global cascade
geometry \citep{harris1963theory}. In subcritical regimes, trees remain shallow
and mass concentrates near roots; in higher-reproduction regimes, deeper tails
become more prevalent. This provides the structural bridge from reply-level
persistence mechanisms to thread-level coordination outcomes.

Empirical discussion-network studies provide concrete structural benchmarks.
\citet{gomez2013structure} document strong root bias in online discussion
cascades, \citet{aragon2017thread} relate threaded interfaces to reciprocity,
and \citet{meital2024branch} show that temporal and structural signals jointly
shape where replies attach in Reddit trees. We use this strand as context for
our structural readouts, specifically depth profiles, branching-by-depth,
reciprocity, and re-entry. In our framework, these are consequences of the
persistence regime rather than separate primary mechanisms.

\subsection{Application Context and Contribution Boundary}

The rise of LLM-based agent systems makes these measurement questions
operationally salient in a new domain. Early simulation evidence shows that
small groups of language-model agents can form relationships and coordinate in
controlled environments \citep{park2023generative}. Moltbook extends that
setting to a public, large-scale platform with archived interaction traces
\citep{simulamet2026observatoryarchive}, enabling direct observation of
agent-to-agent discussion dynamics.

Relative to prior literature, our contribution is not to introduce a new
stochastic-process class. Instead, we provide a mechanism-to-measurement
mapping for an agent-populated service setting: event-time reinforcement
intuition motivates timing diagnostics, survival logic motivates the two-part
incidence/timing decomposition under censoring, and branching logic links those
estimands to depth and coordination limits in observed thread structure.
