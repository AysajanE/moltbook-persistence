\section{Background and Related Work}
\label{sec:background}

We position this study in three layers: OR/MS stochastic-process foundations for service and interaction dynamics, networked conversation structure, and agent-platform applications.

\subsection{OR/MS Stochastic Foundations}

OR/MS models frequently link service dynamics to stochastic-process structure. Self-exciting point processes capture temporal reinforcement \citep{hawkes1971spectra}, survival/hazard models encode time-to-event behavior under censoring \citep{cox1972regression}, and branching-process formulations characterize propagation depth and stability in subcritical regimes \citep{harris1963theory}.

Recent OR/MS work applies these tools directly in operations contexts. Contact-center interactions have been modeled with Hawkes processes \citep{daw2025coproduction}. Queueing models in Management Science, Operations Research, and EJOR analyze impatience and abandonment under constrained capacity \citep{whitt2004efficiency,reed2012hazard,jouini2010online}. OR coordination models also connect decentralized staffing choices to service-quality outcomes in service systems \citep{ren2008callcenter}.

Computational social science contributes complementary diffusion evidence. Response-function and Hawkes-style popularity models on online platforms \citep{crane2008robust,zhao2015seismic,rizoiu2017expecting} motivate our event-time persistence perspective. Relative to that literature, our focus is OR-facing measurement: incidence, conditional timing, and depth stability under explicit availability and staleness mechanisms.

\subsection{Networked Interaction Structure}

Beyond temporal dynamics, researchers have studied the \emph{structural} properties of online discussions. Comment threads form tree structures, and the shape of these trees---their depth, breadth, and branching patterns---reveals how conversations unfold.

\citet{gomez2013structure} modeled discussion cascades using preferential attachment with root bias, showing this captures the tendency for replies to cluster near the original post---a pattern we observe strongly on Moltbook. \citet{aragon2017thread} studied the impact of conversation threading on social reciprocity, finding that threaded interfaces increase bidirectional exchange and reciprocity. More recently, \citet{meital2024branch} studied branching prediction on Reddit---whether new comments reply to leaf nodes or interior nodes---finding that structural, temporal, and linguistic features all contribute, a question directly relevant to the root-concentrated branching we observe on Moltbook.

\subsection{Agent Platforms as a New Application Domain}

The emergence of LLM-powered agents that interact autonomously with digital environments has opened new questions about collective AI behavior. \citet{park2023generative} demonstrated that 25 LLM agents in a controlled sandbox can exhibit believable social behaviors---forming relationships, initiating conversations, and coordinating events. Moltbook extends this paradigm from a controlled simulation to a deployed platform with over 25{,}000 agents, providing the first opportunity to measure LLM agent social dynamics at scale. The Observatory dataset we employ is publicly archived on Hugging Face \citep{simulamet2026observatoryarchive}.

Critical commentary has questioned the authenticity of agent autonomy on Moltbook. \citet{alexander2026afterweekend} observed that agents appear better at founding than continuing projects, with time horizons measured in hours rather than days. \citet{willison2026moltbook} characterized much of the agent output as science-fiction-themed content while acknowledging concrete demonstrations of increased agent capability. These qualitative observations motivate our quantitative investigation.

\subsection{Positioning Our Contribution}

Prior work on conversation dynamics focuses primarily on human-populated platforms. Our contribution is to extend OR/MS measurement ideas to agent-populated networks by formalizing a two-part persistence decomposition (reply incidence and conditional timing), linking these estimands to availability and staleness mechanisms, and connecting them to depth-tail stability diagnostics. This is a mechanism-to-measurement contribution for a new application domain, not a new stochastic-process class.
