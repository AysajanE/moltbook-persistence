\section{Background and Related Work}
\label{sec:background}

This section is organized around the three literature pillars that map
directly to the estimands used in \Cref{sec:model}.

\subsection{Operational Pillars Tied to Estimands}

\textbf{Event-time reinforcement (Hawkes/self-excitation).}
Self-exciting point processes formalize local reinforcement in event arrivals
\citep{hawkes1971spectra}. In this paper, this pillar supports the
availability--staleness interpretation and the secondary
exponential-equivalent half-life diagnostic.

\textbf{Censored reply-time measurement (survival/hazard).}
Survival and hazard models provide the standard framework for time-to-event
data under right censoring \citep{cox1972regression}. This directly underpins
our two-part persistence decomposition into direct-reply incidence and
conditional reply timing.

\textbf{Cascade structure (branching consequences).}
Branching-process intuition links persistence to whether cascades remain
shallow or develop deeper tails \citep{harris1963theory}. This maps to the
structural readouts used here: depth, branching-by-depth, reciprocity, and
re-entry.

\subsection{Compact Context: Discussion Networks and Agent Platforms}

Adjacent work on human online discussions documents root-biased thread
structure and variation in reciprocity \citep{gomez2013structure,aragon2017thread,meital2024branch}.
Early multi-agent social simulations and the Moltbook public archive motivate
the application domain and data source
\citep{park2023generative,simulamet2026observatoryarchive}. These studies
provide contextual grounding, while the three pillars above define the
operational framing used for this paper's methods and results.
