\section{Reproducibility}
\label{sec:reproducibility}

We are committed to open and reproducible research. This section documents resources for reproducing our analysis.

\subsection{Data Availability}
\label{sec:reproducibility:data}

\paragraph{Moltbook Observatory Archive.} Our primary dataset is publicly available on Hugging Face:
\begin{center}
\url{https://huggingface.co/datasets/SimulaMet/moltbook-observatory-archive}
\end{center}
The dataset is released under the MIT license. We provide a download/export script (\texttt{scripts/download\_moltbook\_observatory\_archive.py}) that fetches and exports the required tables to local storage and writes an export manifest (timestamp, dataset identifier, and exported splits).

\paragraph{Reddit comparison data.} Due to Reddit's terms of service, we cannot redistribute raw Reddit data. Instead, we intend to provide:
\begin{itemize}
    \item Post and comment IDs enabling re-collection via Reddit's API,
    \item Pre-computed derived features (timestamps, parent links, anonymized author hashes) sufficient for reproducing our analyses.
\end{itemize}
\todo{Finalize and document the Reddit re-collection pipeline and the exact released derived artifacts.}

\paragraph{Processed datasets.} We intend to release processed intermediate datasets (thread trees, survival times, computed metrics) in Parquet format to enable reproduction without re-running preprocessing, subject to platform terms of service and licensing constraints.

\subsection{Code Availability}
\label{sec:reproducibility:code}

All code associated with this manuscript is available in our repository:
\begin{center}
\url{https://github.com/[REDACTED]/moltbook-persistence}
\end{center}

The repository includes:
\begin{itemize}
    \item \texttt{scripts/}: Data download and export utilities.
    \item \texttt{analysis/}: Analysis entrypoints (added as the pipeline stabilizes; see \texttt{analysis/README.md} for current status).
    \item \texttt{paper/}: \LaTeX{} source for this manuscript.
    \item \texttt{requirements.txt}: Python dependencies (minimum versions).
\end{itemize}

\subsection{Computational Environment}
\label{sec:reproducibility:environment}

\todo{Describe hardware/software environment used for final runs (CPU/RAM/OS) and total compute time once the analysis pipeline is finalized.}

We target Python 3.11. A complete list of Python dependencies is provided in \texttt{requirements.txt}.

\subsection{Reproducing Key Results}
\label{sec:reproducibility:reproduce}

To reproduce the current end-to-end workflow (data export and paper build):

\begin{enumerate}
    \item \textbf{Environment setup:}
    \begin{verbatim}
    python -m venv .venv
    source .venv/bin/activate
    pip install -r requirements.txt
    \end{verbatim}

    \item \textbf{Data download:}
    \begin{verbatim}
    python scripts/download_moltbook_observatory_archive.py \
        --out-dir data/raw/moltbook-observatory-archive
    \end{verbatim}

    \item \textbf{Build the paper PDF (requires a local \LaTeX{} toolchain):}
    \begin{verbatim}
    make paper
    \end{verbatim}
\end{enumerate}

Analysis scripts to reproduce figures and tables are added under \texttt{analysis/} as the pipeline stabilizes; the current state and next planned steps are documented in \texttt{analysis/README.md}.

\subsection{Versioning and Archival}
\label{sec:reproducibility:versioning}

Upon publication, we will:
\begin{itemize}
    \item Archive the repository on Zenodo with a permanent DOI.
    \item Tag the release corresponding to the published paper version.
    \item Include checksums for downloaded data files to verify integrity.
\end{itemize}

\subsection{Contact}
\label{sec:reproducibility:contact}

For questions about reproducing this work, please contact the corresponding author or open an issue on the GitHub repository.
