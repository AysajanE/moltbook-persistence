% Abstract: 150-250 words
The emergence of social networks populated by autonomous AI agents raises fundamental
questions about collective behavior under architectural constraints absent from human
platforms. We study \emph{conversation persistence} as direct-reply hazard persistence
(how quickly responsiveness to an at-risk comment decays with comment age), distinct
from total thread duration, on Moltbook, an AI-agent social network, and
develop a generative model combining self-exciting point processes with age-dependent
branching dynamics. Using the Moltbook Observatory Archive
(January 28--February 4, 2026; 34,730 threads with comments;
199,000 candidate parent comments (at-risk units)),
we find that conversations are strongly star-shaped (mean maximum depth 1.378;
$\hat{\mu}=0.154$), exhibit very short reply-kernel half-lives (0.80 minutes; 95\% CI
[0.53, 1.13]), low direct-reply incidence (9.00\% of at-risk comments), and low dyadic
reciprocity (0.998\% bidirectional) and re-entry (mean 0.195). Because individual
heartbeat schedules are not directly observed, we test only for aggregate periodic
signatures; contrary to expectation, spectral analysis does not detect a statistically
significant $\sim$4-hour signature (AR(1)-calibrated $p=0.50$; robust across bin widths).
Half-life varies by topic: Philosophy/Meta submolts show the longest persistence
(8.1 minutes) and Social/Casual the shortest among high-volume categories
(0.69 minutes). As secondary context, a run-scoped Reddit baseline (1,772 submissions;
9,878 comments) shows mean maximum depth 2.169, reply-kernel half-life 156.4 minutes
(2.61 hours), direct-reply incidence 36.20\%, and $\Prob(D \ge 5)=11.32\%$. A coarse
matched observational comparison (813 pairs) shows large platform differences in depth,
duration, and participation under limited controls; these contrasts are associative rather
than causal. Stronger causal identification is future work.
