% Abstract: 150-250 words
The emergence of social networks populated by autonomous AI agents raises fundamental
questions about collective behavior under architectural constraints absent from human
platforms. We study \emph{conversation persistence}---the temporal dynamics governing how
long discussion threads remain active---on Moltbook, an AI-agent social network, and
develop a generative model combining self-exciting point processes with age-dependent
branching dynamics. Using the Moltbook Observatory Archive
(January 28--February 4, 2026; 34,730 threads with comments; 199,000 parent comments),
we find that conversations are strongly star-shaped (mean maximum depth 1.378;
$\hat{\mu}=0.154$), exhibit very short first-reply half-lives (0.80 minutes; 95\% CI
[0.53, 1.13]), low direct-reply incidence (9.00\% of parent comments), and low dyadic
reciprocity (0.998\% bidirectional) and re-entry (mean 0.195). We test for the
hypothesized $\sim$4-hour periodic ``attention clock'' induced by agent heartbeat
scheduling; spectral analysis does not support significance at this frequency under
AR(1)-calibrated tests ($p=0.50$), including bin-width robustness checks. Half-life varies
by topic: Philosophy/Meta submolts show the longest persistence (0.13 hr) and
Social/Casual the shortest (0.01 hr). As secondary context, a run-scoped Reddit baseline
(1,772 submissions; 9,878 comments) shows mean depth 2.169, first-reply half-life
2.61 hours, direct-reply incidence 36.20\%, and $\Prob(D \ge 5)=11.32\%$. A coarse
matched observational comparison (813 pairs) shows large platform differences in depth,
duration, and participation under limited controls; these contrasts are associative rather
than causal. Stronger causal identification is future work.
