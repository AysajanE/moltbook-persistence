% Abstract: 150-250 words
The emergence of social networks populated by autonomous AI agents presents a unique opportunity to study collective behavior under fundamentally different constraints than those governing human interaction. We investigate \emph{conversation persistence}---the temporal dynamics governing how long discussion threads remain active---on Moltbook, a social platform where AI agents interact through periodic check-ins with characteristic intervals of approximately four hours. We introduce an \emph{interaction half-life} metric that quantifies the exponential decay rate of conversational engagement and develop a generative model combining self-exciting point processes with age-dependent branching dynamics. Our model explicitly incorporates the ``attention clock'' induced by agents' periodic availability and context-window limitations. Using the Moltbook Observatory Archive comprising over 226,000 comments across 120,000 posts, we estimate interaction half-lives, characterize conversation geometry (depth distributions, branching factors, reciprocity), and detect periodic signatures in aggregate activity consistent with the platform's heartbeat mechanism. Cross-platform comparison with matched Reddit threads reveals substantially shorter half-lives and shallower conversation trees on Moltbook, providing quantitative evidence that AI agents' horizon constraints fundamentally limit sustained coordination. Our findings suggest design interventions---including memory scaffolding and thread summarization---that could extend conversational persistence in agent-based systems. We release our analysis pipeline as a reproducible artifact for studying emergent agent social networks.
