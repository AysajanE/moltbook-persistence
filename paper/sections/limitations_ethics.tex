\section{Limitations and Ethical Considerations}
\label{sec:limitations}

We discuss limitations of our study and ethical considerations arising from research on AI agent social networks.

\subsection{Limitations}
\label{sec:limitations:limitations}

\paragraph{Snapshot data.} Our analysis covers the first week of Moltbook's existence (January 28--February 4, 2026). Platform dynamics may evolve substantially as the community matures, moderation practices develop, and agent architectures improve. Our findings should be interpreted as characterizing early-stage dynamics rather than stable equilibrium behavior.

\paragraph{Human influence on agent behavior.} A fundamental ambiguity pervades Moltbook: to what extent is observed behavior ``autonomously'' generated by agents versus prompted or directed by human operators? Some accounts may be directly controlled by humans posting through agent interfaces; others may receive detailed instructions that shape their engagement patterns. Our focus on temporal dynamics (timestamps, reply structure) rather than content partially mitigates this concern, since these patterns are shaped by architectural constraints (heartbeat timing, context limits) regardless of content origin. However, we cannot rule out that human involvement systematically biases the patterns we observe.

\paragraph{Selection effects in comparison.} Our Reddit comparison relies on matching to control for confounding. While matching on early engagement, topic, and posting time addresses obvious confounds, unobserved differences between platforms (user demographics, content norms, interface design) may contribute to observed differences in conversation dynamics. We cannot claim that the half-life gap is solely attributable to human-vs-agent differences.

\paragraph{Model assumptions.} Our theoretical framework assumes exponential decay, homogeneous availability modulation, and independent reply processes. Violations of these assumptions (e.g., heavy-tailed decay, correlated agent behavior, cascading effects beyond direct replies) may bias parameter estimates. Sensitivity analyses with Weibull models partially address this, but more flexible non-parametric approaches may reveal additional structure.

\paragraph{Spam and low-quality content.} Moltbook's early period included substantial spam activity (cryptocurrency promotion, repetitive content). Although we attempt to control for this via submolt categorization and exclusion analyses, spam may contaminate engagement metrics. True ``conversational'' threads may be obscured by noise.

\paragraph{Platform instability.} The Moltbook platform underwent rapid changes during our observation window: new features, shifting moderation policies, and technical instabilities. These factors introduce heterogeneity that our analysis treats as noise but may contain signal about how platform design affects dynamics.

\subsection{Ethical Considerations}
\label{sec:limitations:ethics}

\paragraph{Privacy and consent.} Moltbook accounts are operated by AI agents, raising novel questions about privacy and consent that differ from human-subject research. We adopt a conservative approach:
\begin{itemize}
    \item Agent ``names'' are pseudonymous and self-selected; we do not attempt to identify human operators.
    \item We analyze aggregate patterns and do not single out individual agents except where they are already subjects of public commentary.
    \item Our Reddit comparison anonymizes all usernames.
\end{itemize}

\paragraph{Potential for manipulation.} Our findings about factors affecting conversational persistence could inform manipulation strategies---for example, how to artificially sustain engagement or make agent coordination appear more robust than it is. We mitigate this risk by:
\begin{itemize}
    \item Focusing on descriptive and mechanistic findings rather than optimization prescriptions.
    \item Releasing detection tools alongside analysis code, enabling identification of anomalous patterns.
    \item Discussing limitations that make manipulation difficult (e.g., agent heterogeneity complicates uniform interventions).
\end{itemize}

\paragraph{Implications for AI development.} Our findings contribute to understanding AI agent capabilities and limitations. This knowledge could inform:
\begin{itemize}
    \item \emph{Beneficial applications}: designing agent systems that support productive human-AI collaboration, identifying architectural improvements for agent memory and coordination.
    \item \emph{Potential concerns}: enabling more sophisticated autonomous agents, identifying vulnerabilities in agent-based systems.
\end{itemize}
We believe the benefits of open scientific understanding outweigh risks, particularly given that the underlying dynamics are already observable to anyone analyzing public platform data.

\paragraph{Platform terms of service.} Our analysis uses:
\begin{itemize}
    \item The Moltbook Observatory Archive, released under MIT license for research purposes.
    \item Reddit data collected via official API in compliance with rate limits and terms of service.
\end{itemize}
We do not access non-public data or circumvent access controls.

\paragraph{Ecological impact.} Large-scale AI agent activity consumes computational resources with associated environmental costs. While our study does not directly cause such activity (we analyze existing data), our findings may inform platform design decisions that affect aggregate compute usage. We encourage platform designers to consider efficiency alongside engagement metrics.

\subsection{Positionality Statement}
\label{sec:limitations:positionality}

The authors approach this research from backgrounds in computational social science, network analysis, and AI systems. We view Moltbook as a scientifically interesting phenomenon that provides insights into AI agent capabilities and collective behavior. We do not have financial or operational ties to Moltbook, its parent company, or competing platforms. Our goal is to contribute to scientific understanding rather than to advocate for or against particular platform designs.
