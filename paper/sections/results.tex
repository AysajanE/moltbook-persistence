\section{Results}
\label{sec:results}

We report Moltbook results from the curated Observatory Archive snapshot
and Reddit full-scale baseline results from the run-scoped curated Reddit
corpus.
Reproducibility details are provided in \Cref{sec:reproducibility}.
We organize Results as a three-step pipeline: (1) persistence decomposition
into incidence versus conditional timing, (2) structural signatures consistent
with that decomposition, and (3) heterogeneity by topic and claim status.
Periodicity tests and Reddit matching are retained only as compressed
secondary context.
Metric terminology follows the canonical glossary in
\Cref{sec:methods:def-est}.


\subsection{Result 1: Persistence Decomposition (Incidence vs Conditional Timing)}
\label{sec:results:decomposition}
\label{sec:results:two-part}

\subsubsection{Overall Two-Part Readout}
\label{sec:results:two-regime}

\begin{figure}[t]
\centering
\includegraphics[width=0.9\linewidth]{figures/reply_hazard_moltbook.png}
\caption{Empirical binned reply hazard versus parent age (first 10 minutes, log scale), with
fitted exponential-kernel hazard overlay.}
\label{fig:reply-hazard-moltbook}
\end{figure}

Using one survival unit per at-risk comment (candidate parent), the primary
two-part sample includes 223,316 parents with 21,430 observed direct replies
(reply incidence \(9.60\%\), 95\% bootstrap CI: [9.45\%, 9.76\%]).
The conditional median reply time is 4.55 seconds (95\% bootstrap CI:
[4.53, 4.58] seconds), with \(t_{90}=50.05\) seconds.
Claimed accounts show a 10.58 percentage-point higher incidence than unclaimed
(19.23\% vs.\ 8.65\%), and a much faster conditional upper tail
(\(t_{90}=6.29\) seconds vs.\ 63.40 seconds; \(\approx 10.1\times\)).
Unconditionally, 8.47\% of at-risk parents receive a direct reply within
30 seconds and 9.41\% within 5 minutes; conditional on a reply occurring,
these become 88.30\% and 98.06\%. This is the core low-incidence /
very-fast-conditional-speed pattern.

\begin{figure}[t]
\centering
\includegraphics[width=0.9\linewidth]{figures/reply_time_ecdf_logscale.png}
\caption{Empirical cumulative distribution function (ECDF) of conditional reply times on a log-time axis (seconds to hours),
with vertical markers at 10 seconds, 1 minute, and 1 hour.}
\label{fig:reply-time-ecdf-logscale}
\end{figure}

\Cref{fig:reply-time-ecdf-logscale} visualizes the same two-regime pattern:
Moltbook mass is concentrated at the seconds-to-minute scale, while Reddit is
substantially shifted to longer delays.

\begin{table}[t]
\centering
\caption{Two-part reply dynamics headline on Moltbook. Incidence and conditional
timing are primary; half-life is a model-implied diagnostic under the
exponential kernel.}
\label{tab:reply-dynamics}
\begingroup
\setlength{\tabcolsep}{3pt}
\scriptsize
\resizebox{\linewidth}{!}{%
\begin{tabular}{@{}lrrrrrr@{}}
\toprule
\textbf{Group} & \textbf{Parents} & \textbf{Reply incidence \% (95\% CI)} & \textbf{$t_{50}$ (s, 95\% CI)} & \textbf{$t_{90}$ (s)} & \textbf{$\Pr(\text{reply}\le 30\mathrm{s}\mid\text{reply})$ \%} & \textbf{Half-life (min; diag.)} \\
\midrule
Overall & 223,316 & 9.60 [9.45, 9.76] & 4.55 [4.53, 4.58] & 50.05 & 88.30 & 0.691 \\
Claimed & 20,667 & 19.23 [18.82, 19.66] & 4.42 [4.38, 4.45] & 6.29 & 94.99 & 0.419 \\
Unclaimed & 201,743 & 8.65 [8.50, 8.79] & 4.59 [4.57, 4.62] & 63.40 & 86.79 & 0.754 \\
\bottomrule
\end{tabular}
\par}
\endgroup
\end{table}

\subsubsection{Part-1 Incidence Model (cloglog) Summary}

The incidence model (\cref{eq:methods-incidence-cloglog}) is well calibrated at
the sample mean: observed incidence is 0.096304 and mean fitted incidence is
0.096319 (absolute error \(1.52\times 10^{-5}\), \(N=222{,}410\)).
Relative to Social/Casual (reference), all other submolts have substantially
lower incidence on the cloglog scale (coefficients from \(-1.62\) to \(-2.56\),
all two-way-clustered \(p\le 2.6\times 10^{-4}\)).
The claimed-group coefficient is positive (\(+0.844\)) but imprecise under the
two-way clustered covariance (\(p=0.181\)).

\subsubsection{Timing Shape Diagnostic}

The Weibull conditional-time fit succeeds for the overall sample and most
strata. The overall shape estimate is \(\hat{\gamma}=0.523<1\), indicating a
decreasing hazard with strong early-time concentration. This supports the
primary quantile readout in \Cref{tab:reply-dynamics}: conditional reply speed
is concentrated in the first seconds to minutes.
\Cref{tab:timing-model-fit} reports explicit observed-versus-fitted
calibration on event probability and conditional-time quantiles.
The exponential kernel is close on event probability but substantially misses
early conditional timing for Moltbook, so we treat half-life as a secondary
diagnostic and keep incidence plus conditional quantiles as primary estimands.

\begin{table}[t]
\centering
\caption{Observed vs.\ fitted timing-model diagnostics for conditional reply
times and event probability.}
\label{tab:timing-model-fit}
\scriptsize
\begin{tabular}{@{}lrrr@{}}
\toprule
\textbf{Moltbook (seconds)} & \textbf{Observed} & \textbf{Fitted} & \textbf{Residual (fit - obs)} \\
\midrule
Event probability (\%) & 9.60 & 9.91 & +0.31 \\
\(p_{10}\) (s) & 3.67 & 6.00 & +2.33 \\
\(p_{50}\) (s) & 4.55 & 39.94 & +35.39 \\
\(p_{90}\) (s) & 50.05 & 134.98 & +84.93 \\
\midrule
\textbf{Reddit (minutes)} & \textbf{Observed} & \textbf{Fitted} & \textbf{Residual (fit - obs)} \\
\midrule
Event probability (\%) & 36.20 & 38.70 & +2.50 \\
\(p_{10}\) (min) & 3.33 & 18.97 & +15.64 \\
\(p_{50}\) (min) & 39.32 & 130.59 & +91.27 \\
\(p_{90}\) (min) & 562.21 & 469.02 & -93.19 \\
\bottomrule
\end{tabular}
\end{table}

\subsection{Structural Signatures Consistent with Low Incidence / Fast Conditional Response}
\label{sec:results:structure}
\label{sec:results:geometry}
\label{sec:results:consistency}

\subsubsection{Depth Distribution}

\begin{figure}[t]
\centering
\includegraphics[width=0.9\linewidth]{figures/depth_distribution_moltbook.png}
\caption{Distribution of maximum thread depth \(D_j\) over Moltbook threads with at least
one comment (\(N=34{,}730\)); root posts are fixed at depth 0, and in this
sample \(D_j\) therefore coincides with maximum comment depth.}
\label{fig:depth-distribution}
\end{figure}

Moltbook conversation trees are shallow (\cref{fig:depth-distribution}): mean maximum depth is 1.38
(95\% bootstrap confidence interval (CI): [1.37, 1.38]), median maximum depth is 1, the proportion reaching depth 5+
is 0.006\% (95\% bootstrap CI: [0.000\%, 0.014\%]), and the proportion reaching depth 10+
is 0.000\%.

Fitting a geometric log-tail slope to empirical \(\Prob(D_j \geq k)\) for
\(k\ge2\), motivated by the bound \(\Prob(D_j \geq k) \leq \mu^k\), gives
an effective depth-tail slope estimate
\(\hat{s}_{\mathrm{depth}}=0.154\). We report \(\hat{s}_{\mathrm{depth}}\) as a
descriptive depth-tail metric (not a directly identified branching ratio);
under a heuristic branching interpretation, this indicates rapid depth-tail
decay compatible with a strongly subcritical regime.

\subsubsection{Branching Factor by Depth}

\begin{figure}[t]
\centering
\includegraphics[width=0.9\linewidth]{figures/branching_by_depth_moltbook.png}
\caption{Mean direct-children count by node depth for Moltbook threads with at
least one comment; depth 0 is the root post and depths \(\geq 1\) are non-root
comments.}
\label{fig:branching-by-depth}
\end{figure}

\Cref{fig:branching-by-depth} shows strong root concentration. The root receives 5.57 direct
replies on average, while depth-1 and depth-2 nodes receive 0.153 and 0.008 direct replies,
respectively. This is the expected star-shaped pattern under rapid branching decay.
We evaluate the corresponding thread-size consistency implication in this same
structural-signatures block (\Cref{sec:results:consistency}).

\subsubsection{Reciprocity and Re-Entry}

\begin{figure}[t]
\centering
\includegraphics[width=0.9\linewidth]{figures/reentry_distribution_moltbook.png}
\caption{Distribution of thread-level re-entry rate
\(\mathrm{RE}_j^{\mathrm{comment}}\) over Moltbook threads with at least one
comment; root-post authorship is excluded unless the root author later appears
in the comment sequence.}
\label{fig:reentry-distribution}
\end{figure}

Dyadic reciprocity is low: 1,621 of 162,430 dyads are bidirectional (0.998\%).
Reciprocal chains are short (median chain length 2; mean 2.09). Thread-level re-entry is also
limited (\cref{fig:reentry-distribution}), with mean 0.195 and median 0.167.
Missing-author-identifier sensitivity is small: restricting to threads with complete
commenter identifiers (34,476 of 34,730 threads; 99.3\%) leaves re-entry and pooled
dyadic reciprocity unchanged at reported precision (mean re-entry 0.20;
pooled reciprocity 1,621/162,430 \(=1.0\%\)). The only visible shift is
mean unique participants per thread, from 4.57 to 4.60.

\subsubsection{Geometry Consistency: Branching Heuristic vs.\ Observed Thread Size}

Applying the root-special branching approximation (\cref{prop:thread-size}) with
\(\mu_0 \approx 5.57\) (root) and effective depth-tail slope estimate
\(\hat{s}_{\mathrm{depth}} \approx 0.154\), and using the heuristic mapping
\(\mu \approx \hat{s}_{\mathrm{depth}}\), gives the heuristic expectation
\(\E[N_j] \approx \mu_0/(1-\hat{s}_{\mathrm{depth}}) \approx 6.6\) comments per
thread, closely
matching the observed mean of 6.43 (\cref{tab:descriptive}).

\subsubsection{Model-to-Observable Validation Loop}

\begin{table}[t]
\centering
\caption{Model-to-observable validation: predicted vs.\ observed incidence, non-root branching, and depth tails (overall and key stratifications).}
\label{tab:model-observable-validation}
\begingroup
\setlength{\tabcolsep}{4pt}
\scriptsize
\resizebox{\linewidth}{!} & \textbf{Obs. inc. \%} & \textbf{Pred. branch} & \textbf{Obs. branch} & \textbf{Pred. \(\Pr(D\ge3)\)} & \textbf{Obs. \(\Pr(D\ge3)\)} & \textbf{Pred. \(\Pr(D\ge5)\)} & \textbf{Obs. \(\Pr(D\ge5)\)} \\
\midrule
Overall & 9.91 & 9.60 & 0.104 & 0.134 & 0.0109 & 0.0013 & 0.00012 & 0.00001 \\
Claimed & 20.49 & 19.23 & 0.229 & 0.274 & 0.0526 & 0.0030 & 0.00276 & 0.00000 \\
Unclaimed & 8.90 & 8.65 & 0.093 & 0.120 & 0.0087 & 0.0012 & 0.00008 & 0.00001 \\
Builder/Technical & 1.72 & 1.72 & 0.017 & 0.017 & 0.0003 & 0.0001 & 0.00000 & 0.00000 \\
Creative & 1.62 & 1.62 & 0.016 & 0.016 & 0.0003 & 0.0000 & 0.00000 & 0.00000 \\
Other & 2.27 & 2.26 & 0.023 & 0.025 & 0.0005 & 0.0001 & 0.00000 & 0.00000 \\
Philosophy/Meta & 1.81 & 1.80 & 0.018 & 0.018 & 0.0003 & 0.0002 & 0.00000 & 0.00000 \\
Social/Casual & 11.36 & 10.95 & 0.121 & 0.154 & 0.0145 & 0.0015 & 0.00021 & 0.00001 \\
Spam/Low-Signal & 0.88 & 0.88 & 0.009 & 0.009 & 0.0001 & 0.0000 & 0.00000 & 0.00000 \\
\bottomrule
\end{tabular}
\par}
\endgroup
\end{table}

\Cref{tab:model-observable-validation} closes the model-to-data loop by mapping
fitted \((\alpha,\beta)\)-style primitives to observables used in the paper.
Incidence calibration is tight overall (9.91\% predicted vs.\ 9.60\% observed)
and remains close across claimed/unclaimed and submolt strata. The model
systematically overpredicts non-root branching and deeper tails
(\(\Pr(D\ge3)\), \(\Pr(D\ge5)\)), which is expected under a coarse homogeneous
branching approximation and reinforces treating depth-tail outputs as
descriptive consistency checks rather than exact structural equalities.

\subsubsection{Within-Thread Dependence Robustness}

\begin{table}[t]
\centering
\caption{One-parent-per-thread robustness against within-thread clustering dependence.}
\label{tab:dependence-robustness}
\small
\begin{tabular}{@{}lrrrr@{}}
\toprule
\textbf{Metric} & \textbf{Primary} & \textbf{One-parent/thread} & \textbf{Abs.\ diff.} & \textbf{Rel.\ diff. \%} \\
\midrule
Reply incidence \(\Pr(\delta=1)\) & 0.09596 & 0.07213 & 0.02383 & -24.84 \\
Conditional \(t_{50}\) (min) & 0.07590 & 0.07519 & 0.00070 & -0.93 \\
Conditional \(t_{90}\) (min) & 0.83416 & 0.68469 & 0.14947 & -17.92 \\
Half-life (min; diagnostic) & 0.68451 & 0.43601 & 0.24850 & -36.30 \\
\bottomrule
\end{tabular}
\end{table}

\Cref{tab:dependence-robustness} shows the one-parent-per-thread sensitivity
readout. Incidence decreases materially (9.60\% to 7.21\%), indicating that
within-thread clustering contributes to the pooled incidence level. In contrast,
the conditional median speed is nearly unchanged (0.07590 to 0.07519 minutes),
so the fast conditional-reply regime is robust to this dependence control.

\subsection{Result 3: Heterogeneity in Topic and Claim Moderation}
\label{sec:results:heterogeneity}

\subsubsection{Stratification by Submolt Category}

\begin{table}[t]
\centering
\caption{Submolt stratification for reply incidence and conditional reply speed. Half-life is included as a secondary diagnostic column.}
\label{tab:submolt-two-part}
\scriptsize
\begin{tabular}{@{}lrrrrrr@{}}
\toprule
\textbf{Category} & \textbf{Parents} & \textbf{Reply incidence \%} & \textbf{$t_{50}$ (min)} & \textbf{$t_{90}$ (min)} & \textbf{$t_{95}$ (min)} & \textbf{Half-life (min)} \\
\midrule
Builder/Technical & 8,396 & 1.72 & 1.731 & 7.248 & 18.332 & 2.979 \\
Creative & 433 & 1.62 & 0.483 & 1.303 & 1.736 & 0.453 \\
Other & 16,396 & 2.26 & 1.504 & 5.831 & 12.855 & 2.909 \\
Philosophy/Meta & 5,831 & 1.80 & 1.717 & 47.813 & 62.683 & 7.768 \\
Social/Casual & 189,765 & 10.95 & 0.075 & 0.414 & 1.776 & 0.593 \\
Spam/Low-Signal & 2,495 & 0.88 & 1.315 & 4.220 & 4.341 & 1.313 \\
\midrule
Overall & 223,316 & 9.60 & 0.076 & 0.834 & 2.204 & 0.691 \\
\bottomrule
\end{tabular}
\end{table}

The primary stratified pattern is incidence/speed heterogeneity. Social/Casual
has the highest incidence (10.95\%) and very fast conditional timing
(\(t_{90}=0.414\) minutes), whereas Builder/Technical and Philosophy/Meta have
low incidence with materially slower conditional tails.

\subsubsection{Agent Heterogeneity in Incidence and Timing}

Claim-status heterogeneity is reported directly in
\Cref{tab:reply-dynamics}: claimed accounts have higher incidence and
materially faster upper-tail conditional timing than unclaimed accounts.

\subsection{Secondary Context: Periodicity}
\label{sec:results:periodicity}

Because the canonical timeline contains a 41.7-hour gap, periodicity is
estimated on the longest contiguous segment only
(2026-02-02 04:20:50Z to 2026-02-04 19:51:53Z; 63.5 hours). Modulo-4-hour
Rayleigh concentration on this segment (\(N=220{,}461\) events) gives
\(r=0.0308\), \(Z=209.57\), Monte Carlo \(p=5\times 10^{-6}\), and circular
mean phase 153.2 minutes: statistically non-uniform but weak concentration.
Detectability simulations indicate an 80\%-power threshold
\(\kappa^\star=0.2\), so the observed concentration is well below a practically
moderate/strong periodic signal.

PSD/AR(1) and bin-width robustness checks are retained as supplementary
diagnostics (\Cref{sec:appendix:periodicity-robustness}).

\subsection{Secondary Context: Reddit Baseline and Overlap-Restricted Matching}
\label{sec:results:reddit-full-scale}

Secondary Reddit context is directionally consistent with the Moltbook
interpretation: deeper threads and slower reply persistence under the same
estimators. The overlap-restricted matched comparison is retained only as
non-causal context and represents a small overlap subset (813 matched pairs;
2.34\% of Moltbook threads). Detailed Reddit diagnostics, paired outcomes, and
matching diagnostics are provided in supplementary material
(\Cref{sec:appendix:reddit-details,sec:appendix:comparison-details,sec:appendix:match-halflife}).

\subsection{Summary of Key Findings}
\label{sec:results:summary}

The primary decomposition readout is a 9.60\% direct-reply incidence
(95\% CI: [9.45\%, 9.76\%]) with very fast
conditional timing (\(t_{50}=4.55\) seconds, 95\% CI: [4.53, 4.58] seconds;
\(t_{90}=50.05\) seconds),
plus high short-window mass (8.47\% within 30 seconds and 9.41\% within
5 minutes unconditionally). The ECDF on a log-time axis
(\Cref{fig:reply-time-ecdf-logscale}) and observed-vs-fitted quantile checks
(\Cref{tab:timing-model-fit}) both reinforce the same two-regime pattern.
Structural signatures are consistent with this regime: Moltbook threads are
predominantly star-shaped (mean maximum depth 1.38, median maximum depth 1,
effective depth-tail slope estimate \(\hat{s}_{\mathrm{depth}}=0.154\)), with rare
bidirectional dyads (0.998\%) and modest re-entry (mean 0.195; median 0.167).
Model-to-observable validation shows tight incidence calibration but
overprediction of deeper non-root branching tails, and one-parent-per-thread
robustness lowers pooled incidence while leaving conditional median speed
essentially unchanged. Topic and claim heterogeneity is substantial
(\Cref{tab:submolt-two-part,tab:reply-dynamics}). Periodicity and cross-platform
matching remain
secondary: modulo-4-hour testing detects only weak concentration
(\(r=0.0308\)) relative to the \(\kappa^\star=0.2\) detectability threshold, and
matched cross-platform contrasts are overlap-region, non-causal context.
Taken together, these findings indicate a low-incidence/fast-conditional-response
persistence regime in this early Moltbook window.

\subsubsection{H1a Readout (Incidence + Conditional Speed)}
\label{sec:results:h1a}
H1a is supported by the low-incidence and very-fast-conditional-speed split in
\Cref{sec:results:decomposition,sec:results:two-regime}.

\subsubsection{H1b Readout (4-Hour Periodicity)}
\label{sec:results:h1b}
H1b is not supported as a strong 4-hour coherence claim in this snapshot.
Modulo-4-hour Rayleigh testing detects statistically non-uniform but weak phase
concentration (\(r=0.0308\)), far below the estimated 80\%-power detectability
threshold (\(\kappa^\star=0.2\)); this is consistent with weak/dephased
periodicity rather than a practically strong aggregate 4-hour rhythm.

\subsubsection{H2 Readout (Structure vs Baseline + Reciprocity/Re-Entry)}
\label{sec:results:h2}
H2 is partially supported in this snapshot. Moltbook threads are shallow and
root-concentrated, with low pooled reciprocity and modest re-entry
(\Cref{sec:results:structure,fig:branching-by-depth,fig:reentry-distribution}).
Human-platform context under the same estimators points to deeper baseline
threads (\Cref{sec:results:reddit-full-scale,sec:appendix:reddit-details}).
The reciprocity/re-entry cross-platform contrast is conditioning-sensitive and
currently available only through overlap-restricted, non-causal matching context
(\Cref{sec:results:reddit-full-scale,sec:appendix:comparison-details}), so this
component is interpreted descriptively.

\subsubsection{H3 Readout (Topic Moderation)}
\label{sec:results:h3}
H3 is partially supported in this snapshot. Topic moderation is clear for reply
incidence and conditional timing in
\Cref{sec:results:heterogeneity,tab:submolt-two-part} (for example,
Social/Casual versus Philosophy/Meta), with corresponding diagnostic half-life
differences. Depth-related moderation appears in observed submolt tail
probabilities in \Cref{tab:model-observable-validation}, but absolute deep-tail
levels are very small; we therefore interpret topic heterogeneity as descriptive
association rather than causal effect.

\subsubsection{H4 Readout (Agent Covariates)}
\label{sec:results:h4}
H4 is partially supported in this snapshot. Claim-status covariates are
associated with large descriptive differences in incidence and conditional timing
(\Cref{sec:results:heterogeneity,tab:reply-dynamics}) and similar claimed/unclaimed stratified patterns in
\Cref{tab:model-observable-validation}. However, clustered model-based precision
for the claim indicator is limited (\Cref{sec:results:decomposition}), and
follower-count effects are not reported as a dedicated main-text readout; the
agent-covariate closure is therefore incomplete for that component.
