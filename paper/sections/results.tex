\section{Results}
\label{sec:results}

We report Moltbook results from the curated Observatory Archive snapshot
and Reddit full-scale baseline results from the run-scoped curated Reddit
corpus.
Reproducibility details are provided in \Cref{sec:reproducibility}.
The empirical chain is organized around five results: (1) thread geometry,
(2) persistence decomposition into incidence versus conditional timing,
(3) model-to-data consistency checks, (4) periodicity tests as secondary
evidence, and (5) Reddit baseline and overlap-restricted matching as contextual
comparisons.
Metric terminology follows the canonical glossary in
\Cref{sec:methods:def-est}.

\subsection{Result 1: Thread Geometry on Moltbook}
\label{sec:results:geometry}

\subsubsection{Depth Distribution}

\begin{figure}[t]
\centering
\includegraphics[width=0.9\linewidth]{figures/depth_distribution_moltbook.png}
\caption{Distribution of maximum thread depth \(D_j\) over Moltbook threads with at least
one comment (\(N=34{,}730\)); root posts are fixed at depth 0, and in this
sample \(D_j\) therefore coincides with maximum comment depth.}
\label{fig:depth-distribution}
\end{figure}

Moltbook conversation trees are shallow (\cref{fig:depth-distribution}): mean maximum depth is 1.38
(95\% bootstrap CI: [1.37, 1.38]), median maximum depth is 1, the proportion reaching depth 5+
is 0.006\% (95\% bootstrap CI: [0.000\%, 0.014\%]), and the proportion reaching depth 10+
is 0.000\%.

Fitting a geometric log-tail slope to empirical \(\Prob(D_j \geq k)\) for
\(k\ge2\), motivated by the bound \(\Prob(D_j \geq k) \leq \mu^k\), gives
\(\hat{\mu}=0.154\). We report \(\hat{\mu}\) as an effective tail-slope
parameter (descriptive), indicating rapid depth-tail decay consistent with a
strongly subcritical branching regime.

\subsubsection{Branching Factor by Depth}

\begin{figure}[t]
\centering
\includegraphics[width=0.9\linewidth]{figures/branching_by_depth_moltbook.png}
\caption{Mean direct-children count by node depth for Moltbook threads with at
least one comment; depth 0 is the root post and depths \(\geq 1\) are non-root
comments.}
\label{fig:branching-by-depth}
\end{figure}

\Cref{fig:branching-by-depth} shows strong root concentration. The root receives 5.57 direct
replies on average, while depth-1 and depth-2 nodes receive 0.153 and 0.008 direct replies,
respectively. This is the expected star-shaped pattern under rapid branching decay.
We evaluate the corresponding thread-size consistency implication explicitly in
Result 3 (\Cref{sec:results:consistency}).

\subsubsection{Reciprocity and Re-Entry}

\begin{figure}[t]
\centering
\includegraphics[width=0.9\linewidth]{figures/reentry_distribution_moltbook.png}
\caption{Distribution of thread-level re-entry rate
\(\mathrm{RE}_j^{\mathrm{comment}}\) over Moltbook threads with at least one
comment; root-post authorship is excluded unless the root author later appears
in the comment sequence.}
\label{fig:reentry-distribution}
\end{figure}

Dyadic reciprocity is low: 1,621 of 162,430 dyads are bidirectional (0.998\%).
Reciprocal chains are short (median chain length 2; mean 2.09). Thread-level re-entry is also
limited (\cref{fig:reentry-distribution}), with mean 0.195 and median 0.167.
Missing-author-ID sensitivity is small: restricting to threads with complete
commenter IDs (34,476 of 34,730 threads; 99.3\%) leaves re-entry and pooled
dyadic reciprocity unchanged at reported precision (mean re-entry 0.20;
pooled reciprocity 1,621/162,430 \(=1.0\%\)). The only visible shift is
mean unique participants per thread, from 4.57 to 4.60.

\subsection{Result 2: Two-Part Reply Dynamics (Incidence and Conditional Reply Speed)}
\label{sec:results:two-part}

\subsubsection{Overall Two-Part Readout}
\label{sec:results:two-regime}

\begin{figure}[t]
\centering
\includegraphics[width=0.9\linewidth]{figures/reply_hazard_moltbook.png}
\caption{Empirical binned reply hazard versus parent age (first 10 minutes, log scale), with
fitted exponential-kernel hazard overlay.}
\label{fig:reply-hazard-moltbook}
\end{figure}

Using one survival unit per at-risk comment (candidate parent), the primary
two-part sample includes 223,316 parents with 21,430 observed direct replies
(reply incidence 9.60\%).
\Cref{tab:reply-dynamics} reports the primary estimands: incidence and
conditional timing quantiles.
The conditional median reply time is 0.076 minutes (4.55 seconds), with
\(t_{90}=0.834\) minutes (50.0 seconds) and \(t_{95}=2.204\) minutes
(132.2 seconds). Unconditionally, 8.47\% of at-risk parents receive a direct
reply within 30 seconds and 9.41\% within 5 minutes; conditional on a reply
occurring, these become 88.30\% and 98.06\%.
This is the core low-incidence/ultra-fast-conditional-speed regime.

\begin{table}[t]
\centering
\caption{Two-part reply dynamics on Moltbook (primary readout). Reply incidence and conditional reply speed are primary; half-life is secondary diagnostic.}
\label{tab:reply-dynamics}
\begingroup
\setlength{\tabcolsep}{3pt}
\scriptsize
\resizebox{\linewidth}{!} & \textbf{$t_{50}$ (min)} & \textbf{$t_{90}$ (min)} & \textbf{$t_{95}$ (min)} & \textbf{$\Pr(\text{reply}\le 30\mathrm{s})$ \%} & \textbf{$\Pr(\text{reply}\le 5\mathrm{m})$ \%} & \textbf{$\Pr(\text{reply}\le 30\mathrm{s}\mid\text{reply})$ \%} & \textbf{$\Pr(\text{reply}\le 5\mathrm{m}\mid\text{reply})$ \%} & \textbf{Half-life (min)} \\
\midrule
Overall & 223,316 & 9.60 & 0.076 & 0.834 & 2.204 & 8.47 & 9.41 & 88.30 & 98.06 & 0.685 \\
Claimed & 20,667 & 19.23 & 0.074 & 0.105 & 0.494 & 18.27 & 18.94 & 94.99 & 98.52 & 0.408 \\
Unclaimed & 201,743 & 8.65 & 0.077 & 1.057 & 2.416 & 7.51 & 8.47 & 86.79 & 97.95 & 0.747 \\
\bottomrule
\end{tabular}
\par}
\endgroup
\end{table}

\subsubsection{Part-1 Incidence Model (cloglog) Summary}

The incidence model (\cref{eq:methods-incidence-cloglog}) is well calibrated at
the sample mean: observed incidence is 0.096304 and mean fitted incidence is
0.096319 (absolute error \(1.52\times 10^{-5}\), \(N=222{,}410\)).
Relative to Social/Casual (reference), all other submolts have substantially
lower incidence on the cloglog scale (coefficients from \(-1.62\) to \(-2.56\),
all two-way-clustered \(p\le 2.6\times 10^{-4}\)).
The claimed-group coefficient is positive (\(+0.844\)) but imprecise under the
two-way clustered covariance (\(p=0.181\)).

\subsubsection{Timing Shape Diagnostic}

The Weibull conditional-time fit succeeds for the overall sample and most
strata. The overall shape estimate is \(\hat{\gamma}=0.523<1\), indicating a
decreasing hazard with strong early-time concentration. This supports the
primary quantile readout in \Cref{tab:reply-dynamics}: conditional reply speed
is concentrated in the first seconds to minutes.

\subsubsection{Stratification by Submolt Category}

\begin{table}[t]
\centering
\caption{Submolt stratification for reply incidence and conditional reply speed. Half-life is included as a secondary diagnostic column.}
\label{tab:submolt-two-part}
\scriptsize
\begin{tabular}{@{}lrrrrrr@{}}
\toprule
\textbf{Category} & \textbf{Parents} & \textbf{Reply incidence \%} & \textbf{$t_{50}$ (min)} & \textbf{$t_{90}$ (min)} & \textbf{$t_{95}$ (min)} & \textbf{Half-life (min)} \\
\midrule
Builder/Technical & 8,396 & 1.72 & 1.731 & 7.248 & 18.332 & 2.979 \\
Creative & 433 & 1.62 & 0.483 & 1.303 & 1.736 & 0.453 \\
Other & 16,396 & 2.26 & 1.504 & 5.831 & 12.855 & 2.909 \\
Philosophy/Meta & 5,831 & 1.80 & 1.717 & 47.813 & 62.683 & 7.768 \\
Social/Casual & 189,765 & 10.95 & 0.075 & 0.414 & 1.776 & 0.593 \\
Spam/Low-Signal & 2,495 & 0.88 & 1.315 & 4.220 & 4.341 & 1.313 \\
\midrule
Overall & 223,316 & 9.60 & 0.076 & 0.834 & 2.204 & 0.685 \\
\bottomrule
\end{tabular}
\end{table}

The primary stratified pattern is incidence/speed heterogeneity. Social/Casual
has the highest incidence (10.95\%) and very fast conditional timing
(\(t_{90}=0.414\) minutes), whereas Builder/Technical and Philosophy/Meta have
low incidence with materially slower conditional tails.

\subsubsection{Agent Heterogeneity in Incidence and Timing}

\begin{table}[t]
\centering
\caption{Claim-status heterogeneity in two-part reply dynamics.}
\label{tab:agent-heterogeneity}
\small
\begin{tabular}{@{}lrrrrr@{}}
\toprule
\textbf{Group} & \textbf{Parents} & \textbf{Reply incidence \%} & \textbf{$t_{50}$ (min)} & \textbf{$t_{90}$ (min)} & \textbf{Half-life (min)} \\
\midrule
Claimed & 20,667 & 19.23 & 0.074 & 0.105 & 0.408 \\
Unclaimed & 201,743 & 8.65 & 0.077 & 1.057 & 0.747 \\
\bottomrule
\end{tabular}
\end{table}

\Cref{tab:agent-heterogeneity} shows that claimed accounts have higher
incidence (19.23\% vs.\ 8.65\%) and faster upper-tail conditional timing
(\(t_{90}=0.105\) vs.\ 1.057 minutes). Half-life remains secondary and is
consistent with the same direction (0.408 vs.\ 0.747 minutes).

\subsection{Result 3: Model-to-Observable Validation and Dependence Robustness}
\label{sec:results:consistency}

\subsubsection{Geometry Consistency: Branching Heuristic vs.\ Observed Thread Size}

Applying the root-special branching approximation (\cref{prop:thread-size}) with
\(\mu_0 \approx 5.57\) (root) and descriptive \(\hat{\mu} \approx 0.154\)
(non-root tail-slope proxy) gives the heuristic expectation
\(\E[N_j] \approx \mu_0/(1-\hat{\mu}) \approx 6.6\) comments per thread, closely
matching the observed mean of 6.43 (\cref{tab:descriptive}).

\subsubsection{Model-to-Observable Validation Loop}

\begin{table}[t]
\centering
\caption{Model-to-observable validation: predicted vs.\ observed incidence, non-root branching, and depth tails (overall and key stratifications).}
\label{tab:model-observable-validation}
\begingroup
\setlength{\tabcolsep}{4pt}
\scriptsize
\resizebox{\linewidth}{!} & \textbf{Obs. inc. \%} & \textbf{Pred. branch} & \textbf{Obs. branch} & \textbf{Pred. \(\Pr(D\ge3)\)} & \textbf{Obs. \(\Pr(D\ge3)\)} & \textbf{Pred. \(\Pr(D\ge5)\)} & \textbf{Obs. \(\Pr(D\ge5)\)} \\
\midrule
Overall & 9.91 & 9.60 & 0.104 & 0.134 & 0.0109 & 0.0013 & 0.00012 & 0.00001 \\
Claimed & 20.49 & 19.23 & 0.229 & 0.274 & 0.0526 & 0.0030 & 0.00276 & 0.00000 \\
Unclaimed & 8.90 & 8.65 & 0.093 & 0.120 & 0.0087 & 0.0012 & 0.00008 & 0.00001 \\
Builder/Technical & 1.72 & 1.72 & 0.017 & 0.017 & 0.0003 & 0.0001 & 0.00000 & 0.00000 \\
Creative & 1.62 & 1.62 & 0.016 & 0.016 & 0.0003 & 0.0000 & 0.00000 & 0.00000 \\
Other & 2.27 & 2.26 & 0.023 & 0.025 & 0.0005 & 0.0001 & 0.00000 & 0.00000 \\
Philosophy/Meta & 1.81 & 1.80 & 0.018 & 0.018 & 0.0003 & 0.0002 & 0.00000 & 0.00000 \\
Social/Casual & 11.36 & 10.95 & 0.121 & 0.154 & 0.0145 & 0.0015 & 0.00021 & 0.00001 \\
Spam/Low-Signal & 0.88 & 0.88 & 0.009 & 0.009 & 0.0001 & 0.0000 & 0.00000 & 0.00000 \\
\bottomrule
\end{tabular}
\par}
\endgroup
\end{table}

\Cref{tab:model-observable-validation} closes the model-to-data loop by mapping
fitted \((\alpha,\beta)\)-style primitives to observables used in the paper.
Incidence calibration is tight overall (9.91\% predicted vs.\ 9.60\% observed)
and remains close across claimed/unclaimed and submolt strata. The model
systematically overpredicts non-root branching and deeper tails
(\(\Pr(D\ge3)\), \(\Pr(D\ge5)\)), which is expected under a coarse homogeneous
branching approximation and reinforces treating depth-tail outputs as
descriptive consistency checks rather than exact structural equalities.

\subsubsection{Within-Thread Dependence Robustness}

\begin{table}[t]
\centering
\caption{One-parent-per-thread robustness against within-thread clustering dependence.}
\label{tab:dependence-robustness}
\small
\begin{tabular}{@{}lrrrr@{}}
\toprule
\textbf{Metric} & \textbf{Primary} & \textbf{One-parent/thread} & \textbf{Abs.\ diff.} & \textbf{Rel.\ diff. \%} \\
\midrule
Reply incidence \(\Pr(\delta=1)\) & 0.09596 & 0.07213 & 0.02383 & -24.84 \\
Conditional \(t_{50}\) (min) & 0.07590 & 0.07519 & 0.00070 & -0.93 \\
Conditional \(t_{90}\) (min) & 0.83416 & 0.68469 & 0.14947 & -17.92 \\
Half-life (min; diagnostic) & 0.68451 & 0.43601 & 0.24850 & -36.30 \\
\bottomrule
\end{tabular}
\end{table}

\Cref{tab:dependence-robustness} shows the one-parent-per-thread sensitivity
readout. Incidence decreases materially (9.60\% to 7.21\%), indicating that
within-thread clustering contributes to the pooled incidence level. In contrast,
the conditional median speed is nearly unchanged (0.07590 to 0.07519 minutes),
so the fast conditional-reply regime is robust to this dependence control.

\subsection{Result 4 (Secondary): Periodicity Tests}
\label{sec:results:periodicity}

\begin{figure}[t]
\centering
\includegraphics[width=0.9\linewidth]{figures/psd_moltbook.png}
\caption{Power spectral density of aggregate comment activity in the longest contiguous
comment-time segment.}
\label{fig:psd}
\end{figure}

Because the canonical timeline contains a 41.7-hour gap, periodicity is estimated on the
longest contiguous segment (2026-02-02 04:20:50Z to 2026-02-04 19:51:53Z; 63.5 hours).

At the 4-hour target frequency (\(f=0.25\) hr\(^{-1}\)), the peak-to-background ratio is
6.41, defined as \(P(f_\tau^\star)/\widetilde{P}_{\mathrm{bg}}\), where
\(P(f_\tau^\star)\) is Welch power at the nearest frequency bin to
\(0.25\) hr\(^{-1}\) and \(\widetilde{P}_{\mathrm{bg}}\) is the median Welch
power across positive frequencies excluding the
\(\pm 0.03\) hr\(^{-1}\) neighborhood around \(f_\tau^\star\). However, AR(1)-calibrated tests do not support significance
(Fisher's \(g=0.094\), \(p=0.686\); target-frequency power \(p=0.501\)).
The dominant spectral component is at 0.156 hr\(^{-1}\) (period 6.4 hours), not 4 hours.

The non-significant 4-hour conclusion is stable across bin widths on the same contiguous
segment: target-frequency AR(1)-calibrated \(p\)-values are 0.508 (5 min), 0.501 (15 min),
and 0.556 (30 min). The dominant period shifts from 5.33 to 8.0 hours across these binning
choices, but none imply a significant 4-hour peak; the corresponding robustness
plot is provided in \Cref{sec:appendix:periodicity-robustness}.

Event-time modulo-4-hour testing provides a complementary readout. On the same
contiguous segment (\(N=220{,}461\) events), the Rayleigh resultant length is
\(r=0.0308\), with \(Z=209.57\) and Monte Carlo
\(p=5\times 10^{-6}\); the circular mean phase is 153.2 minutes within the
4-hour cycle. The effect size is small (\(r\approx 3.1\%\) phase concentration)
despite formal significance. Detectability simulations indicate correct null
size (power \(=0.0496\) at \(\kappa=0\)) and very high power for moderate
concentration (power \(=1.0\) at \(\kappa=0.2\), with the 0.8-power crossing at
\(\kappa=0.2\)).

These findings are coherent with the non-significant AR(1)-calibrated
target-frequency PSD result: Rayleigh can detect weak but non-uniform phase
concentration at very large \(N\), while narrowband spectral tests at exactly
4 hours remain non-significant when coherence is low and schedules are
dephased/jittered.

Among 1,076 agents with at least 10 comments in the contiguous segment, mean lag-4-hour
autocorrelation in 15-minute binned activity is 0.111
(95\% bootstrap CI: [0.100, 0.122]).
This does not contradict the non-significant aggregate 4-hour spectral peak:
lag-4-hour autocorrelation reflects short-lag persistence and does not uniquely identify
a 4-hour periodic driver, and it is consistent with weak or dephased heartbeat-like
behavior that is not strong enough to produce a significant aggregate spectral peak
in this short contiguous window.

\subsection{Result 5 (Secondary Context): Reddit Baseline and Overlap-Restricted Matching}
\label{sec:results:reddit-full-scale}

The Reddit baseline corpus (\Cref{sec:data:reddit}) contains 1,772
submissions, 9,878 comments, and 1,104 threads with at least one comment. Comment timestamps
span 2026-01-31T00:03:20Z to 2026-02-04T23:59:34Z. Under the same estimators,
Reddit is deeper and slower: mean maximum depth 2.17 (vs.\ 1.38 on Moltbook),
reply incidence 36.2\% (vs.\ 9.60\%), and reply-kernel half-life 2.61 hours
(95\% CI: [2.29, 2.95], vs.\ 0.011 hours on Moltbook). Reddit periodicity also
does not show a significant 4-hour target peak under AR(1)-calibrated tests
(\(p=0.685\)), despite detectable periodic structure at other frequencies
(Fisher's \(g\), \(p<0.001\)). Full Reddit diagnostics are reported in
\Cref{sec:appendix:reddit-details}.

\subsubsection{Coarse Matched Cross-Platform Context}
\label{sec:results:comparison}

The matched comparison is used only as non-causal overlap-region context.
Under deterministic coarse matching (\Cref{sec:methods:comparison}), 813 pairs
are formed; paired means are lower for Moltbook across comments per thread,
depth, unique participants, thread duration, and re-entry
(\Cref{tab:cross-platform-paired-effects}), and all two-sided Wilcoxon
signed-rank tests reject at conventional levels (\(p < 10^{-8}\)). Because only
2.34\% of Moltbook threads appear in the matched sample, this readout should
not be interpreted as a platform-wide causal effect. Flow accounting, balance
diagnostics, paired plots, and matched-subset half-life details are reported in
\Cref{sec:appendix:comparison-details,sec:appendix:balance,sec:appendix:match-halflife}.

\subsection{Summary of Key Findings}
\label{sec:results:summary}

Overall, Moltbook threads are strongly star-shaped (mean maximum depth 1.38, median
maximum depth 1, effective tail-slope parameter \(\hat{\mu}=0.154\)), with rare bidirectional dyads (0.998\%) and
modest re-entry (mean 0.195; median 0.167). The primary two-part readout is a
9.60\% direct-reply incidence with ultra-fast conditional timing
(\(t_{50}=0.076\) minutes, \(t_{90}=0.834\) minutes, \(t_{95}=2.204\) minutes),
plus high short-window mass (8.47\% within 30 seconds and 9.41\% within
5 minutes unconditionally). Model-to-observable validation shows tight
incidence calibration but overprediction of deeper non-root branching tails.
One-parent-per-thread robustness lowers pooled incidence but leaves conditional
median speed essentially unchanged. Periodicity and cross-platform matching are
secondary: 4-hour periodicity is not detected under AR(1)-calibrated PSD tests,
while event-time Rayleigh detects only a small concentration effect, and
matched cross-platform contrasts are overlap-region, non-causal context.
Taken together, the core empirical spine supports a ``fast response or
silence'' persistence regime in this early Moltbook window.

\subsubsection{H1a Readout (Incidence + Conditional Speed)}
\label{sec:results:h1a}
H1a is supported by the low-incidence and ultra-fast-conditional-speed split in
\Cref{sec:results:two-regime}.

\subsubsection{H1b Readout (4-Hour Periodicity)}
\label{sec:results:h1b}
H1b is not detected in this snapshot under AR(1)-calibrated target-frequency
tests (\(p=0.501\) at 15-minute bins; 0.508 and 0.556 at 5- and 30-minute
bins). Event-time Rayleigh testing indicates statistically detectable but small
phase concentration (\(r=0.0308\)), consistent with weak/dephased periodicity
rather than a strong aggregate 4-hour spectral line.

In the coarse matched cross-platform design, Moltbook remains lower on comments,
depth, unique participants, duration, and re-entry in matched support
(\Cref{tab:cross-platform-paired-effects}); these are secondary contextual
overlap-region estimates rather than platform-wide causal effects. Detailed
matching diagnostics and matched-subset half-life estimates are provided in
\Cref{sec:appendix:comparison-details,sec:appendix:balance,sec:appendix:match-halflife}.
