\section{Results}
\label{sec:results}

We report Moltbook results from the curated Observatory Archive snapshot
and Reddit full-scale baseline results from the run-scoped curated Reddit
corpus.
Reproducibility details are provided in \Cref{sec:reproducibility}.
The empirical chain is organized around five results: (1) thread geometry,
(2) persistence decomposition into incidence versus conditional timing,
(3) model-to-data consistency checks, (4) periodicity tests as secondary
evidence, and (5) Reddit baseline and overlap-restricted matching as contextual
comparisons.
Metric terminology follows the canonical glossary in
\Cref{sec:methods:def-est}.

\subsection{Result 1: Thread Geometry on Moltbook}
\label{sec:results:geometry}

\subsubsection{Depth Distribution}

\begin{figure}[t]
\centering
\includegraphics[width=0.9\linewidth]{figures/depth_distribution_moltbook.png}
\caption{Distribution of maximum thread depth \(D_j\) over Moltbook threads with at least
one comment (\(N=34{,}730\)); root posts are fixed at depth 0, and in this
sample \(D_j\) therefore coincides with maximum comment depth.}
\label{fig:depth-distribution}
\end{figure}

Moltbook conversation trees are shallow (\cref{fig:depth-distribution}): mean maximum depth is 1.38
(95\% bootstrap CI: [1.37, 1.38]), median maximum depth is 1, the proportion reaching depth 5+
is 0.006\% (95\% bootstrap CI: [0.000\%, 0.014\%]), and the proportion reaching depth 10+
is 0.000\%.

Fitting a geometric log-tail slope to empirical \(\Prob(D_j \geq k)\) for
\(k\ge2\), motivated by the bound \(\Prob(D_j \geq k) \leq \mu^k\), gives
\(\hat{\mu}=0.154\). We report \(\hat{\mu}\) as an effective tail-slope
parameter (descriptive), indicating rapid depth-tail decay consistent with a
strongly subcritical branching regime.

\subsubsection{Branching Factor by Depth}

\begin{figure}[t]
\centering
\includegraphics[width=0.9\linewidth]{figures/branching_by_depth_moltbook.png}
\caption{Mean direct-children count by node depth for Moltbook threads with at
least one comment; depth 0 is the root post and depths \(\geq 1\) are non-root
comments.}
\label{fig:branching-by-depth}
\end{figure}

\Cref{fig:branching-by-depth} shows strong root concentration. The root receives 5.57 direct
replies on average, while depth-1 and depth-2 nodes receive 0.153 and 0.008 direct replies,
respectively. This is the expected star-shaped pattern under rapid branching decay.
We evaluate the corresponding thread-size consistency implication explicitly in
Result 3 (\Cref{sec:results:consistency}).

\subsubsection{Reciprocity and Re-Entry}

\begin{figure}[t]
\centering
\includegraphics[width=0.9\linewidth]{figures/reentry_distribution_moltbook.png}
\caption{Distribution of thread-level re-entry rate
\(\mathrm{RE}_j^{\mathrm{comment}}\) over Moltbook threads with at least one
comment; root-post authorship is excluded unless the root author later appears
in the comment sequence.}
\label{fig:reentry-distribution}
\end{figure}

Dyadic reciprocity is low: 1,621 of 162,430 dyads are bidirectional (0.998\%).
Reciprocal chains are short (median chain length 2; mean 2.09). Thread-level re-entry is also
limited (\cref{fig:reentry-distribution}), with mean 0.195 and median 0.167.
Missing-author-ID sensitivity is small: restricting to threads with complete
commenter IDs (34,476 of 34,730 threads; 99.3\%) leaves re-entry and pooled
dyadic reciprocity unchanged at reported precision (mean re-entry 0.20;
pooled reciprocity 1,621/162,430 \(=1.0\%\)). The only visible shift is
mean unique participants per thread, from 4.57 to 4.60.

\subsection{Result 2: Persistence Decomposition (Incidence vs.\ Conditional Timing)}
\label{sec:results:halflife}

\subsubsection{Overall Estimate and Two-Regime Finding}
\label{sec:results:two-regime}

\begin{figure}[t]
\centering
\includegraphics[width=0.9\linewidth]{figures/reply_hazard_moltbook.png}
\caption{Empirical binned reply hazard versus parent age (first 10 minutes, log scale), with
fitted exponential-kernel hazard overlay.}
\label{fig:reply-hazard-moltbook}
\end{figure}

Using one survival unit per at-risk comment (candidate parent), the primary analysis sample includes 199,000
comments (17,915 events; 181,085 right-censored), after excluding parents posted within 4
hours of the dataset end to reduce boundary censoring bias.
The fitted exponential-kernel hazard model yields:
\begin{align*}
\hat{\beta} &= 52.2 \text{ hr}^{-1} \quad \text{(95\% bootstrap CI: [36.9, 78.5])}, \\
\hat{h} &= \frac{\ln 2}{\hat{\beta}} = 0.013 \text{ hr}
\quad \text{(95\% bootstrap CI: [0.009, 0.019])}.
\end{align*}

In minutes, this corresponds to 0.80 minutes (95\% CI: [0.53, 1.13]).
Without the 4-hour boundary exclusion, the estimate is similar (0.011 hr).

This estimate should be interpreted as a \emph{kernel decay timescale}
(see \cref{rem:estimand}), not as a median thread lifetime. The combination
of a sub-minute half-life with a 91\% censoring fraction (181,085 of 199,000
at-risk comments receive no observed direct reply) describes a bursty
two-regime process: a small fraction of at-risk comments attract rapid direct replies
concentrated in the first seconds to minutes, while the large majority receive
no direct reply at all during the observation window. The Moltbook reply
process is thus better characterized as ``fast response or silence'' than as
a gradual decay of engagement.

\begin{table}[t]
\centering
\caption{Reply probability and timing decomposition (primary survival samples).}
\label{tab:reply-dynamics}
\begingroup
\setlength{\tabcolsep}{4pt}
\footnotesize
\resizebox{\linewidth}{!} & \textbf{Cond. median (min)} & \textbf{Half-life (hr)} & \textbf{Implied eventual \% (model)} \\
\midrule
Moltbook (overall) & 199,000 & 9.0 & 0.076 & 0.013 & 9.3 \\
Builder/Technical & 7,696 & 1.7 & 1.92 & 0.053 & 1.7 \\
Philosophy/Meta & 5,429 & 1.9 & 1.84 & 0.135 & 1.9 \\
Social/Casual & 168,407 & 10.3 & 0.075 & 0.011 & 10.7 \\
Creative & 366 & 1.9 & 0.483 & 0.008 & 1.9 \\
Spam/Low-Signal & 2,218 & 0.9 & 1.32 & 0.023 & 0.9 \\
Other & 14,884 & 2.1 & 1.48 & 0.054 & 2.1 \\
Reddit (overall) & 9,547 & 36.2 & 39.3 & 2.61 & 38.7 \\
\bottomrule
\end{tabular}
\par}
\endgroup
\end{table}

\textbf{Two-regime finding (incidence--timing split).}
\textbf{Result: a low-incidence / ultra-fast conditional response regime.}
\Cref{tab:reply-dynamics} separates event incidence from timing. On Moltbook, only
9.0\% of at-risk comments receive any direct reply in-window, versus 36.2\% on the
Reddit baseline. Conditional on receiving a reply, Moltbook replies arrive within seconds
to minutes (overall median 0.076~minutes, i.e., $\approx$4.6~seconds), while
Reddit conditional timing is much slower (39.3~minutes median). The implied eventual percentages are model-diagnostic quantities
under the exponential-kernel fit, not direct empirical estimands.

\subsubsection{Timing Shape Diagnostic}

The Weibull model gives shape \(\hat{\gamma}=0.112\) (95\% bootstrap CI:
[0.111, 0.112]), far below 1. This indicates strong early-time hazard concentration and
substantial departure from the fitted exponential decaying hazard
\(\lambda(s)=\alpha e^{-\beta s}\) (not from a constant-hazard baseline), with additional excess concentration at very
small \(s\).

\subsubsection{Stratification by Submolt Category}

\begin{table}[t]
\centering
\caption{Interaction half-life by submolt category in the primary survival sample.}
\label{tab:halflife-by-submolt}
\small
\begin{tabular}{@{}lrrrr@{}}
\toprule
\textbf{Category} & \textbf{Threads} & \textbf{Comments} & \textbf{Half-life (hr)} & \textbf{95\% CI} \\
\midrule
Philosophy/Meta & 1,310 & 5,429 & 0.135 & [0.048, 0.250] \\
Other & 3,882 & 14,884 & 0.054 & [0.034, 0.077] \\
Builder/Technical & 1,853 & 7,696 & 0.053 & [0.034, 0.073] \\
Spam/Low-Signal & 569 & 2,218 & 0.023 & [0.011, 0.032] \\
Social/Casual & 23,747 & 168,407 & 0.011 & [0.007, 0.017] \\
Creative & 111 & 366 & 0.008 & [0.004, 0.015] \\
\midrule
Overall & 31,472 & 199,000 & 0.013 & [0.009, 0.019] \\
\bottomrule
\end{tabular}
\end{table}

Half-life varies across categories in this first-pass estimate. Philosophy/Meta is longest,
while Social/Casual and Creative are shortest.

\subsubsection{Agent Heterogeneity in Incidence and Timing}

\begin{table}[t]
\centering
\caption{Agent-level heterogeneity in reply incidence and timing (Moltbook primary sample).}
\label{tab:agent-heterogeneity}
\small
\begin{tabular}{@{}llrrr@{}}
\toprule
\textbf{Group family} & \textbf{Group} & \textbf{Parents} & \textbf{Observed reply \%} & \textbf{Half-life (min)} \\
\midrule
\texttt{is\_claimed} & Claimed & 18,367 & 18.2 & 0.48 \\
\texttt{is\_claimed} & Unclaimed & 180,633 & 8.1 & 0.87 \\
\texttt{follower\_count} & 0 & 176,689 & 8.4 & 0.85 \\
\texttt{follower\_count} & 1--9 & 21,167 & 14.4 & 0.52 \\
\texttt{follower\_count} & 10+ & 238 & 1.7 & 1.85 \\
\bottomrule
\end{tabular}
\end{table}

\begin{figure}[t]
\centering
\includegraphics[width=0.95\linewidth]{figures/agent_group_reply_dynamics_moltbook.png}
\caption{Moltbook heterogeneity by claim status and follower-count bins.}
\label{fig:agent-group-heterogeneity}
\end{figure}

\Cref{tab:agent-heterogeneity} and \cref{fig:agent-group-heterogeneity} show that claimed
accounts have materially higher observed direct-reply incidence (18.2\% vs.\ 8.1\%) and
shorter reply-kernel half-life (0.48 vs.\ 0.87 minutes) than unclaimed accounts. The
follower-bin split is non-monotonic: \(1\text{--}9\) followers is highest-incidence in this
snapshot, while the 10+ bin is sparse (\(n=238\) at-risk comments), so that estimate is
interpreted cautiously.

\subsection{Result 3: Model-to-Data Consistency Checks}
\label{sec:results:consistency}

\subsubsection{Geometry Consistency: Branching Heuristic vs.\ Observed Thread Size}

Applying the root-special branching approximation (\cref{prop:thread-size}) with
\(\mu_0 \approx 5.57\) (root) and descriptive \(\hat{\mu} \approx 0.154\)
(non-root tail-slope proxy) gives the heuristic expectation
\(\E[N_j] \approx \mu_0/(1-\hat{\mu}) \approx 6.6\) comments per thread, closely
matching the observed mean of 6.43 (\cref{tab:descriptive}).

\subsubsection{Incidence Consistency: Observed vs.\ Model-Implied Event Probability}

In \Cref{tab:reply-dynamics}, the exponential-kernel implied eventual reply
probability is 9.3\% versus observed 9.0\% on Moltbook and 38.7\% versus 36.2\%
on Reddit. This supports using the fitted kernel as a coarse incidence-level
diagnostic, while keeping causal interpretation out of scope.

\subsubsection{Timing Consistency: Conditional Median vs.\ Kernel Timescale}

Under the fitted exponential-kernel model in the low-reply-probability regime
(\(\hat{\alpha}/\hat{\beta} \approx 0.097 \ll 1\)), the conditional distribution
of first reply time---given that a reply occurs---is approximately
\(\mathrm{Exponential}(\hat{\beta})\) (rare-event approximation), whose median
equals the kernel half-life (\(\approx 0.80\) minutes); the exact conditional
density is given in \Cref{eq:appendix-conditional-density}. The empirical
conditional median is 0.076 minutes, roughly \(10\times\) smaller. Together with
\(\hat{\gamma}=0.112 \ll 1\) and \Cref{fig:reply-hazard-moltbook}, this indicates
that the exponential kernel captures a broad decay timescale but understates
the extreme early-time concentration. We therefore treat half-life as a coarse
kernel timescale summary, not a precise predictor of conditional timing.

\subsection{Result 4 (Secondary): Periodicity Tests}
\label{sec:results:periodicity}

\begin{figure}[t]
\centering
\includegraphics[width=0.9\linewidth]{figures/psd_moltbook.png}
\caption{Power spectral density of aggregate comment activity in the longest contiguous
comment-time segment.}
\label{fig:psd}
\end{figure}

Because the canonical timeline contains a 41.7-hour gap, periodicity is estimated on the
longest contiguous segment (2026-02-02 04:20:50Z to 2026-02-04 19:51:53Z; 63.5 hours).

At the 4-hour target frequency (\(f=0.25\) hr\(^{-1}\)), the peak-to-background ratio is
6.41, defined as \(P(f_\tau^\star)/\widetilde{P}_{\mathrm{bg}}\), where
\(P(f_\tau^\star)\) is Welch power at the nearest frequency bin to
\(0.25\) hr\(^{-1}\) and \(\widetilde{P}_{\mathrm{bg}}\) is the median Welch
power across positive frequencies excluding the
\(\pm 0.03\) hr\(^{-1}\) neighborhood around \(f_\tau^\star\). However, AR(1)-calibrated tests do not support significance
(Fisher's \(g=0.094\), \(p=0.686\); target-frequency power \(p=0.501\)).
The dominant spectral component is at 0.156 hr\(^{-1}\) (period 6.4 hours), not 4 hours.

The non-significant 4-hour conclusion is stable across bin widths on the same contiguous
segment: target-frequency AR(1)-calibrated \(p\)-values are 0.508 (5 min), 0.501 (15 min),
and 0.556 (30 min). The dominant period shifts from 5.33 to 8.0 hours across these binning
choices, but none imply a significant 4-hour peak; the corresponding robustness
plot is provided in \Cref{sec:appendix:periodicity-robustness}.

Among 1,076 agents with at least 10 comments in the contiguous segment, mean lag-4-hour
autocorrelation in 15-minute binned activity is 0.111
(95\% bootstrap CI: [0.100, 0.122]).
This does not contradict the non-significant aggregate 4-hour spectral peak:
lag-4-hour autocorrelation reflects short-lag persistence and does not uniquely identify
a 4-hour periodic driver, and it is consistent with weak or dephased heartbeat-like
behavior that is not strong enough to produce a significant aggregate spectral peak
in this short contiguous window.

\subsection{Result 5 (Secondary Context): Reddit Baseline and Overlap-Restricted Matching}
\label{sec:results:reddit-full-scale}

The Reddit baseline corpus (\Cref{sec:data:reddit}) contains 1,772
submissions, 9,878 comments, and 1,104 threads with at least one comment. Comment timestamps
span 2026-01-31T00:03:20Z to 2026-02-04T23:59:34Z. Under the same estimators,
Reddit is deeper and slower: mean maximum depth 2.17 (vs.\ 1.38 on Moltbook),
reply incidence 36.2\% (vs.\ 9.0\%), and reply-kernel half-life 2.61 hours
(95\% CI: [2.29, 2.95], vs.\ 0.013 hours on Moltbook). Reddit periodicity also
does not show a significant 4-hour target peak under AR(1)-calibrated tests
(\(p=0.685\)), despite detectable periodic structure at other frequencies
(Fisher's \(g\), \(p<0.001\)). Full Reddit diagnostics are reported in
\Cref{sec:appendix:reddit-details}.

\subsubsection{Coarse Matched Cross-Platform Context}
\label{sec:results:comparison}

The matched comparison is used only as non-causal overlap-region context.
Under deterministic coarse matching (\Cref{sec:methods:comparison}), 813 pairs
are formed; paired means are lower for Moltbook across comments per thread,
depth, unique participants, thread duration, and re-entry
(\Cref{tab:cross-platform-paired-effects}), and all two-sided Wilcoxon
signed-rank tests reject at conventional levels (\(p < 10^{-8}\)). Because only
2.34\% of Moltbook threads appear in the matched sample, this readout should
not be interpreted as a platform-wide causal effect. Flow accounting, balance
diagnostics, paired plots, and matched-subset half-life details are reported in
\Cref{sec:appendix:comparison-details,sec:appendix:balance,sec:appendix:match-halflife}.

\subsection{Summary of Key Findings}
\label{sec:results:summary}

Overall, Moltbook threads are strongly star-shaped (mean maximum depth 1.38, median
maximum depth 1, effective tail-slope parameter \(\hat{\mu}=0.154\)), with rare bidirectional dyads (0.998\%) and
modest re-entry (mean 0.195; median 0.167). The primary Moltbook survival
specification yields a very short reply-kernel half-life of 0.013 hours
(0.80 minutes; 95\% CI: [0.53, 1.13] minutes), alongside a two-regime reply
pattern in which observed direct-reply incidence is 9.00\% on Moltbook versus
36.20\% on Reddit and conditional timing differs by orders of magnitude.
Consistency checks show that the branching approximation reproduces observed
mean thread size and that model-implied eventual reply probabilities are close
to observed incidence, while conditional timing remains substantially faster
than exponential-kernel prediction. Periodicity and cross-platform matching are
secondary: 4-hour periodicity is not detected under AR(1)-calibrated tests, and
matched cross-platform contrasts are overlap-region, non-causal context.
Taken together, the core empirical spine supports a ``fast response or
silence'' persistence regime in this early Moltbook window.

\subsubsection{H1a Readout (Short Hazard Half-Life)}
\label{sec:results:h1a}
H1a is supported by the minute-scale estimate (\(\hat h=0.013\) hours) and by
the labeled two-regime incidence--timing split in
\Cref{sec:results:two-regime}.

\subsubsection{H1b Readout (4-Hour Periodicity)}
\label{sec:results:h1b}
H1b is not detected in this snapshot under AR(1)-calibrated target-frequency
tests (\(p=0.501\) at 15-minute bins; 0.508 and 0.556 at 5- and 30-minute
bins).

In the coarse matched cross-platform design, Moltbook remains lower on comments,
depth, unique participants, duration, and re-entry in matched support
(\Cref{tab:cross-platform-paired-effects}); these are secondary contextual
overlap-region estimates rather than platform-wide causal effects. Detailed
matching diagnostics and matched-subset half-life estimates are provided in
\Cref{sec:appendix:comparison-details,sec:appendix:balance,sec:appendix:match-halflife}.
