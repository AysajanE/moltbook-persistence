\section{Results}
\label{sec:results}

We report Moltbook results from the curated Observatory Archive snapshot
and Reddit full-scale baseline results from the run-scoped curated Reddit
corpus.
Reproducibility and data-availability details are provided in the required
manuscript statements.
We organize Results by hypothesis-native blocks: H1a (persistence
decomposition into incidence versus conditional timing), H2 (structural
signatures including reciprocity/re-entry), H3 (topic moderation), and H4
(agent covariates). H1b periodicity tests and Reddit baseline context are retained only
as compressed secondary context.
Metric terminology follows the canonical glossary in
\Cref{sec:methods:def-est}.


\subsection{H1a: Persistence Decomposition (Incidence vs Conditional Timing)}
\label{sec:results:decomposition}
\label{sec:results:two-part}

\subsubsection{Overall Two-Part Readout}
\label{sec:results:two-regime}

Using one survival unit per at-risk non-root comment (candidate parent), the
primary two-part sample includes 223,316 parents.
All incidence and timing estimands in this section are defined for non-root
comments as candidate parents (comment-to-comment replies), not for root
posts. Primary incidence is
horizon-standardized: \(p_{5\mathrm{m}}=9.42\%\) and \(p_{1\mathrm{h}}=9.82\%\).
The in-window ever-reply share is 9.60\% (95\% bootstrap confidence interval
(CI): [9.45\%, 9.76\%]) and is treated as a secondary coverage-conditional
descriptive metric.
The conditional median reply time is 4.55 seconds (95\% bootstrap CI:
[4.53, 4.58] seconds), with \(t_{90}=50.05\) seconds.
Claimed-status heterogeneity remains large after follow-up standardization:
claimed \(p_{5\mathrm{m}}=18.95\%\), \(p_{1\mathrm{h}}=19.56\%\), versus
unclaimed \(p_{5\mathrm{m}}=8.48\%\), \(p_{1\mathrm{h}}=8.86\%\)
(\(\approx 2.2\times\)).
Conditional medians are similar (4.42 [4.38, 4.45] seconds for claimed vs.\
4.59 [4.57, 4.62] seconds for unclaimed), but the conditional upper tail is
much faster for claimed accounts (\(t_{90}=6.29\) seconds, 95\% bootstrap CI:
[6.06, 6.83], vs.\ 63.40 seconds, 95\% bootstrap CI: [52.95, 77.31];
\(\approx 10.1\times\)).
Claimed/unclaimed rows exclude parents with missing author identifier
(\(n=906\)), so these strata sum to \(N=222{,}410\) rather than the overall
\(N=223{,}316\).
Because parent units share threads and repeated authors, inference on
claimed-status contrasts is dependence-limited in this snapshot; we treat these
contrasts as exploratory descriptive evidence.
Conditional on a reply occurring, 88.30\% arrive within 30 seconds and 98.06\%
within 5 minutes. This is the core low-incidence / very-fast-conditional-speed
pattern.

\begin{figure}[t]
\centering
\includegraphics[width=0.9\linewidth]{figures/reply_time_ecdf_logscale.png}
\caption{Empirical cumulative distribution function (ECDF) of conditional reply times on a log-time axis (seconds to hours),
with vertical markers at 10 seconds, 1 minute, and 1 hour.}
\label{fig:reply-time-ecdf-logscale}
\end{figure}

\Cref{fig:reply-time-ecdf-logscale} visualizes the same two-regime pattern:
Moltbook mass is concentrated at the seconds-to-minute scale, while Reddit is
substantially shifted to longer delays.

\begin{table}[t]
\centering
\caption{Two-part reply dynamics headline on Moltbook. Horizon-standardized
incidence (\(p_{5\mathrm{m}},p_{1\mathrm{h}}\)) is primary; in-window
ever-reply share (\(p_{\mathrm{obs}}\)) and kernel half-life are secondary
descriptive diagnostics.}
\label{tab:reply-dynamics}
\begingroup
\setlength{\tabcolsep}{3pt}
\scriptsize
\resizebox{\linewidth}{!} & \textbf{\(p_{1\mathrm{h}}\) \%} & \textbf{\(p_{\mathrm{obs}}\) \% (secondary)} & \textbf{$t_{50}$ (s, 95\% CI)} & \textbf{$t_{90}$ (s)} & \textbf{Kernel half-life (diagnostic; min)} \\
\midrule
Overall & 223,316 & 9.42 & 9.82 & 9.60 & 4.55 [4.53, 4.58] & 50.05 & 0.691 \\
Claimed & 20,667 & 18.95 & 19.56 & 19.23 & 4.42 [4.38, 4.45] & 6.29 & 0.419 \\
Unclaimed & 201,743 & 8.48 & 8.86 & 8.65 & 4.59 [4.57, 4.62] & 63.40 & 0.754 \\
\addlinespace[2pt]
\multicolumn{8}{@{}l@{}}{\footnotesize Note: Claimed/unclaimed excludes parents with missing author identifier (\(n=906\)).} \\
\bottomrule
\end{tabular}
\par}
\endgroup
\end{table}

\subsubsection{Part-1 Incidence Model (cloglog) Summary}

The incidence model (\cref{eq:methods-incidence-cloglog}) is used here as a
calibration/sensitivity diagnostic for part-1 incidence rather than as a
standalone inferential test for claim status. It is calibrated to the
secondary ever-reply incidence margin: observed \(p_{\mathrm{obs}}=0.096304\)
and mean fitted incidence 0.096319 (absolute error \(1.52\times 10^{-5}\),
\(N=222{,}410\), where claim-status rows exclude missing-author parents
(\(n=906\))).
Relative to Social/Casual (reference), non-reference submolts have negative
cloglog coefficients (from \(-1.62\) to \(-2.56\)). The claimed-group
coefficient is positive (\(+0.844\)) but has wide two-way-clustered uncertainty
(95\% CI: [\(-0.39\), 2.08]). Accordingly, claim-status interpretation relies
on the follow-up-standardized descriptive contrasts in
\Cref{tab:reply-dynamics} and Supplementary Section S6.5.

\subsubsection{Timing Shape Diagnostic}

Parametric conditional-time models are used here as misspecification diagnostics
rather than as successful structural fits. Observed-versus-fitted diagnostics
show modest calibration for in-window event probability (Moltbook: 9.60\%
observed vs.\ 9.91\% fitted; Reddit: 36.20\% observed vs.\ 38.70\% fitted) but
large early-quantile errors. For Moltbook, fitted \(p_{50}\) is 39.94 seconds
versus 4.55 seconds observed, and fitted \(p_{90}\) is 134.98 seconds versus
50.05 seconds observed. Analogous quantile overstatement appears for Reddit in
Supplementary Section S6.1. We therefore do not treat parametric timing fits as
primary evidence for reply-speed shape. Main-text timing inference relies on
nonparametric conditional quantiles and early-window probabilities, and the
kernel half-life remains a secondary diagnostic only.

\subsubsection{Coverage-Gap Robustness}
\label{sec:results:gap-robustness}

Coverage-gap diagnostics indicate a comment-stream interruption in the archive
rather than complete platform inactivity (Supplementary Section S6.4).
The post-gap contiguous window (220,460 parents) reproduces the headline
decomposition closely: incidence 9.71\%, \(t_{50}=4.55\) seconds, and
\(t_{90}=48.66\) seconds versus 9.60\%, 4.55 seconds, and 50.05 seconds in the
full window (Supplementary Section S6.4). Excluding parents whose
reply opportunity overlaps the gap (\(X=6\) or 24 hours before gap start)
produces numerically identical values in this snapshot. The pre-gap contiguous
window is only 2.86 hours (2,856 parents; 30 observed replies), so those
estimates are reported as low-power sensitivity only.
Horizon-standardized risk-set probabilities remain close at short horizons
(30 seconds and 5 minutes) and rise modestly at 1 hour, consistent with
short-follow-up censoring near the observation boundary
(Supplementary Section S6.4).

\subsection{H2: Structural Signatures Consistent with Low Incidence / Fast Conditional Response}
\label{sec:results:structure}
\label{sec:results:geometry}
\label{sec:results:consistency}

\subsubsection{Depth Distribution}

\begin{figure}[t]
\centering
\includegraphics[width=0.9\linewidth]{figures/depth_distribution_moltbook.png}
\caption{Distribution of maximum thread depth \(D_j\) over Moltbook threads with at least
one comment (\(N=34{,}730\)); root posts are fixed at depth 0, and in this
sample \(D_j\) therefore coincides with maximum comment depth.}
\label{fig:depth-distribution}
\end{figure}

Moltbook conversation trees are shallow (\cref{fig:depth-distribution}): mean maximum depth is 1.38
(95\% bootstrap confidence interval (CI): [1.37, 1.38]), median maximum depth is 1, the proportion reaching depth 5+
is 0.006\% (95\% bootstrap CI: [0.000\%, 0.014\%]), and the proportion reaching depth 10+
is 0.000\%.

Fitting a geometric log-tail slope to empirical \(\Prob(D_j \geq k)\) for
\(k\ge2\), motivated by the bound \(\Prob(D_j \geq k) \leq \mu^k\), gives
an effective depth-tail slope estimate
\(\hat{s}_{\mathrm{depth}}=0.154\). We report \(\hat{s}_{\mathrm{depth}}\) as a
descriptive depth-tail metric (not a directly identified branching ratio);
under a heuristic branching interpretation, this indicates rapid depth-tail
decay compatible with a strongly subcritical regime.

\subsubsection{Branching Factor by Depth}

\begin{figure}[t]
\centering
\includegraphics[width=0.9\linewidth]{figures/branching_by_depth_moltbook.png}
\caption{Mean direct-children count by node depth for Moltbook threads with at
least one comment; depth 0 is the root post and depths \(\geq 1\) are non-root
comments.}
\label{fig:branching-by-depth}
\end{figure}

\Cref{fig:branching-by-depth} shows strong root concentration. The root receives 5.57 direct
replies on average, while depth-1 and depth-2 nodes receive 0.153 and 0.008 direct replies,
respectively. This is the expected star-shaped pattern under rapid branching decay.
We evaluate the corresponding thread-size consistency implication in this same
structural-signatures block (\Cref{sec:results:consistency}).

\subsubsection{Reciprocity and Re-Entry}

\begin{figure}[t]
\centering
\includegraphics[width=0.9\linewidth]{figures/reentry_distribution_moltbook.png}
\caption{Distribution of thread-level re-entry rate
\(\mathrm{RE}_j^{\mathrm{comment}}\) over Moltbook threads with at least one
comment; root-post authorship is excluded unless the root author later appears
in the comment sequence.}
\label{fig:reentry-distribution}
\end{figure}

Dyadic reciprocity is low: 1,621 of 162,430 dyads are bidirectional (0.998\%).
Reciprocal chains are short (median chain length 2; mean 2.09). Thread-level re-entry is also
limited (\cref{fig:reentry-distribution}), with mean 0.195 and median 0.167.
Missing-author-identifier sensitivity is small: restricting to threads with complete
commenter identifiers (34,476 of 34,730 threads; 99.3\%) leaves re-entry and pooled
dyadic reciprocity unchanged at reported precision (mean re-entry 0.20;
pooled reciprocity 1,621/162,430 \(=1.0\%\)). The only visible shift is
mean unique participants per thread, from 4.57 to 4.60.

\subsubsection{Geometry Coherence (Back-of-the-Envelope): Branching Heuristic vs.\ Observed Thread Size}

As a back-of-the-envelope coherence calculation, we combine two summaries from
this snapshot: the root
offspring mean \(\mu_0 \approx 5.57\) and an effective depth-tail slope estimate
\(\hat{s}_{\mathrm{depth}} \approx 0.154\). Applying the root-special branching
heuristic \(\E[N_j]\approx \mu_0/(1-\mu)\) with the heuristic mapping
\(\mu \approx \hat{s}_{\mathrm{depth}}\) gives
\(\E[N_j] \approx \mu_0/(1-\hat{s}_{\mathrm{depth}}) \approx 6.6\) comments per
thread, close to the observed mean of 6.43 (\cref{tab:descriptive}).

\subsubsection{Model-to-Observable Coherence Check}

Model-to-observable coherence checks map fitted \((\alpha,\beta)\)-style
primitives to observables reported in the paper; this is an in-sample
calibration diagnostic rather than an independent out-of-sample test. Incidence
calibration is tight overall (9.91\% predicted vs.\ 9.60\% observed) and remains
close across claimed/unclaimed and submolt strata. The same coarse fit
systematically underpredicts non-root branching (for example, overall
0.104 predicted vs.\ 0.134 observed) while overpredicting deeper tails
\((\Pr(D\ge3),\Pr(D\ge5))\). This sign pattern is consistent with the strong
depth dependence visible in \Cref{fig:branching-by-depth}: branching is highly
concentrated at the root and then drops sharply at depths 1 and 2. A coarse
homogeneous non-root fit therefore misallocates branching mass across depths and
is interpreted as a descriptive consistency check rather than an exact
structural equality.

\subsubsection{Within-Thread Dependence Robustness}

One-parent-per-thread sensitivity lowers pooled incidence from 9.60\% to 7.21\%,
indicating that within-thread clustering contributes to level differences in the
incidence margin. Conditional timing changes are materially smaller: \(t_{50}\)
shifts from 4.55 to 4.51 seconds and \(t_{90}\) from 50.05 to 41.08 seconds.
The kernel half-life diagnostic correspondingly decreases from 0.685 to 0.436
minutes. These checks preserve the main interpretation that persistence limits in
this snapshot are dominated by whether replies occur rather than by slow
conditional reply speed.

\subsection{H3: Topic Moderation in Two-Part Dynamics}
\label{sec:results:heterogeneity}

\subsubsection{Stratification by Submolt Category}

\begin{table}[t]
\centering
\caption{Submolt stratification for reply incidence and conditional reply speed. Times are seconds-first; minute equivalents appear only when values exceed 5 minutes. The kernel half-life diagnostic is included only as a secondary column.}
\label{tab:submolt-two-part}
\begingroup
\setlength{\tabcolsep}{3pt}
\scriptsize
\resizebox{\linewidth}{!} & \textbf{$t_{50}$ (s)} & \textbf{$t_{90}$ (s)} & \textbf{$t_{95}$ (s)} & \textbf{Kernel half-life (diagnostic; min)} \\
\midrule
Builder/Technical & 8,396 & 1.72 & 103.87 & 434.87 (7.25 min) & 1099.93 (18.33 min) & 2.979 \\
Creative & 433 & 1.62 & 28.97 & 78.20 & 104.18 & 0.453 \\
Other & 16,396 & 2.26 & 90.26 & 349.84 (5.83 min) & 771.27 (12.85 min) & 2.909 \\
Philosophy/Meta & 5,831 & 1.80 & 103.00 & 2868.78 (47.81 min) & 3760.96 (62.68 min) & 7.768 \\
Social/Casual & 189,765 & 10.95 & 4.52 & 24.83 & 106.58 & 0.593 \\
Spam/Low-Signal & 2,495 & 0.88 & 78.88 & 253.22 & 260.44 & 1.313 \\
\midrule
Overall & 223,316 & 9.60 & 4.55 & 50.05 & 132.23 & 0.691 \\
\bottomrule
\end{tabular}
\par}
\endgroup
\end{table}

The primary stratified pattern is incidence/speed heterogeneity. Social/Casual
has the highest incidence (10.95\%) and very fast conditional timing
(\(t_{90}=24.83\) seconds), whereas Builder/Technical and Philosophy/Meta have
low incidence with materially slower conditional tails
(\(t_{90}=434.87\) seconds [7.25 minutes] and \(2868.78\) seconds
[47.81 minutes], respectively).
Category-cluster bootstrap intervals preserve this ranking (Supplementary
Section S7): Social/Casual remains high-incidence/fast-tail, while
Philosophy/Meta remains low-incidence/slow-tail with low median re-entry.
Follow-up-standardized incidence at fixed horizons yields the same ordering
(Supplementary Section S6.5): for example, Social/Casual has
\(p_{5\mathrm{m}}=10.80\%\) and \(p_{1\mathrm{h}}=11.21\%\), while
Philosophy/Meta has \(p_{5\mathrm{m}}=1.32\%\) and \(p_{1\mathrm{h}}=1.70\%\).
Because submolt categories are assigned by deterministic keyword triggers on
submolt names, we report keyword-mapping sensitivity checks in Supplementary
Section S2.1; the Social/Casual versus Philosophy/Meta incidence/tail ordering
is unchanged under alternative trigger lists and exclusion rules.

\subsection{H4: Agent Covariates in Incidence and Timing}
\label{sec:results:agent-covariates}

Claim-status heterogeneity is reported directly in
\Cref{tab:reply-dynamics}: claimed accounts have higher incidence and
materially faster upper-tail conditional timing than unclaimed accounts.

\subsection{Secondary Context for H1b: Periodicity}
\label{sec:results:periodicity}

On the longest contiguous segment (63.5 hours; \(N=220{,}461\) events),
modulo-4-hour concentration is small (\(r=0.0308\)) even though Rayleigh
testing rejects exact uniformity (\(Z=209.57\), Monte Carlo
\(p=5\times10^{-6}\)). At this sample size, the detectability simulation
reports \(\kappa^\star=0.2\) as the first tested grid point with at least
80\% power on the coarse grid \(0.0,0.2,\ldots\), so it should not be read as
a sharp threshold. The corresponding mean resultant length at \(\kappa=0.2\)
is \(\rho\approx0.0995\), still well above observed \(r=0.0308\). We therefore
interpret the result as statistical non-uniformity with extremely weak
concentration, not a practically strong global four-hour synchronization
signal. Supplementary PSD/AR(1) and bin-width diagnostics similarly show no
dominant four-hour line.

\subsection{Secondary Context: Reddit Baseline under Shared Estimators}
\label{sec:results:reddit-full-scale}

Secondary Reddit context is retained only as descriptive orientation under
shared estimators. We do not treat these cross-platform contrasts as an
identification anchor because platforms differ in interface, ranking,
moderation, and participation incentives, and because curated Reddit data
include known curation losses. Matched-overlap support is sparse (813 pairs;
2.34\% of Moltbook threads), and matched-subset survival on Moltbook has only
22 events. Accordingly, baseline mechanics, overlap diagnostics, and matched
analyses are deferred to Supplementary Sections S3.3 and S4--S4.4.

\subsection{Summary of Key Findings}
\label{sec:results:summary}

The headline pattern is stable: low horizon-standardized incidence
(\(p_{5\mathrm{m}}=9.42\%\), \(p_{1\mathrm{h}}=9.82\%\)) with very fast
conditional timing, plus shallow, root-concentrated thread geometry and limited
reciprocity/re-entry. Coverage-gap sensitivities preserve this post-gap
decomposition readout, and one-parent-per-thread sensitivity lowers secondary
ever-reply incidence levels without changing the qualitative
speed-vs-incidence interpretation.
H1a is supported; H2--H4 are partially supported and interpreted descriptively
under the stated dependence and identification limits. Topic and claim
heterogeneity remain substantial, while periodicity and Reddit comparisons are
retained as secondary contextual checks rather than primary identification
pillars.
