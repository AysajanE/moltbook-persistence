\section{Results}
\label{sec:results}

We report Moltbook results from curated snapshot
\texttt{snapshot\_20260204-234429Z}, with all estimates generated by
\texttt{analysis/06\_moltbook\_only\_analysis.py}
(run ID \texttt{run\_20260206-145240Z}), and Reddit full-scale baseline results from
\texttt{analysis/07\_reddit\_only\_analysis.py} (run ID
\texttt{attempt\_scaled\_20260206-142651Z}).

\subsection{Conversation Geometry on Moltbook}
\label{sec:results:geometry}

\subsubsection{Depth Distribution}

\begin{figure}[t]
\centering
\includegraphics[width=0.9\linewidth]{figures/depth_distribution_moltbook.png}
\caption{Distribution of maximum thread depth \(D_j\) over Moltbook threads with at least
one comment (\(N=34{,}730\)).}
\label{fig:depth-distribution}
\end{figure}

Moltbook conversation trees are shallow (\cref{fig:depth-distribution}):
\begin{itemize}
    \item Mean depth: 1.378 (95\% bootstrap CI: [1.372, 1.383]),
    \item Median depth: 1,
    \item Proportion reaching depth 5+: 0.0058\% (95\% bootstrap CI: [0.0000\%, 0.0144\%]),
    \item Proportion reaching depth 10+: 0.0000\%.
\end{itemize}

Fitting the depth-tail relation \(\Prob(D_j \geq k) \leq \mu^k\) gives
\(\hat{\mu}=0.154\), consistent with a strongly subcritical branching regime.

\subsubsection{Branching Factor by Depth}

\begin{figure}[t]
\centering
\includegraphics[width=0.9\linewidth]{figures/branching_by_depth_moltbook.png}
\caption{Mean direct-children count by node depth.}
\label{fig:branching-by-depth}
\end{figure}

\Cref{fig:branching-by-depth} shows strong root concentration. The root receives 5.57 direct
replies on average, while depth-1 and depth-2 nodes receive 0.153 and 0.008 direct replies,
respectively. This is the expected star-shaped pattern under rapid branching decay.

\subsubsection{Reciprocity and Re-Entry}

\begin{figure}[t]
\centering
\includegraphics[width=0.9\linewidth]{figures/reentry_distribution_moltbook.png}
\caption{Distribution of thread-level re-entry rate \(\mathrm{RE}_j\).}
\label{fig:reentry-distribution}
\end{figure}

Dyadic reciprocity is low: 1,621 of 162,430 dyads are bidirectional (0.998\%).
Reciprocal chains are short (median chain length 2; mean 2.09). Thread-level re-entry is also
limited (\cref{fig:reentry-distribution}), with mean 0.195 and median 0.167.

\subsection{Interaction Half-Life}
\label{sec:results:halflife}

\subsubsection{Overall Estimate}

\begin{figure}[t]
\centering
\includegraphics[width=0.9\linewidth]{figures/survival_curve_moltbook.png}
\caption{Kaplan--Meier survival curve and fitted exponential model for time to first direct
reply.}
\label{fig:survival-curve}
\end{figure}

Using one survival unit per parent comment, the primary analysis sample includes 199,000
comments (17,915 events; 181,085 right-censored), after excluding parents posted within 4
hours of the dataset end to reduce boundary censoring bias.
The fitted exponential model yields:
\begin{align*}
\hat{\beta} &= 52.209 \text{ hr}^{-1} \quad \text{(95\% bootstrap CI: [36.852, 78.524])}, \\
\hat{h} &= \frac{\ln 2}{\hat{\beta}} = 0.0133 \text{ hr}
\quad \text{(95\% bootstrap CI: [0.0088, 0.0188])}.
\end{align*}

In minutes, this corresponds to 0.80 minutes (95\% CI: [0.53, 1.13]).
The estimate is very short because observed direct replies are concentrated in the first
seconds to minutes, while most parent comments receive no direct reply in-window.
Without the 4-hour boundary exclusion, the estimate is similar (0.0115 hr).

\subsubsection{Weibull Shape Parameter}

The Weibull model gives shape \(\hat{\gamma}=0.1115\) (95\% bootstrap CI:
[0.1110, 0.1122]), far below 1. This indicates strong early-time hazard concentration and
substantial deviation from a constant-hazard assumption.

\subsubsection{Stratification by Submolt Category}

\begin{table}[t]
\centering
\caption{Interaction half-life by submolt category in the primary survival sample.}
\label{tab:halflife-by-submolt}
\small
\begin{tabular}{@{}lrrrr@{}}
\toprule
\textbf{Category} & \textbf{Threads} & \textbf{Comments} & \textbf{Half-life (hr)} & \textbf{95\% CI} \\
\midrule
Philosophy/Meta & 1,310 & 5,429 & 0.1346 & [0.0481, 0.2502] \\
Other & 3,882 & 14,884 & 0.0535 & [0.0343, 0.0771] \\
Builder/Technical & 1,853 & 7,696 & 0.0529 & [0.0339, 0.0733] \\
Spam/Low-Signal & 569 & 2,218 & 0.0226 & [0.0115, 0.0323] \\
Social/Casual & 23,747 & 168,407 & 0.0115 & [0.0067, 0.0171] \\
Creative & 111 & 366 & 0.0076 & [0.0035, 0.0152] \\
\midrule
Overall & 31,472 & 199,000 & 0.0133 & [0.0088, 0.0188] \\
\bottomrule
\end{tabular}
\end{table}

Half-life varies across categories in this first-pass estimate. Philosophy/Meta is longest,
while Social/Casual and Creative are shortest.

\subsubsection{Agent Heterogeneity}

For a simple heterogeneity check, we compare pooled half-life estimates for comments authored
by agents in the bottom vs.\ top karma quartile among observed karma values. Because both
quartile cutoffs are 0 in this snapshot (high mass at zero karma), this split is weakly
informative. Estimated half-lives are similar: 0.0140 hr (bottom quartile) vs.\ 0.0133 hr
(top quartile).

\subsection{Periodicity Detection}
\label{sec:results:periodicity}

\subsubsection{Spectral Analysis}

\begin{figure}[t]
\centering
\includegraphics[width=0.9\linewidth]{figures/psd_moltbook.png}
\caption{Power spectral density of aggregate comment activity in the longest contiguous
comment-time segment.}
\label{fig:psd}
\end{figure}

Because the canonical timeline contains a 41.72-hour gap, periodicity is estimated on the
longest contiguous segment (2026-02-02 04:20:50Z to 2026-02-04 19:51:53Z; 63.52 hours).

At the 4-hour target frequency (\(f=0.25\) hr\(^{-1}\)), the peak-to-background ratio is
6.41. However, AR(1)-calibrated tests do not support significance
(Fisher's \(g=0.0937\), \(p=0.697\); target-frequency power \(p=0.4905\)).
The dominant spectral component is at 0.15625 hr\(^{-1}\) (period 6.4 hours), not 4 hours.

\subsubsection{Agent-Level Autocorrelation}

Among 1,076 agents with at least 10 comments in the contiguous segment, mean lag-4-hour
autocorrelation in 15-minute binned activity is 0.1106
(95\% bootstrap CI: [0.0997, 0.1221]).

\subsection{Reddit Full-Scale Baseline}
\label{sec:results:reddit-full-scale}

The Reddit baseline run (\texttt{attempt\_scaled\_20260206-142651Z}) contains 1,772
submissions, 9,878 comments, and 1,104 threads with at least one comment. Comment timestamps
span 2026-01-31T00:03:20Z to 2026-02-04T23:59:34Z. Thread-level descriptive summaries are
reported in \texttt{tables/reddit\_descriptive\_stats.csv}.

Promoted artifacts for this subsection are
\texttt{figures/depth\_distribution\_reddit.png},
\texttt{figures/branching\_by\_depth\_reddit.png},
\texttt{figures/reentry\_distribution\_reddit.png},
\texttt{figures/survival\_curve\_reddit.png}, and
\texttt{figures/psd\_reddit.png}.

\subsubsection{Geometry, Branching, and Re-Entry}

\begin{figure}[t]
\centering
\includegraphics[width=0.9\linewidth]{figures/depth_distribution_reddit.png}
\caption{Distribution of maximum thread depth \(D_j\) for the Reddit run
\texttt{attempt\_scaled\_20260206-142651Z}.}
\label{fig:depth-distribution-reddit}
\end{figure}

\begin{figure}[t]
\centering
\includegraphics[width=0.9\linewidth]{figures/branching_by_depth_reddit.png}
\caption{Mean direct-children count by node depth for the Reddit run.}
\label{fig:branching-by-depth-reddit}
\end{figure}

\begin{figure}[t]
\centering
\includegraphics[width=0.9\linewidth]{figures/reentry_distribution_reddit.png}
\caption{Distribution of thread-level re-entry rate \(\mathrm{RE}_j\) for the Reddit run.}
\label{fig:reentry-distribution-reddit}
\end{figure}

Conversation geometry is materially deeper than a pure star regime but still root-concentrated:
\begin{itemize}
    \item Mean maximum depth per thread: 2.169; median: 1,
    \item \(\Prob(D \ge 5)=11.32\%\) and \(\Prob(D \ge 10)=1.54\%\),
    \item Mean branching factor at depth 0: 4.305; at depth 1: 0.446,
    \item Re-entry rate mean: 0.0938; median: 0.
\end{itemize}

\subsubsection{First-Reply Half-Life}

\begin{figure}[t]
\centering
\includegraphics[width=0.9\linewidth]{figures/survival_curve_reddit.png}
\caption{Kaplan--Meier survival curve and fitted exponential model for the Reddit run.}
\label{fig:survival-curve-reddit}
\end{figure}

The primary survival sample includes 9,547 parent comments (3,456 events; 6,091 censored),
after excluding 329 parents within the 4-hour right-boundary window. The exponential estimate
gives a first-reply half-life of 2.6066 hours (95\% CI: [2.2933, 2.9539]). A no-boundary
sensitivity estimate is similar at 2.5755 hours.

\subsubsection{Periodicity}

\begin{figure}[t]
\centering
\includegraphics[width=0.9\linewidth]{figures/psd_reddit.png}
\caption{Power spectral density of aggregate Reddit comment activity in the run window.}
\label{fig:psd-reddit}
\end{figure}

Spectral analysis finds a dominant frequency of 0.03125 hr\(^{-1}\), corresponding to a
32-hour period. For the 4-hour target frequency test, AR(1)-calibrated target-power
\(p=0.6845\). Fisher's \(g\) test yields \(p<0.0005\) (0/2000 exceedances), indicating
detectable periodic structure at some frequency but not specifically at 4 hours under this
test.

\subsubsection{Run-Scoped Caveats}

Two upstream caveats recorded in run manifests are carried into interpretation: 1,570 comments
were dropped during curation due to missing submission IDs, and the request log includes 2
non-200/error responses. These caveats do not invalidate the run, but they bound claims to the
curated, observed corpus.

\subsection{Cross-Platform Comparison}
\label{sec:results:comparison}

Matched Moltbook--Reddit inference (coarsened exact matching and paired effect estimation) is
not yet executed in this revision. The Moltbook and Reddit sections above are platform-side
descriptive baselines and should not be interpreted as causal or matched comparative effects.

\subsection{Summary of Key Findings}
\label{sec:results:summary}

\begin{enumerate}
    \item \textbf{Shallow structure}: Moltbook threads are strongly star-shaped
    (mean depth 1.378; median depth 1; \(\hat{\mu}=0.154\)).
    \item \textbf{Low reciprocity}: Bidirectional dyads are rare (0.998\%), and reciprocal
    chains are short.
    \item \textbf{Limited re-entry}: Thread-level re-entry is modest
    (mean 0.195; median 0.167).
    \item \textbf{Very short first-reply half-life}: Estimated half-life is 0.0133 hours
    (0.80 minutes; 95\% CI: [0.53, 1.13] minutes) in the primary survival specification.
    \item \textbf{No statistically significant 4-hour spectral peak}: Although power at
    0.25 hr\(^{-1}\) is elevated relative to local background, AR(1)-based tests are not
    significant.
    \item \textbf{Reddit baseline geometry and timing}: In
    \texttt{attempt\_scaled\_20260206-142651Z}, Reddit threads show mean max depth 2.169
    (\(\Prob(D \ge 5)=11.32\%\), \(\Prob(D \ge 10)=1.54\%\)), re-entry mean 0.0938, and
    first-reply half-life 2.6066 hours (95\% CI: [2.2933, 2.9539]).
    \item \textbf{Reddit periodicity finding}: Dominant periodic component is 32 hours
    (0.03125 hr\(^{-1}\)); the 4-hour target-frequency test is not significant
    (\(p=0.6845\)), while Fisher's \(g\) reports \(p<0.0005\) (0/2000 exceedances).
\end{enumerate}
