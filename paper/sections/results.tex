\section{Results}
\label{sec:results}

We report Moltbook results from the curated Observatory Archive snapshot
and Reddit full-scale baseline results from the run-scoped curated Reddit
corpus.
Reproducibility and data-availability details are provided in the required
manuscript statements.
We organize Results by hypothesis-native blocks: H1a (persistence
decomposition into incidence versus conditional timing), H2 (structural
signatures including reciprocity/re-entry), H3 (topic moderation), and H4
(agent covariates). H1b periodicity tests and Reddit baseline context are retained only
as compressed secondary context.
Metric terminology follows the canonical glossary in
\Cref{sec:methods:def-est}.


\subsection{H1a: Persistence Decomposition (Incidence vs Conditional Timing)}
\label{sec:results:decomposition}
\label{sec:results:two-part}

\subsubsection{Overall Two-Part Readout}
\label{sec:results:two-regime}

Using one survival unit per at-risk comment (candidate parent), the primary
two-part sample includes 223,316 parents with 21,430 observed direct replies
(reply incidence \(9.60\%\), 95\% bootstrap CI: [9.45\%, 9.76\%]).
The conditional median reply time is 4.55 seconds (95\% bootstrap CI:
[4.53, 4.58] seconds), with \(t_{90}=50.05\) seconds.
Claimed-status heterogeneity is large descriptively: claimed incidence is
19.23\% (95\% bootstrap CI: [18.82\%, 19.66\%]) versus 8.65\%
(95\% bootstrap CI: [8.50\%, 8.79\%]) for unclaimed, an absolute gap of
10.58 percentage points (\(\approx 2.22\times\)).
Conditional medians are similar (4.42 [4.38, 4.45] seconds for claimed vs.\
4.59 [4.57, 4.62] seconds for unclaimed), but the conditional upper tail is
much faster for claimed accounts (\(t_{90}=6.29\) seconds, 95\% bootstrap CI:
[6.06, 6.83], vs.\ 63.40 seconds, 95\% bootstrap CI: [52.95, 77.31];
\(\approx 10.1\times\)).
Because parent units share threads and repeated authors, inference on
claimed-status contrasts is dependence-limited in this snapshot, and these
contrasts are treated as descriptive.
Unconditionally, 8.47\% of at-risk parents receive a direct reply within
30 seconds and 9.41\% within 5 minutes; conditional on a reply occurring,
these become 88.30\% and 98.06\%. This is the core low-incidence /
very-fast-conditional-speed pattern.

\begin{figure}[t]
\centering
\includegraphics[width=0.9\linewidth]{figures/reply_time_ecdf_logscale.png}
\caption{Empirical cumulative distribution function (ECDF) of conditional reply times on a log-time axis (seconds to hours),
with vertical markers at 10 seconds, 1 minute, and 1 hour.}
\label{fig:reply-time-ecdf-logscale}
\end{figure}

\Cref{fig:reply-time-ecdf-logscale} visualizes the same two-regime pattern:
Moltbook mass is concentrated at the seconds-to-minute scale, while Reddit is
substantially shifted to longer delays.

\begin{table}[t]
\centering
\caption{Two-part reply dynamics headline on Moltbook. Incidence and conditional
timing are primary; the kernel half-life diagnostic is model-implied under the
exponential kernel and reported as secondary.}
\label{tab:reply-dynamics}
\begingroup
\setlength{\tabcolsep}{3pt}
\scriptsize
\resizebox{\linewidth}{!}{%
\begin{tabular}{@{}lrrrrrrr@{}}
\toprule
\textbf{Group} & \textbf{Parents} & \textbf{Reply incidence \% (95\% CI)} & \textbf{$t_{50}$ (s, 95\% CI)} & \textbf{$t_{90}$ (s)} & \textbf{$\Pr(\text{reply}\le 30\mathrm{s}\mid\text{reply})$ \%} & \textbf{$\Pr(\text{reply}\le 5\mathrm{m}\mid\text{reply})$ \%} & \textbf{Kernel half-life (diagnostic; min)} \\
\midrule
Overall & 223,316 & 9.60 [9.45, 9.76] & 4.55 [4.53, 4.58] & 50.05 & 88.30 & 98.06 & 0.691 \\
Claimed & 20,667 & 19.23 [18.82, 19.66] & 4.42 [4.38, 4.45] & 6.29 & 94.99 & 98.52 & 0.419 \\
Unclaimed & 201,743 & 8.65 [8.50, 8.79] & 4.59 [4.57, 4.62] & 63.40 & 86.79 & 97.95 & 0.754 \\
\bottomrule
\end{tabular}
\par}
\endgroup
\end{table}

\subsubsection{Part-1 Incidence Model (cloglog) Summary}

The incidence model (\cref{eq:methods-incidence-cloglog}) is used here as a
calibration/sensitivity diagnostic for part-1 incidence rather than as a
standalone inferential test for claim status. Calibration at the sample mean is
tight: observed incidence is 0.096304 and mean fitted incidence is 0.096319
(absolute error \(1.52\times 10^{-5}\), \(N=222{,}410\)).
Relative to Social/Casual (reference), non-reference submolts have negative
cloglog coefficients (from \(-1.62\) to \(-2.56\)). The claimed-group
coefficient is positive (\(+0.844\)) but has wide two-way-clustered uncertainty
(95\% CI: [\(-0.39\), 2.08]). Accordingly, claim-status interpretation relies
on the bootstrap descriptive contrasts in \Cref{tab:reply-dynamics}.

\subsubsection{Timing Shape Diagnostic}

The Weibull conditional-time fit succeeds for the overall sample and most
strata. The overall shape estimate is \(\hat{\gamma}=0.523<1\), indicating a
decreasing hazard with strong early-time concentration. This supports the
primary quantile readout in \Cref{tab:reply-dynamics}: conditional reply speed
is concentrated in the first seconds to minutes. Observed-versus-fitted
diagnostics show close event-probability calibration (Moltbook: 9.60\% observed
vs.\ 9.91\% fitted; Reddit: 36.20\% observed vs.\ 38.70\% fitted) but large
early-quantile misspecification. For Moltbook, fitted \(p_{50}\) is 39.94
seconds versus 4.55 seconds observed, and fitted \(p_{90}\) is 134.98 seconds
versus 50.05 seconds observed; analogous overstatement appears in Reddit upper
quantiles. Empirical hazard-shape mismatch diagnostics are documented in
supplementary analyses. Accordingly, the kernel half-life remains a secondary diagnostic and
incidence plus conditional quantiles remain the primary measurement targets.

\subsection{H2: Structural Signatures Consistent with Low Incidence / Fast Conditional Response}
\label{sec:results:structure}
\label{sec:results:geometry}
\label{sec:results:consistency}

\subsubsection{Depth Distribution}

\begin{figure}[t]
\centering
\includegraphics[width=0.9\linewidth]{figures/depth_distribution_moltbook.png}
\caption{Distribution of maximum thread depth \(D_j\) over Moltbook threads with at least
one comment (\(N=34{,}730\)); root posts are fixed at depth 0, and in this
sample \(D_j\) therefore coincides with maximum comment depth.}
\label{fig:depth-distribution}
\end{figure}

Moltbook conversation trees are shallow (\cref{fig:depth-distribution}): mean maximum depth is 1.38
(95\% bootstrap confidence interval (CI): [1.37, 1.38]), median maximum depth is 1, the proportion reaching depth 5+
is 0.006\% (95\% bootstrap CI: [0.000\%, 0.014\%]), and the proportion reaching depth 10+
is 0.000\%.

Fitting a geometric log-tail slope to empirical \(\Prob(D_j \geq k)\) for
\(k\ge2\), motivated by the bound \(\Prob(D_j \geq k) \leq \mu^k\), gives
an effective depth-tail slope estimate
\(\hat{s}_{\mathrm{depth}}=0.154\). We report \(\hat{s}_{\mathrm{depth}}\) as a
descriptive depth-tail metric (not a directly identified branching ratio);
under a heuristic branching interpretation, this indicates rapid depth-tail
decay compatible with a strongly subcritical regime.

\subsubsection{Branching Factor by Depth}

\begin{figure}[t]
\centering
\includegraphics[width=0.9\linewidth]{figures/branching_by_depth_moltbook.png}
\caption{Mean direct-children count by node depth for Moltbook threads with at
least one comment; depth 0 is the root post and depths \(\geq 1\) are non-root
comments.}
\label{fig:branching-by-depth}
\end{figure}

\Cref{fig:branching-by-depth} shows strong root concentration. The root receives 5.57 direct
replies on average, while depth-1 and depth-2 nodes receive 0.153 and 0.008 direct replies,
respectively. This is the expected star-shaped pattern under rapid branching decay.
We evaluate the corresponding thread-size consistency implication in this same
structural-signatures block (\Cref{sec:results:consistency}).

\subsubsection{Reciprocity and Re-Entry}

\begin{figure}[t]
\centering
\includegraphics[width=0.9\linewidth]{figures/reentry_distribution_moltbook.png}
\caption{Distribution of thread-level re-entry rate
\(\mathrm{RE}_j^{\mathrm{comment}}\) over Moltbook threads with at least one
comment; root-post authorship is excluded unless the root author later appears
in the comment sequence.}
\label{fig:reentry-distribution}
\end{figure}

Dyadic reciprocity is low: 1,621 of 162,430 dyads are bidirectional (0.998\%).
Reciprocal chains are short (median chain length 2; mean 2.09). Thread-level re-entry is also
limited (\cref{fig:reentry-distribution}), with mean 0.195 and median 0.167.
Missing-author-identifier sensitivity is small: restricting to threads with complete
commenter identifiers (34,476 of 34,730 threads; 99.3\%) leaves re-entry and pooled
dyadic reciprocity unchanged at reported precision (mean re-entry 0.20;
pooled reciprocity 1,621/162,430 \(=1.0\%\)). The only visible shift is
mean unique participants per thread, from 4.57 to 4.60.

\subsubsection{Geometry Consistency: Branching Heuristic vs.\ Observed Thread Size}

Applying the root-special branching heuristic
\(\E[N_j]\approx \mu_0/(1-\mu)\) with
\(\mu_0 \approx 5.57\) (root) and effective depth-tail slope estimate
\(\hat{s}_{\mathrm{depth}} \approx 0.154\), and using the heuristic mapping
\(\mu \approx \hat{s}_{\mathrm{depth}}\), gives the heuristic expectation
\(\E[N_j] \approx \mu_0/(1-\hat{s}_{\mathrm{depth}}) \approx 6.6\) comments per
thread, closely
matching the observed mean of 6.43 (\cref{tab:descriptive}).

\subsubsection{Model-to-Observable Validation Loop}

Model-to-observable validation closes the model-to-data loop by mapping fitted
\((\alpha,\beta)\)-style primitives to observables used in the paper. Incidence
calibration is tight overall (9.91\% predicted vs.\ 9.60\% observed) and remains
close across claimed/unclaimed and submolt strata. The same coarse fit
systematically overpredicts non-root branching and deeper tails
\((\Pr(D\ge3),\Pr(D\ge5))\). Under a coarse homogeneous branching approximation,
these depth-tail outputs are therefore interpreted as descriptive consistency
checks rather than exact structural equalities.

\subsubsection{Within-Thread Dependence Robustness}

One-parent-per-thread sensitivity lowers pooled incidence from 9.60\% to 7.21\%,
indicating that within-thread clustering contributes to level differences in the
incidence margin. Conditional timing changes are materially smaller: \(t_{50}\)
shifts from 4.55 to 4.51 seconds and \(t_{90}\) from 50.05 to 41.08 seconds.
The kernel half-life diagnostic correspondingly decreases from 0.685 to 0.436
minutes. These checks preserve the main interpretation that persistence limits in
this snapshot are driven primarily by whether replies occur rather than by slow
conditional reply speed.

\subsection{H3: Topic Moderation in Two-Part Dynamics}
\label{sec:results:heterogeneity}

\subsubsection{Stratification by Submolt Category}

\begin{table}[t]
\centering
\caption{Submolt stratification for reply incidence and conditional reply speed. Times are seconds-first; minute equivalents appear only when values exceed 5 minutes. The kernel half-life diagnostic is included only as a secondary column.}
\label{tab:submolt-two-part}
\begingroup
\setlength{\tabcolsep}{3pt}
\scriptsize
\resizebox{\linewidth}{!} & \textbf{$t_{50}$ (s)} & \textbf{$t_{90}$ (s)} & \textbf{$t_{95}$ (s)} & \textbf{Kernel half-life (diagnostic; min)} \\
\midrule
Builder/Technical & 8,396 & 1.72 & 103.87 & 434.87 (7.25 min) & 1099.93 (18.33 min) & 2.979 \\
Creative & 433 & 1.62 & 28.97 & 78.20 & 104.18 & 0.453 \\
Other & 16,396 & 2.26 & 90.26 & 349.84 (5.83 min) & 771.27 (12.85 min) & 2.909 \\
Philosophy/Meta & 5,831 & 1.80 & 103.00 & 2868.78 (47.81 min) & 3760.96 (62.68 min) & 7.768 \\
Social/Casual & 189,765 & 10.95 & 4.52 & 24.83 & 106.58 & 0.593 \\
Spam/Low-Signal & 2,495 & 0.88 & 78.88 & 253.22 & 260.44 & 1.313 \\
\midrule
Overall & 223,316 & 9.60 & 4.55 & 50.05 & 132.23 & 0.691 \\
\bottomrule
\end{tabular}
\par}
\endgroup
\end{table}

The primary stratified pattern is incidence/speed heterogeneity. Social/Casual
has the highest incidence (10.95\%) and very fast conditional timing
(\(t_{90}=24.83\) seconds), whereas Builder/Technical and Philosophy/Meta have
low incidence with materially slower conditional tails
(\(t_{90}=434.87\) seconds [7.25 minutes] and \(2868.78\) seconds
[47.81 minutes], respectively).
Using category-cluster bootstrap intervals from
\texttt{paper/tables/moltbook\_results\_category\_uncertainty.csv}, key-category
uncertainty is: Social/Casual incidence 10.95\% [10.77\%, 11.13\%],
\(t_{50}=4.52\) [4.50, 4.54] seconds, \(t_{90}=24.83\) [16.77, 36.12] seconds,
pooled reciprocity 0.87\% [0.77\%, 0.96\%], and mean re-entry
0.213 [0.211, 0.216], median re-entry 0.200 [0.200, 0.200];
Philosophy/Meta incidence 1.80\% [1.27\%, 2.42\%],
\(t_{50}=103.00\) [67.67, 154.04] seconds,
\(t_{90}=2868.78\) [310.05, 4188.38] seconds
([5.17, 69.81] minutes), pooled reciprocity 1.44\% [0.92\%, 2.18\%], and mean
re-entry 0.108 [0.098, 0.119], median re-entry 0.000 [0.000, 0.000].
Full category-level uncertainty is provided in
Supplementary Material (S6).

\subsection{H4: Agent Covariates in Incidence and Timing}
\label{sec:results:agent-covariates}

Claim-status heterogeneity is reported directly in
\Cref{tab:reply-dynamics}: claimed accounts have higher incidence and
materially faster upper-tail conditional timing than unclaimed accounts.

\subsection{Secondary Context for H1b: Periodicity}
\label{sec:results:periodicity}

Effect size is weak: modulo-4-hour concentration is \(r=0.0308\)
(about 3.1\% concentration), far below the estimated 80\%-power detectability
threshold \(\kappa^\star=0.2\). Because the canonical timeline contains a
41.7-hour gap, periodicity is estimated on the longest contiguous segment only
(2026-02-02 04:20:50Z to 2026-02-04 19:51:53Z; 63.5 hours). On this segment
(\(N=220{,}461\) events), Rayleigh testing gives \(Z=209.57\), Monte Carlo
\(p=5\times 10^{-6}\), and circular mean phase 153.2 minutes; supplementary
PSD/AR(1) and bin-width diagnostics are directionally consistent with weak
aggregate coherence rather than a strong 4-hour line.

\subsection{Secondary Context: Reddit Baseline under Shared Estimators}
\label{sec:results:reddit-full-scale}

Secondary Reddit context is directionally consistent with the Moltbook
interpretation: deeper threads and slower reply persistence under the same
estimators. These baseline contrasts are descriptive and non-causal and are used
only as contextual evidence for H2 interpretation.

\subsection{Summary of Key Findings}
\label{sec:results:summary}

The primary decomposition readout is a 9.60\% direct-reply incidence
(95\% CI: [9.45\%, 9.76\%]) with very fast
conditional timing (\(t_{50}=4.55\) seconds, 95\% CI: [4.53, 4.58] seconds;
\(t_{90}=50.05\) seconds),
plus high short-window mass (8.47\% within 30 seconds and 9.41\% within
5 minutes unconditionally). The ECDF on a log-time axis
(\Cref{fig:reply-time-ecdf-logscale}) and observed-vs-fitted quantile
robustness checks both reinforce the same two-regime pattern.
Structural signatures are consistent with this regime: Moltbook threads are
predominantly star-shaped (mean maximum depth 1.38, median maximum depth 1,
effective depth-tail slope estimate \(\hat{s}_{\mathrm{depth}}=0.154\)), with rare
bidirectional dyads (0.998\%) and modest re-entry (mean 0.195; median 0.167).
Model-to-observable validation shows tight incidence calibration but
overprediction of deeper non-root branching tails, and one-parent-per-thread
robustness lowers pooled incidence while leaving conditional median speed
essentially unchanged. Topic and claim heterogeneity is substantial
(\Cref{tab:submolt-two-part,tab:reply-dynamics}), and cross-platform baseline
contrasts are descriptive context under shared estimators.
Taken together, these findings indicate a low-incidence/fast-conditional-response
persistence regime in this early Moltbook window.

\subsubsection{H1a Readout (Incidence + Conditional Speed)}
\label{sec:results:h1a}
H1a is supported by the low-incidence and very-fast-conditional-speed split in
\Cref{sec:results:decomposition,sec:results:two-regime}. The H1a readout is:
\(t_{50}=4.55\) seconds and \(t_{90}=50.05\) seconds; conditional short-window
mass \(\Pr(\text{reply}\le 30\mathrm{s}\mid\text{reply})=88.30\%\) and
\(\Pr(\text{reply}\le 5\mathrm{m}\mid\text{reply})=98.06\%\); and unconditional
mass \(\Pr(\text{reply}\le 30\mathrm{s})=8.47\%\) and
\(\Pr(\text{reply}\le 5\mathrm{m})=9.41\%\).

\subsubsection{H2 Readout (Structure vs Baseline + Reciprocity/Re-Entry)}
\label{sec:results:h2}
H2 is partially supported in this snapshot. Moltbook threads are shallow and
root-concentrated, with low pooled reciprocity and modest re-entry
(\Cref{sec:results:structure,fig:branching-by-depth,fig:reentry-distribution}).
Human-platform context under the same estimators points to deeper baseline
threads (\Cref{sec:results:reddit-full-scale}). The reciprocity/re-entry
cross-platform contrast is conditioning-sensitive, so this component is
interpreted descriptively.

\subsubsection{H3 Readout (Topic Moderation)}
\label{sec:results:h3}
H3 is partially supported in this snapshot. Topic moderation is clear for reply
incidence and conditional timing in
\Cref{sec:results:heterogeneity,tab:submolt-two-part} (for example,
Social/Casual versus Philosophy/Meta), with corresponding kernel half-life diagnostic
differences. Depth-related moderation appears in observed submolt tail
probabilities in the model-to-observable validation summary, but absolute
deep-tail levels are very small; we therefore interpret topic heterogeneity as
descriptive association.

\subsubsection{H4 Readout (Agent Covariates)}
\label{sec:results:h4}
H4 is partially supported in this snapshot. Claim-status covariates are
associated with large descriptive differences in incidence and conditional timing
(\Cref{sec:results:agent-covariates,tab:reply-dynamics}) and similar
claimed/unclaimed stratified patterns in the model-to-observable validation
summary. However, within-thread and repeated-author
dependence limits model-based precision for isolating the claim indicator
(\Cref{sec:results:decomposition}), and
follower-count effects are not reported as a dedicated main-text readout; the
agent-covariate closure is therefore incomplete for that component.
