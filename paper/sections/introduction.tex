\section{Introduction}
\label{sec:introduction}

The rapid advancement of large language models (LLMs) has enabled a new class of autonomous AI agents capable of sustained interaction with digital environments \citep{anthropic2024claude}. A striking manifestation of this capability is \emph{Moltbook}, a social network launched in January 2026 that restricts posting privileges to AI agents while permitting human observation \citep{willison2026moltbook}. Within weeks of its launch, the platform accumulated hundreds of thousands of agent accounts, generating rich datasets of agent-to-agent interaction unprecedented in scale and accessibility.

This emergence of agent-populated social platforms raises fundamental questions about collective AI behavior. Can autonomous agents sustain the extended, multi-turn dialogues necessary for meaningful collaboration? How do architectural constraints---particularly context-window limitations and periodic activation schedules---shape the temporal dynamics of agent discourse? And what distinguishes agent-driven conversations from human-driven ones in structural and temporal terms?

\subsection{Motivation and Research Questions}

Early observations of Moltbook revealed a striking pattern: agents appear substantially better at \emph{initiating} projects than \emph{sustaining} them \citep{alexander2026afterweekend}. Threads that begin with ambitious coordination proposals---collaborative research, collective governance, creative projects---frequently become ``graveyards of abandoned initiatives'' within days. This pattern suggests that agent social networks may face intrinsic \emph{persistence limitations} stemming from the temporal constraints under which individual agents operate.

We focus on a specific architectural feature that may explain these dynamics: the \emph{heartbeat mechanism}. Moltbook agents are typically configured to check the platform at regular intervals (approximately every four hours), creating a mechanically induced ``attention clock'' that synchronizes agent activity while limiting continuous engagement \citep{willison2026moltbook}. This design feature, combined with the finite context windows of underlying LLMs, suggests that agent conversations should exhibit characteristic temporal signatures distinct from human-driven platforms.

Our central research questions are:
\begin{enumerate}
    \item \textbf{Interaction half-life:} How quickly does conversational engagement decay on Moltbook, and does this decay exhibit the hypothesized $\sim$4-hour characteristic timescale?
    \item \textbf{Conversation geometry:} How do structural properties of discussion trees (depth, branching, reciprocity) differ between Moltbook and human platforms?
    \item \textbf{Heterogeneity and coordination:} Which factors---topic domain, agent reputation, early engagement---predict extended conversational persistence?
\end{enumerate}

\subsection{Hypotheses}
\label{sec:introduction:hypotheses}

Guided by the horizon-limited cascade framework (\Cref{sec:model}) and prior qualitative observations \citep{willison2026moltbook, alexander2026afterweekend}, we test the following hypotheses:
\begin{enumerate}
    \item \textbf{H1 (Short half-life and attention clock).} Moltbook threads exhibit a shorter interaction half-life than matched Reddit threads, and aggregate activity shows periodic structure near the heartbeat cadence ($\tau \approx 4$ hours).
    \item \textbf{H2 (Shallow, star-shaped structure).} Moltbook comment trees are shallower and more root-concentrated, with lower reciprocity and re-entry than matched Reddit threads.
    \item \textbf{H3 (Topic moderates persistence).} Persistence depends on topic/community: builder/technical submolts exhibit longer half-lives and deeper trees than low-signal or purely social submolts, controlling for early engagement.
    \item \textbf{H4 (Agent heterogeneity).} A small set of high-reputation agents disproportionately sustain long-lived threads, exhibiting lower decay rates and higher within-thread re-entry behavior.
\end{enumerate}

We operationalize these hypotheses via the metrics and estimators described in \Cref{sec:methods} and evaluate them in \Cref{sec:results}.

\subsection{Approach and Contributions}

We develop a \emph{horizon-limited interaction cascade} model that formalizes conversation dynamics on agent platforms. Our framework combines self-exciting point processes (Hawkes processes) with age-dependent branching dynamics, explicitly incorporating periodic availability modulation to capture the heartbeat mechanism. The model yields closed-form expressions linking platform design parameters to observable quantities: interaction half-life, expected conversation depth, and agent re-entry rates.

We validate this framework empirically using the Moltbook Observatory Archive \citep{simulamet2026observatoryarchive} and a run-scoped curated Reddit corpus. The present manuscript version reports: (1)~descriptive characterization of conversation geometry, (2)~estimation of temporal decay parameters via survival analysis, and (3)~a full-scale Reddit-side baseline analysis run (\texttt{attempt\_scaled\_20260206-142651Z}) using the same geometry/survival/periodicity estimators. Matched Moltbook--Reddit inferential comparisons remain future work.

Our contributions are as follows:

\paragraph{A portable metric for collective persistence.} We introduce \emph{interaction half-life} as a first-order descriptor of whether a multi-agent community can sustain coordination beyond single-shot posting. This metric is estimable from timestamped reply data and directly comparable across platforms.

\paragraph{Mechanism-grounded explanation.} We link observed shallow conversation depth to measurable temporal decay, connecting platform-level scheduling (heartbeat cadence) and individual-level constraints (context limits) to emergent collective behavior.

\paragraph{Design implications.} Our findings identify evidence-based levers---memory scaffolding, thread summarization, return-to-thread incentives---that could extend coordination horizons in agent-based systems.

\paragraph{Reproducible workflow.} We provide code and documentation to support replication (data export, manuscript source, run manifests, and analysis scripts), including run-scoped Reddit validation and analysis entrypoints.

\subsection{Paper Organization}

The remainder of this paper is organized as follows. \Cref{sec:background} reviews related work on information cascades, conversation modeling, and emerging research on AI agent systems. \Cref{sec:model} presents our formal model of horizon-limited interaction cascades. \Cref{sec:data} describes our datasets and preprocessing pipeline. \Cref{sec:methods} details our empirical methodology. \Cref{sec:results} presents our findings. \Cref{sec:discussion} interprets results and discusses implications for platform design. \Cref{sec:limitations} addresses limitations and ethical considerations. \Cref{sec:reproducibility} provides reproducibility details.
