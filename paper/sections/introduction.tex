\section{Introduction}
\label{sec:introduction}

The rapid advancement of large language models (LLMs) has enabled a new class of autonomous AI agents capable of sustained interaction with digital environments. A striking manifestation of this capability is \emph{Moltbook}, a social network launched in January 2026 that restricts posting privileges to AI agents while permitting human observation \citep{willison2026moltbook}. In this paper, an \emph{AI agent account} denotes an account whose posting and commenting actions are generated by an LLM-driven agent process rather than direct human operation. In the first archived week, the platform accumulated over 25{,}000 such agent accounts and 119{,}677 posts \citep{simulamet2026observatoryarchive}, generating rich datasets of agent-to-agent interaction unprecedented in scale and accessibility.

This emergence of agent-populated social platforms raises fundamental questions about collective AI behavior. Can autonomous agents sustain the extended, multi-turn dialogues necessary for meaningful collaboration? How do architectural constraints---particularly context-window limitations and periodic activation schedules---shape the temporal dynamics of agent discourse? And what distinguishes agent-driven conversations from human-driven ones in structural and temporal terms?

\subsection{Motivation and Research Questions}

Early observations of Moltbook revealed a striking pattern: agents appear substantially better at \emph{initiating} projects than \emph{sustaining} them \citep{alexander2026afterweekend}. Threads that begin with ambitious coordination proposals---collaborative research, collective governance, creative projects---frequently become what \citet{alexander2026afterweekend} termed ``a graveyard of abandoned projects'' within days. This pattern suggests that agent social networks may face intrinsic \emph{persistence limitations} stemming from the temporal constraints under which individual agents operate.

We focus on a specific architectural feature that may explain these dynamics: the \emph{heartbeat mechanism}. Moltbook agents are typically configured to check the platform at regular intervals (approximately every four hours), creating a mechanically induced ``attention clock'' \citep{willison2026moltbook}. In observatory data, however, per-account heartbeat schedules are latent: we observe timestamped posts/comments but not direct scheduler logs. We therefore treat heartbeat effects as a testable aggregate hypothesis: sufficiently synchronized check-ins should yield spectral concentration near 4 hours, whereas substantial dephasing or jitter can make the same mechanism weakly detectable or undetectable in aggregate over finite windows. This framing is paired with finite context windows, which motivate separate tests of rapid within-thread staleness.

Our central research questions are methodological: which estimands provide a
coherent decomposition of conversation persistence, how these estimands map to
minimal stochastic-process mechanisms, and how reliably they can be estimated
under heavy censoring and finite observation windows. Concretely, we target
reply incidence, conditional reply timing, depth-tail decay, and periodicity
detectability, then examine how these estimands vary by topic and agent
attributes. Thread duration is analyzed separately as a distinct outcome.

\subsection{Hypotheses}
\label{sec:introduction:hypotheses}

Guided by the horizon-limited cascade framework (\Cref{sec:model}) and prior
qualitative observations \citep{willison2026moltbook, alexander2026afterweekend},
we test four hypotheses. H1a posits short reply-kernel half-lives consistent with
architectural staleness constraints. H1b posits that heartbeat scheduling can
generate aggregate periodic structure near the hypothesized cadence
($\tau \approx 4$ hours) when check-ins are sufficiently synchronized; under
dephasing or jitter, aggregate detectability may be weak in finite samples.
H2 posits shallower, more root-concentrated Moltbook trees than human-platform
baselines, with lower reciprocity and conditioning-sensitive re-entry profiles; the
re-entry contrast is treated as conditioning-sensitive and may change direction
across overlap-restricted matched strata.
H3 posits topic-level moderation of persistence, including systematic differences
in half-life and depth across submolts. H4 posits that agent-level covariates
(account claim status and follower count) are associated with variation in reply
incidence and conversational persistence.

We operationalize these hypotheses via the estimands defined in \Cref{sec:model}
and the estimators in \Cref{sec:methods}, and evaluate them in \Cref{sec:results}.
For H1b, non-significant evidence at the target frequency is interpreted as
``not detected in this snapshot'' rather than as evidence of absence.

\subsection{Preview of Findings}

In this first-week snapshot, Moltbook conversations are strongly star-shaped,
with minute-scale reply-kernel decay, low direct-reply incidence, and minimal
reciprocity. Spectral analysis does not detect a statistically significant
4-hour periodic peak. A run-scoped Reddit baseline shows materially longer
persistence and deeper threads. Taken together, these patterns are consistent
with a ``fast response or silence'' regime driven by architectural constraints
on agent attention.

\subsection{Approach and Contributions}

We develop a \emph{horizon-limited interaction cascade} measurement framework
for agent-platform conversations. The framework combines self-exciting point
process dynamics with age-dependent branching and periodic availability
modulation, then maps these mechanisms to observable estimands:
reply incidence, conditional reply timing, depth-tail decay, and aggregate
periodic signatures.

We evaluate this framework empirically with a Moltbook-first design using the Moltbook
Observatory Archive \citep{simulamet2026observatoryarchive}. A run-scoped curated Reddit
corpus is used as secondary contextual baseline, not as a gating causal comparison. The
present manuscript reports: (1)~descriptive characterization of Moltbook conversation
geometry, (2)~estimation of Moltbook temporal decay parameters via survival analysis,
(3)~spectral tests for Moltbook periodic activity signatures, (4)~a full-scale Reddit-side
baseline analysis under the same estimators, and (5)~a coarse matched observational
comparison with paired effect estimation.

Our contributions are fourfold. First, we formalize a two-component persistence
measurement target that separates direct-reply incidence from conditional reply
timing, with half-life treated as a kernel-timescale summary. Second, we
provide a mechanism-to-estimand mapping that links staleness, branching, and
availability modulation to identifiable empirical quantities. Third, we
implement and calibrate an estimation workflow (survival-based timing
inference, depth-tail summaries, and periodicity tests) under a common
reproducible pipeline. Fourth, we derive downstream design implications as
model-consistent hypotheses, while keeping the primary contribution in
measurement and inference.

\subsection{Paper Organization}

The remainder of this paper is organized as follows. \Cref{sec:background}
reviews related work on stochastic-process modeling of interaction systems.
\Cref{sec:model} presents the minimal model and primary estimands.
\Cref{sec:data} describes data construction and preprocessing.
\Cref{sec:methods} details estimation and empirical procedures.
\Cref{sec:results} reports empirical findings and model-consistency checks.
\Cref{sec:discussion} interprets implications, including short downstream design
considerations. \Cref{sec:limitations} addresses limitations and ethical
considerations. \Cref{sec:conclusion} concludes with key takeaways and future
work. \Cref{sec:reproducibility} summarizes reproducibility details (Appendix
A), and \Cref{sec:appendix} provides supplementary derivations and operational
details (Appendix B).
